\documentclass{lecturenotes}

\usepackage[colorlinks,urlcolor=blue]{hyperref}
\usepackage{doi}
\usepackage{xspace}
\usepackage{agda}
\usepackage{fontspec}
\usepackage{enumerate}
\usepackage{mathtools}
\setsansfont{Fira Code}
\usepackage{newunicodechar}
\newunicodechar{∣}{\ensuremath{\mid}}

\newcommand{\agdanats}{\textsf{ℕ}\xspace}
\newcommand{\funcdef}[4]{#1 \colon #2 #3 #4}
\newcommand{\totfunc}[3]{\funcdef{#1}{#2}{\rightarrow}{#3}}
\newcommand{\parfunc}[3]{\funcdef{#1}{#2}{\rightharpoonup}{#3}}
\newcommand{\sem}[1]{\llbracket #1 \rrbracket}
\newcommand{\lift}[1]{#1_\bot}
\newcommand{\semapp}[2]{\sem{#1}(#2)}
\newcommand{\semappsig}[1]{\semapp{#1}{\sigma}}
\newcommand{\imp}[1]{\textsf{#1}}


\title{Denotational Semantics}
\coursenumber{CSE 410/510}
\coursename{Programming Language Theory}
\lecturenumber{11}
\semester{Spring 2025}
\professor{Professor Andrew K. Hirsch}

\begin{document}
\maketitle

\section{Motivation}
With operational semantics, we expressed evaluation in ``the way a computer does it".
However, in practice, we often want to consider our programs as being members in a mathematical space.
This provides us with several benefits:
\begin{itemize}
  \item We can apply abstractions that don't exist within programming.
  \item We can talk about mathematical objects rather than programs,
    which allows us to leverage potentially thousands of years of intuitions versus a few decades of programming intuitions.
\end{itemize}

As an example, functions in a programming language generally take in inputs of a certain type and produce outputs of a certain type.
Modeling functions in a programming language as mathematical functions gives us access to insights about functions as well as language-agnostic ways of speaking about concepts like recursion, higher-order functions, or currying.

\section{Denotational Semantics}
Let $a, b, c$ be any AExpr, BExpr, or CExpr, respectively.
We can define denotational semantics for each of them as
\begin{align*}
  \totfunc{\sem{a}&}{\text{state}}{\mathbb{N}}\\
  \totfunc{\sem{b}&}{\text{state}}{\textbf{2}}\\
  \parfunc{\sem{c}&}{\text{state}}{\text{state}}
\end{align*}

One should notice that the semantics $\sem{c}$ are defined as a partial function.
This is done because there are certain commands that produce non-terminating programs.
We will say $\semappsig{c} = \bot$ if the command does not terminate.
All that's left is to define each function.

\subsection{AExpr}
\[
  \semappsig{a} = \begin{cases}
    n & a = \widetilde{n}\\
    \sigma(x)  & a = x\\
    \semappsig{a_1} + \semappsig{a_2}  & a = a_1 + a_2\\
    \semappsig{a_1} - \semappsig{a_2}  & a = a_1 - a_2\\
    \semappsig{a_1} * \semappsig{a_2}  & a = a_1 * a_2
  \end{cases}
\]

\subsection{BExpr}
\[
  \semappsig{b} = \begin{cases}
   \imp{false} & b = \imp{false}\\
   \imp{true}  & b = \imp{true}\\
    \semappsig{b_1} \wedge \semappsig{b_2} & b = b_1\wedge b_2\\
    \semappsig{b_1} \vee \semappsig{b_2} & b = b_1\vee b_2\\
   \imp{true} & b = a_1 == a_2~\&~\semappsig{a_1} = \semappsig{a_2}\\
   \imp{false} & b = a_1 == a_2~\&~\semappsig{a_1} \ne \semappsig{a_2}\\
   \imp{true} & b = a_1 < a_2~\&~\semappsig{a_1} < \semappsig{a_2}\\
   \imp{false} & b = a_1 < a_2~\&~\semappsig{a_1} \nless \semappsig{a_2}\\
   \imp{true} & b = !b_1~\&~\semappsig{b_1} = \imp{false}\\
   \imp{false} & b = !b_1~\&~\semappsig{b_1} = \imp{true}\\
  \end{cases}
\]

\subsection{CExpr}
The semantics for commands follows a similar structure, with the only deviation occuring with the semantics for a $\imp{while}$ loop.

\[
  \semappsig{c} = \begin{cases}
    \sigma & c = \imp{skip}\\
    \sigma[x \mapsto \semappsig{e}]  & c = x \coloneq e \\
    \semapp{c_2}{\semappsig{c_1}} & c = c_1; c_2\\
    % \semapp{c_2}{\semappsig{c_1}} & c = c_1; c_2~\&~\semappsig{c_1} \neq \bot\\
    % \bot & c = c_1; c_2~\&~\semappsig{c_1} = \bot\\
    \sem{c_1}(\sigma) & c = \imp{if}~b~\imp{then}~c_1~\imp{else}~c_2~\&~\sem{b}(\sigma)~=~\imp{true}\\
    \sem{c_2}(\sigma) & c = \imp{if}~b~\imp{then}~c_1~\imp{else}~c_2~\&~\sem{b}(\sigma)~=~\imp{false}\\
    \text{fix}(\Gamma)(\sigma) & c = \imp{while}~b~\imp{do}~c'\\
  \end{cases}
\]

\noindent We can define $\Gamma$ as
\[
  \Gamma(f)(\sigma) = \begin{cases}
    f(\semappsig{c}) & \semappsig{b} = true \\
    \sigma  & \semappsig{b} = false \\
  \end{cases}
\]

\noindent This may make more sense by looking at an equivalent program.
\begin{gather*}
  w \equiv \imp{while}~b~\imp{do}~c \\
  w \sim \imp{if}~b~\imp{then}~c;w~\imp{else}~\imp{skip}
\end{gather*}

\noindent The semantics of both of these programs should be equivalent, which means $\sem{w} = \sem{\imp{if}~b~\imp{then}~c;w~\imp{else}~\imp{skip}}$.
However, it is hopefully obvious to see that

\[
  \semappsig{w} = \semappsig{\imp{if}~b~\imp{then}~c;w~\imp{else}~\imp{skip}} = \Gamma(\sem{w})(\sigma)
\]

\noindent This means $\sem{w}$ should be a fixed point of $\Gamma$.
\begin{defn}[Fixed Point]
  For a function $\totfunc{f}{X}{X}$, a fixed point of $f$ is an element $x \in X$ such that
  \[
    x = f(x)
  \]
\end{defn}

\noindent More specifically, we want $\sem{w}$ to be a \textit{least fixed-point} of $\Gamma$, which we denote as $\text{fix}(\Gamma)$.
If we have a least fixed point $x$ of a function $f$, given any other fixed point $y$ for $f$, we have
\[
  x \leq y 
\]

\subsubsection{Partial Function Ordering}
\noindent It's easy to see the type of $\totfunc{\Gamma}
                                                {(\text{state}\rightharpoonup\text{state})}
                                                {(\text{state}\rightharpoonup\text{state})}$.
How do we order partial functions so that we can find a least fixed-point?

\begin{defn}[Order for Partial Functions]
  Given $\parfunc{f,g}{X}{Y}$, $f \leq g$ if $\forall x,y$ $f(x) = y$, then $g(x) = y$.
\end{defn}

\subsubsection{fix(\Gamma) Examples}
Let's look at a few example programs to get a feel for $\text{fix}(\Gamma)$.
Take the following program
\[
  \sem{\imp{while}~\imp{true}~\imp{do}~\imp{i}\coloneq\imp{i+1}}
\]

We can write $\Gamma$ as
\[
  \Gamma(f)(\sigma) = \begin{cases}
    f(\sigma[i \mapsto \sigma(i) + 1]) & true \\
    \sigma & false
  \end{cases}
\]

\noindent Our semantics require a least fixed point, so the smallest $g$ such that $g(\sigma) = \Gamma(g)(\sigma)$.
A good candidate is $g(\sigma)~=~\bot$.
We therefore have $\Gamma(g)(\sigma) = g(\sigma) = \bot$.
There is no partial function less than $g$, so this is a least fixed-point.
Intuitely, this means the semantics of the loop are that it never terminates.
(Try to convince yourself that another function $h$ doesn't work as a fixed point.)

\noindent For another example, let's step through the semantics for the following program.
\[
  p = \left\llbracket
  \begin{aligned}
    & \imp{i}~\coloneq \imp{0};\\
    & \imp{while}~\imp{i}<5~\imp{do}~\imp{i}\coloneq~\imp{i+1}
  \end{aligned}
  \right\rrbracket
\]
Taking $w = \imp{while}~\imp{i}<5~\imp{do}~\imp{i}\coloneq~\imp{i+1}$, we compute
\begin{align*}
  \semappsig{p} & = \semapp{w}{\semappsig{\imp{i}\coloneq~\imp{0}}}\\
  & = \semapp{w}{\sigma[i \mapsto 0]}\\
  & = \text{fix}(\Gamma)(\sigma[i \mapsto 0])
\end{align*}
As a reminder, we have
\[
  \Gamma(f)(\sigma) = \begin{cases}
    f(\sigma[i \mapsto \sigma(i) + 1]) & \sigma(i) < 5 \\
    \sigma & \sigma(i) \ge 5
  \end{cases}
\]
and we want an $f$ such that $f = \Gamma(f)$.
A valid choice for $f$ is then
\[
  g(\sigma) = \begin{cases}
    \sigma[i \mapsto 5] & \sigma(i) < 5\\
    \sigma & \sigma(i) \ge 5
  \end{cases}
\]
It can be easily verified that $g = \Gamma(g)$, and therefore $g$ is a fixed-point for $\Gamma$.
Choose a state $\sigma$ such that $\sigma(i) < 5$.
\[
  g(\sigma) = \sigma[i \mapsto 5] = g(\sigma[i \mapsto \sigma(i) + 1]) = \Gamma(g)(\sigma)
\]
Now choose a state such that $\sigma(i) \ge 5$.
\[
  g(\sigma) = \sigma = \Gamma(g)(\sigma)
\]
We ignore the details of why, but it is also the least-fixed point, meaning fix($\Gamma$) = $g$.
Rewriting our original semantics with this information gives us
\begin{align*}
  \semappsig{p} & = \semapp{w}{\semappsig{\imp{i}\coloneq~\imp{0}}}\\
  & = \semapp{w}{\sigma[i \mapsto 0]}\\
  & = \text{fix}(\Gamma)(\sigma[i \mapsto 0])\\
  & = g(\sigma[i \mapsto 0])\\
  & = \sigma[i \mapsto 5]
\end{align*}
In other words, the semantics of this program are the same as having a state where $\imp{i}$ is 5.
\end{document}
