\documentclass{lecturenotes}

\usepackage[colorlinks,urlcolor=blue]{hyperref}
\usepackage{doi}
\usepackage{xspace}
\usepackage{agda}
\usepackage{fontspec}
\usepackage{enumerate}
\usepackage{mathpartir}
\setsansfont{Fira Code}
\usepackage{newunicodechar}
\newunicodechar{∣}{\ensuremath{\mid}}
\newunicodechar{′}{\ensuremath{{}^\prime}}
\newunicodechar{ˡ}{\ensuremath{{}^{\textsf{l}}}}
\newunicodechar{ʳ}{\ensuremath{{}^{\textsf{r}}}}
\newunicodechar{≤}{\ensuremath{\mathord{\leq}}}
\newunicodechar{≡}{\ensuremath{\mathord{\equiv}}}
\newunicodechar{≐}{\ensuremath{\mathord{\doteq}}}
\newunicodechar{∘}{\ensuremath{\circ}}
\newunicodechar{≃}{\ensuremath{\simeq}}
\newunicodechar{≲}{\ensuremath{\precsim}}
\newunicodechar{⊎}{\ensuremath{\uplus}}

\newcommand{\agdanats}{\textsf{ℕ}\xspace}

\newcommand{\nil}{\ensuremath{\textsf{[]}}}
\newcommand{\cons}{\ensuremath{\mathbin{\textsf{::}}}}
\newcommand{\app}{\ensuremath{\mathbin{\textsf{++}}}}
\newcommand{\singlestep}{\ensuremath{\rightarrow}}
\newcommand{\multistep}{\ensuremath{\rightarrow^*}}
\newcommand{\bigstep}{\ensuremath{\Downarrow}}
\newcommand{\config}[1]{\ensuremath{\textlangle #1, \sigma \textrangle}}
\newcommand{\configp}[1]{\ensuremath{\textlangle #1, \sigma\prime \textrangle}}
\newcommand{\prog}[1]{\ensuremath{\widetilde{#1}}}
\newcommand{\case}[3]{
  \textbf{Case #1:} #2
  \begin{quote}
    #3
  \end{quote}
}
\newcommand{\casesc}[3]{
  \textbf{Case} \textsc{#1}\textbf{:} #2
  \begin{quote}
    #3
  \end{quote}
}

\title{Big-Step Operational Semantics}
\coursenumber{CSE 410/510}
\coursename{Programming Language Theory}
\lecturenumber{10}
\semester{Spring 2025}
\professor{Professor Andrew K. Hirsch}

\begin{document}
\maketitle

\section{Executive Summary}
\label{sec:executive-summary}

We previously covered small step semantics, however, we only covered the \textbf{single step relation (\singlestep)}. \\

\noindent
In this lecture introduces two \emph{multi-step} semantic relations:
\begin{itemize}
\item The \textbf{multi-step relation (\multistep)} (part of small-step semantics)
\item The \textbf{big-step relation (\bigstep)}
\end{itemize}

\noindent
The multi-step relation is constructed from the single-step relation using the \textbf{Reflexive-Transitive Closure}. \\

\noindent
Building on these relations, three theorems are introduced showing small-step semantics and big-step semantics are of equivalent power:
\begin{itemize}
\item \textbf{One-Step Completeness}
\item \textbf{Big-Step Completeness}
\item \textbf{Big-Step Soundness}
\end{itemize}

\section{The Multi-Step Relation (\multistep)}
\label{sec:multi-step-relation}

A ``reflexive-transitive closure'' is a construct from set theory that we use to create the \multistep relation from the \singlestep relation. \\

\noindent
$R^*$ is the notation for the reflexive-transitive closure of a relation, $R$, and is defined below:
\noindent
\begin{mathpar}
  \infer*[left=R-refl]{ }{xR^*x}
  \and
  \infer*[left=R-step]{xRy \\ yR^*z}{xR^*z}
\end{mathpar} \\

\noindent
If we let $R = \singlestep $ for single-step (\singlestep), then multi-step (\multistep) is the reflexive-transitive closure of $R$, $R^* = \multistep $:
\begin{mathpar}
  R \mapsto R^*
  \and
  \singlestep \, \mapsto \, \multistep
\end{mathpar} \\

\noindent
Breaking it down, the reflexive-transitive closure $R^*$ is a super set of $R$ such that for every ``chain'' of relations in $R$:
$$x_{0}Rx_{1}, x_{1}Rx_{2}, ..., x_{n-1}Rx_{n}$$
$R^*$ also contains the relation:
$$x_{0}R^*x_{n}$$
For \singlestep\, the reflexive-transitive closure, \multistep, can be interpreted as the set of every valid sequence of single steps that a program can take. \\

\noindent
Later in lecture, we show \multistep\ and \bigstep\ are equivalent.

\section{The Big-Step Relation (\bigstep)}

Unlike the \singlestep\ relation, which describes the sequence of single steps needed for an expression to become a value, the \bigstep\ relation directly encodes the value that a given expression will become. \\

\noindent
Unlike in \singlestep, constants such as $\prog{n}$, are directly evaluated to values.
\begin{mathpar}
  \inferrule* [left=const]
  { }
  {\config{\prog{n}} \bigstep n}
  \and
  \inferrule* [left=var]
  { }
  {\config{x} \bigstep \sigma(x)}
  \and
  \inferrule* [left=plus]
  {\config{e_1} \bigstep n \\ e_2,\sigma\bigstep m}
  {\config{e_1 + e_2} \bigstep n + m}
  \and
  \inferrule* [left=minus]
  {\config{e_1} \bigstep n \\ \config{e_2} \bigstep m}
  {\config{e_1 - e_2} \bigstep n - m}
\end{mathpar}
The rules for plus and minus can be generalized:
\begin{mathpar}
  \inferrule* [left=generalized]
  {\config{e_1} \bigstep n \\ e_2,\sigma\bigstep m \\ \odot\in\{+,-,*\}}
  {\config{e_1 \odot e_2} \bigstep n \odot m}
\end{mathpar}

\section{Big-Step Soundness}

\noindent
\textbf{Theorem:}
$$\forall e,\sigma,n \textbf{ if } e,\sigma \bigstep n \textbf{ then } e,\sigma \multistep \prog{n}$$
That is, if you reason about the result of an expression using \bigstep, then you will always get the same result using \singlestep.
NOTE: This theorem does not show that this is true the other way around, that is what the big-step completeness theorem does. \\

\noindent
\textbf{Proof:} By induction on the rules of \bigstep.
\begin{quote}

  \casesc{const}
  {$e = n$.}
  {
    The reflexive property of \multistep proves that $\prog{n},\sigma \multistep \prog{n}$.
  }

  \casesc{var}
  {$e = x$ and $n = \sigma(x)$.}
  {
    $\config{x} \singlestep \sigma(x) = \prog{n}$.
    Since \multistep contains \singlestep, $\config{x} \multistep \sigma(x) = \prog{n}$.
    NOTE: we are to prove that \multistep\ contains \singlestep\ in homework 2.
  }

  \casesc{plus,times,minus}
  {$e = e_1 \odot e_2$ where $\odot \in \{+,-,*\}$ and $\config{e_1} \bigstep n$ and $\config{e_2} \bigstep m$.}
  {
    By the Inductive Hypothesis, $\config{e_1} \multistep \prog{n}$ and $\config{e_2} \multistep \prog{m}$:
    \begin{quote}
      $e_1 \odot e_1 \multistep \prog{n} \odot e_2 \multistep \prog{n} \odot \prog{m} \singlestep \prog{n \odot m}$ \\
    \end{quote}
    This chain of single/multi-step relations can be collapsed via transitivity, thus: \\
    $e_1 \odot e_2 \multistep \prog{n \odot m}$.
  }

  \qed
\end{quote}

\section{Single-Step Completeness}

\textbf{Theorem:}
$$\forall e_1,e_2,\sigma,n \textbf{ if [1] } \config{e_1} \singlestep e_2 \textbf{ and [2] } \config{e_2} \bigstep n \textbf{ then } \config{e_1} \bigstep n$$

\noindent
\textbf{Proof:} By induction on the rules of \singlestep.

\begin{quote}
  \casesc{var}
  {$e_1 = x$ and $e_2 = \prog{\sigma(x)}$.}
  {
    For premise [2], the only \bigstep\ rule that works is the \textsc{const} rule:
    \begin{mathpar}
      \inferrule* [left=const]
      { }
      {\config{\prog{n}} \bigstep n}
    \end{mathpar}
    thus we get $\config{x} \bigstep \sigma(x)$. \\

    By premise [2], $\config{\prog{\sigma(x)}} \bigstep \sigma(x)$. \\

    From [1] and [2], it follows that $\config{x} \bigstep \sigma(x)$
  }
  
  \case{Left-Hand Side}
  {
    \ \\
    $e_1 = e_{11} \odot e_{12}$ where $\odot \in \{+,-,*\}$  and \\
    $e_2 = e_{21} \odot e_{12}$ where $\config{e_{11}} \singlestep e_{21}$ and \\
    $\config{e_{21}} \bigstep n$ and $\config{e_{12}} \bigstep m$ and $\config{e_2} \bigstep n \odot m$.
  }
  {
    By the Inductive Hypothesis:
    \begin{quote}
      For $e_1$, $\config{e_{11}} \bigstep n$. Thus $\config{e_{11} \odot e_{12}} \bigstep n \odot m$. \\
    \end{quote}
  }

  \case{Right-Hand Side} {}
  {Similar to Left-Hand Side.}

  \case{Operator}
  {$e_1 = \prog{n} \odot \prog{m}$ and $e_2 = \prog{n \odot m}$.}
  {
    \config{$\prog{n} \odot\ \prog{m}} \bigstep n \odot m$ and $\config{\prog{n \odot m}} \bigstep n \odot m$.
  }

  \qed
\end{quote}

\section{Big-Step Completeness}

\textbf{Theorem:}
$$\forall e,\sigma,n \textbf{ if } \config{e} \multistep \prog{n} \textbf{ then } \config{e} \bigstep n$$

\textbf{Proof:} By induction on \multistep.

\begin{quote}
  \case{refl}
  {$e = \prog{n}$}
  {
    $\config{n} \bigstep n$
  }

  \case{step}
  {$\config{e} \singlestep e\prime$ and $\config{e\prime} \multistep \prog{n} $}
  {
    By the Inductive Hypothesis: $\config{e\prime} \bigstep n$
    \begin{quote}
      By single step completeness $\config{e} \bigstep n$.
    \end{quote}
  }

  \qed
\end{quote}

\end{document}

%%% Local Variables:
%%% mode: latex
%%% TeX-master: t
%%% TeX-engine: luatex
%%% TeX-command-default: "Make"
%%% End:
