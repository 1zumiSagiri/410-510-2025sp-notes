\documentclass{lecturenotes}

\usepackage[colorlinks,urlcolor=blue]{hyperref}
\usepackage{doi}
\usepackage{xspace}
\usepackage{agda}
\usepackage{fontspec}
\usepackage{enumerate}
\usepackage{mathpartir}
\setsansfont{Fira Code}
\usepackage{newunicodechar}
\newunicodechar{∣}{\ensuremath{\mid}}
\usepackage{stmaryrd}

\newcommand{\agdanats}{\textsf{ℕ}\xspace}

\newunicodechar{∣}{\ensuremath{\mid}}
\newunicodechar{′}{\ensuremath{{}^\prime}}
\newunicodechar{ˡ}{\ensuremath{{}^{\textsf{l}}}}
\newunicodechar{ʳ}{\ensuremath{{}^{\textsf{r}}}}
\newunicodechar{ᵣ}{\ensuremath{{}_{\textsf{r}}}}
\newunicodechar{⊗}{\ensuremath{\otimes}}
\newunicodechar{∷}{\ensuremath{\mathrel{::}}}
\newunicodechar{∀}{\ensuremath{\textsf{\forall}}}
\newunicodechar{∎}{\ensuremath{\textsf{\blacksquare}}}
\newunicodechar{ℕ}{\ensuremath{\mathbb{N}}}
\newunicodechar{≤}{\ensuremath{\textsf{\le}}}

\title{Axiomatic Semantics}
\coursenumber{CSE 410/510}
\coursename{Programming Language Theory}
\lecturenumber{12}
\semester{Spring 2025}
\professor{Professor Andrew K. Hirsch}

\begin{document}
\maketitle

\section{Axiomatic Semantics and Hoare Logic}

Axiomatic semantics and Hoare Logic are similar concepts where Hoare Logic is a specific form of axiomatic semantics.
Axiomatic semantics refers to a way of giving semantics to a programming language.
They describe the assertions (invariants) that hold true about its state before and after executing a body of code.
Hoare Logic uses a system called a Hoare Triple in order to portray this idea.
A Hoare triple takes the form: \{ $P$ \} $c$ \{ $Q$ \} , where P and Q are of type: state \rightarrow \; set.
$P$ and $Q$ represent loop invariants, and $c$ is the loop body.
A loop invariant is a condition that holds before and after each iteration of a loop. \newline \newline

\section{An Example}

Statements inside of curly braces are loop invariants. \newline \newline
\{ true \} \newline
$x$ := $\tilde{n}$ ; $y$ := $\tilde{m}$ ; $z$ := $\tilde{o}$; \newline
while $o$ < $y'$ do \newline
$z$ := $z$ + $x$ ; \newline
$y$ := $y$ - $\tilde{1}$ \newline
\{ $z$ = $nm$ \} \newline
(computes the value of $z$ = $x$ * $y$, in other words, adding x to z, y times) \newline
\newline
\{ true \} (true is a preimposed condition since it must hold prior to running a body) \newline
$x$ := $\tilde{n}$ \newline
\{ $x$ = $n$ \} \newline
$y$ := $\tilde{m}$ \newline
\{ $x$ = $n$ \wedge \;$y$ = $m$ \} \newline
$z$ := $\tilde{o}$ \newline
\{ $x$ = $n$ \wedge \;$y$ = $m$ \wedge \;$z$ = $o$ \} \newline
\underline{Loop Invariant:} $z$ = $n$ ($m$ - $y$) \wedge \;$o$ \le \;$y$ \le \;$m$ \wedge \;$x$ = $n$ \rightarrow \; true \newline
\newline
while $o$ < $y$ do \newline
\{ $z$ = $n$ ($m$ - $y$) \wedge \;$o$ \le \;$y$ \le \;$m$ \wedge \;$x$ = $n$ \wedge \;$o$ < $y$ \} \newline
$z$ := $z$ + $x$ ; \newline
\{ $z$ = ($n$ + 1) ($m$ - $y$) \wedge \;$o$ < $y$ \le \;$m$ \wedge \;$x$ = $n$ \} \newline
$y$ := $y$ - $\tilde{1}$ ; \newline
\{ $z$ = $n$ ($m$ - $y$) \wedge \;$o$ \le \;$y$ \le \;$m$ \wedge \;$x$ = $n$ \} \newline
\newline
\{ $z$ = $n$ ($m$ - $y$) \wedge \;$o$ = $y$ \wedge \;$x$ = $n$ \} \newline
\newline
\{ $z$ = $nm$ \} \newline \newline

\section{Booleans and Hoare Triples}

As mentioned earlier, a Hoare triple takes the form: \{ $P$ \} $c$ \{ $Q$ \} , where P and Q are of type: state \rightarrow \; set.
$P$ and $Q$ represent loop invariants, and $c$ is the loop body. $P$ and $Q$ can be described as predicates of type: state \rightarrow \; set.
We can write boolean expressions about the state, \sigma \; , and its relations to the predicates, $P$ and $Q$. \newline
\newline
$\uparrow$ : Bexpr $\rightarrow$ State $\rightarrow$ Set \newline
($\uparrow$ b) $(\sigma)$ = $\llbracket$ $b$ $\rrbracket$ $(\sigma)$ \newline
\newline
$\uparrow$ takes a boolean expression and gives us back a predicate about states. \newline
\newline
$(P \wedge Q)$ $(\sigma)$ = $P$ $(\sigma)$ $\wedge$ $Q$ $(\sigma)$ \newline
If $P$ and $Q$ hold about state $\sigma$ then $P$ holds for $\sigma$ and $Q$ hold for $\sigma$ \newline
\newline
$P \Rightarrow Q$ = $\forall$ $\sigma$. $P$ $(\sigma)$ $\Rightarrow$ $Q$ $(\sigma)$ \newline
$P$ implies $Q$ is equal to saying that for all states $\sigma$, $P$ of $\sigma$ implies $Q$ of $\sigma$ \newline
\newline
$P$ [$x$ $\mapsto$ $f$] $(\sigma)$ $\stackrel{\triangle}{\iff}$ $P$ ($\sigma$ [$x$ $\mapsto$ $f(\sigma)$]) \newline \newline


\section{Inductive Defenition of Hoare Logic}

\begin{mathpar}
  \infer*[left=Reasoning]{$R$ \Rightarrow $P$ \\ $Q$ \Rightarrow $S$ \\ \vdash \{ $P$ \} \;c \; \{ $Q$ \}}{\vdash \{ $R$ \} \; $c$ \; \{$S$ \}}
\end{mathpar}

\begin{mathpar}
  \infer*[left=Skip]{{}}{\vdash \{ $P$ \} \; Skip \; \{ $P$ \}}
\end{mathpar}

\begin{mathpar}
  \infer*[left=Assgn]{{}}{\{ $P$ \; [$x$ \; \mapsto \llbracket \; $e$ \; \rrbracket] \}$x$ \; := \; $e$ \; \{ $P$ \}}
\end{mathpar}

\begin{mathpar}
\infer*[left=Seq]{\vdash \{ $P$ \} \; c_1 \; \{ $Q$ \} \\ \vdash \{ $Q$ \} \; c_2 \; \{ $R$ \}}{\vdash \{ $P$ \} \; c_1 \; ; c_2 \; \{ $R$ \}}
\end{mathpar}

\begin{mathpar}
  \infer*[left=if]{\vdash \{ $P$ \; \wedge \;\uparrow b \} \; c_1 \; \{ $Q$ \} \\ \vdash \{ $P$ \; \wedge \;\uparrow \neg \; b \} \; c_2 \; \{ $Q$ \}}
  {\vdash \{ $P$ \} \; if \; $b$ \; then \; c_1 \; else \; c_2 \; \{ $Q$ \}}
\end{mathpar}

\begin{mathpar}
\infer*[left=while]{\vdash \{ $P$ \; \wedge \;\uparrow b \} \; $c$ \; \{ $P$ \}}{\vdash \{ $P$ \} \; while \; $b$ \; do \; $c$ \; \{ $P$ \; \wedge \;\uparrow \neg \; b \}}
\end{mathpar}
\newline \newline

\section{Axiomatic Soundness}

If $\vdash$ \{$P$\} $c$ \{$Q$\} and $P$ $(\sigma)$ \newline
and $\llbracket$ $c$ $\rrbracket$ $(\sigma)$ = $\sigma'$ \newline
Then $Q$ $(\sigma')$

\end{document}

%%% Local Variables:
%%% mode: latex
%%% TeX-master: t
%%% TeX-engine: luatex
%%% TeX-command-default: "Make"
%%% End:
