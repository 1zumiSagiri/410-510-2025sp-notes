\documentclass{lecturenotes}

\usepackage[colorlinks,urlcolor=blue]{hyperref}
\usepackage{doi}
\usepackage{xspace}
\usepackage{fontspec}
\usepackage{enumerate}
\usepackage{mathpartir}
\usepackage{pl-syntax/pl-syntax}
\usepackage{forest}
\usepackage{stmaryrd}
\usepackage{epigraph}
\usepackage{xspace}

\setsansfont{Fira Code}

\newcommand{\abs}[2]{\ensuremath{\lambda #1.\,#2}}
\newcommand{\tabs}[3]{\ensuremath{\lambda #1 \colon #2.\,#3}}
\newcommand{\dbabs}[1]{\ensuremath{\lambda.\,#1}}
\newcommand{\dbind}[1]{\ensuremath{\text{\textasciigrave}#1}}
\newcommand{\app}[2]{\ensuremath{#1\;#2}}
\newcommand{\utype}{\textsf{unit}\xspace}
\newcommand{\unit}{\ensuremath{\textsf{(}\mkern0.5mu\textsf{)}}}
\newcommand{\prodtype}[2]{\ensuremath{#1 \times #2}}
\newcommand{\pair}[2]{\ensuremath{(#1, #2)}}
\newcommand{\projl}[1]{\ensuremath{\pi_1\mkern2mu#1}}
\newcommand{\projr}[1]{\ensuremath{\pi_2\mkern3mu#1}}
\newcommand{\sumtype}[2]{\ensuremath{#1 + #2}}
\newcommand{\injl}[1]{\ensuremath{\textsf{inj}_1\mkern2mu#1}}
\newcommand{\injr}[1]{\ensuremath{\textsf{inj}_2\mkern3mu#1}}
\newcommand{\case}[5]{\ensuremath{\textsf{case}\mkern5mu#1\mkern5mu\textsf{of}\mkern5mu\injl{#2} \Rightarrow #3;\mkern5mu\injr{#4} \Rightarrow #5\mkern5mu\textsf{end}}}
\newcommand{\vtype}{\textsf{void}\xspace}
\newcommand{\vcase}[1]{\ensuremath{\textsf{case}\mkern5mu#1\mkern5mu\textsf{of}\mkern5mu\textsf{end}}}
\newcommand{\rectype}[2]{\ensuremath{\mu #1.\,#2}}
\newcommand{\roll}[1]{\textsf{roll}\mkern2mu#1}
\newcommand{\unroll}[1]{\textsf{unroll}\mkern2mu#1}
\newcommand{\fatype}[2]{\ensuremath{\forall #1.\,#2}}
\newcommand{\Abs}[2]{\Lambda #1.\,#2}
\newcommand{\App}[2]{#1\;[#2]}
\newcommand{\extype}[2]{\ensuremath{\exists #1.\,#2}}
\newcommand{\pack}[3]{\ensuremath{\textsf{pack}\mkern5mu#1\mathrel{\textsf{as}}#2\mathrel{\textsf{in}}#3}}
\newcommand{\unpack}[4]{\ensuremath{\textsf{unpack}\mkern5mu#1\mathrel{\textsf{as}} #2, #3 \mathrel{\textsf{in}} #4}}

\newcommand{\FV}{\text{FV}}
\newcommand{\BV}{\text{BV}}

\newcommand{\toform}[1]{\ensuremath{\lceil #1 \rceil}}
\newcommand{\totype}[1]{\ensuremath{\lfloor #1 \rfloor}}

\newcommand{\neutral}[1]{#1\;\text{ne}}
\newcommand{\nf}[1]{#1\;\text{nf}}

\newcommand{\subtype}{\ensuremath{\mathrel{\mathord{<}\mathord{:}}}}

\title{Call-By-Value Logical Relations}
\coursenumber{CSE 410/510}
\coursename{Programming Language Theory}
\lecturenumber{24}
\semester{Spring 2025}
\professor{Professor Andrew K. Hirsch}

\begin{document}
\maketitle

\begin{itemize}
\item Last week, we saw how we can use \emph{logical relations} to prove that STLC terms fully normalize under full $\beta$ reduction.
\item However, we want to know something more: we want to know that in CBV $\lambda$-calculus, all terms terminate with a \emph{value}.
\item Technically, we don't know that this holds, because we may need to rewrite under $\lambda$s in order to obtain a normal form.
  We know that, for values, it doesn't matter what the body of a function is, but does that help at all?
\item Instead of trying to massage the last proof, we are going to build a new logical-relations proof.
\item In fact, when dealing with a CBV language, there's a nice structure we can put on logical relations:
  instead of just writing one relation, we write a \emph{value} relation and an \emph{expression} relation.
\item Then, we can constrain the \emph{shape} of the values.
\item This lets us get away without explicitly mentioning all of the elimination forms, while still ensuring that everything can eliminate correctly.
\item Let's see an example:
\end{itemize}

$$
\begin{array}{r@{\;\triangleq\;}l}
  \mathcal{V}(\utype)(v) & v = \unit\\
  \mathcal{V}(\vtype)(v) & \bot\\
  \mathcal{V}(\prodtype{\tau_1}{\tau_2})(v) & \exists v_1, v_2.\, v = (v_1, v_2) \land \mathcal{V}(\tau_1)(v_1) \land \mathcal{V}(\tau_2)(v_2)\\
  \mathcal{V}(\sumtype{\tau_1}{\tau_2})(v) & \exists v'.\, (v = \injl{v'} \land \mathcal{V}(\tau_1)(v')) \lor (v = \injr{v'} \land \mathcal{V}(\tau_2)(v'))\\
  \mathcal{V}(\tau_1 \to \tau_2)(v) & \exists x, e.\, v = \tabs{x}{\tau_1}{e} \land \forall v'.\, \mathcal{V}(\tau_1)(v') \to \mathcal{E}(\tau_2)(e[x \mapsto v'])\\
  \mathcal{E}(\tau)(e) & \exists v.\, e \to^\ast v \land \mathcal{V}(\tau)(v)
\end{array}
$$

\begin{itemize}
\item Here $\mathcal{V}(\tau)(v)$ means ``$v$ acts like a value of type $\tau$,'' while $\mathcal{E}(\tau)(e)$ means ``$e$ acts like an expression of type $\tau$.''
\item It's easy to verify that $\mathcal{V}(\tau)(v) \to \mathcal{E}(\tau)(v)$.
  This makes sense: something that acts like a value of type~$\tau$ should also act like an expression of type~$\tau$.
  After all, values are expressions.
\item It's also easy to verify that if $\mathcal{E}(\tau)(e)$, then $e \to^\ast v$ where $v$ is a value.
  Note that we did not have to repeat it for all of the cases of the value function; instead, it's only in the expression relation.
\item Note also that we did not have to talk about \textsf{case} expressions in the case for sum types.
  Instead, we can just constrain $v$ to be either an $\textsf{inj}_1$ or a $\textsf{inj}_2$.
  \begin{itemize}
  \item \textbf{Stop and Think:} Why can we do this now, but not before?
  \item Because variables can be normal forms.
    But values can only have variables under $\lambda$s, so we know that there can't be a variable before a constructor here.
  \end{itemize}
\item We again write $$\mathcal{C}(\Gamma)(\gamma) \triangleq \forall x : \tau \in \Gamma.\, \mathcal{V}(\tau)(\gamma(x))$$
  Here, $\gamma$ is a value substitution---it maps all variables to values.
\item We can then again define $$\Gamma \vDash e : \tau \triangleq \forall \gamma.\,\mathcal{C}(\Gamma)(\gamma) \to \mathcal{E}(\tau)(e)$$
\item We will spend most of the rest of the lecture proving the fundamental theorem.
\end{itemize}

\newpage
\begin{thm}[Fundamental Theorem]
  If $\Gamma \vdash e : \tau$, then $\Gamma \vDash e : \tau$.
\end{thm}
\begin{proof}
  By induction on the evidence of $\Gamma \vdash e : \tau$.

  \noindent \textbf{Case \textsc{Axiom}:}
  In this case, $e = x$ and $x : \tau \in \Gamma$.
  Now, let $\gamma$ be such that $\mathcal{C}(\Gamma)(\gamma)$.
  Then $e[\gamma] = \gamma(x)$ and $\mathcal{V}(\tau)(\gamma(x))$, so $\mathcal{E}(\tau)(\gamma(x))$, as desired.

  \noindent\textbf{Case \textsc{Unit-I}:}
  In this case, $e = \unit$.
  Let $\gamma$ be such that $\mathcal{C}(\Gamma)(\gamma)$.
  Then $e[\gamma] = \unit[\gamma] = \unit$.
  But $\mathcal{V}(\utype)(\unit)$, so $\mathcal{E}(\utype)(\unit)$ as desired.

  \noindent\textbf{Case \textsc{Void-E}:}
  In this case, $e = \vcase{e'}$ and $\Gamma \vdash e' : \vtype$.
  By IH, then, $\Gamma \vDash e' : \vtype$.
  Now, let $\gamma$ be such that $\mathcal{C}(\Gamma)(\gamma)$.
  Then $\mathcal{E}(\vtype)(e'[\gamma])$ by IH, and so there is a $v$ such that $e'[\gamma] \to^\ast v$ and $\mathcal{V}(\vtype)(v)$.
  But this is impossible, so we are done.

  \noindent\textbf{Case \textsc{$\to$-I}:}
  In this case, $\tau = \tau_1 \to \tau_2$ and $e = \tabs{x}{\tau_1}{e'}$ where $\Gamma, x : \tau_1 \vdash e' : \tau_2$.
  Now, let $\gamma$ be such that $\mathcal{C}(\Gamma)(\gamma)$.
  Then $e[\gamma] = (\tabs{x}{\tau_1}{e'})[\gamma] = \tabs{x}{\tau_1}{(e'[\gamma])}$.
  Moreover, let $v$ be such that $\mathcal{V}(\tau_1)(v)$.
  Then $\mathcal{C}(\Gamma, x : \tau)(\gamma[x \mapsto v])$.
  But, $e'[\gamma][x \mapsto v] = e'[\gamma[x \mapsto v]]$ and so $\mathcal{E}(\tau_2)(e'[\gamma][x \mapsto v])$ by IH.
  But then $\mathcal{V}(\tau_1 \to \tau_2)(e)$, and thus $\mathcal{E}(\tau_1 \to \tau_2)(e)$ as desired.

  \noindent\textbf{Case \textsc{$\to$-E}:}
  In this case, $e = \app{f}{a}$ where $\Gamma \vdash f : \tau' \to \tau$ and $\Gamma \vdash a : \tau'$ for some $\tau'$.
  By IH, $\Gamma \vDash f : \tau' \to \tau$ and $\Gamma \vDash a : \tau'$.
  Let $\gamma$ be such that $\mathcal{C}(\Gamma)(\gamma)$.
  Then $e[\gamma] = (\app{f}{a})[\gamma] = \app{(f[\gamma])}{(a[\gamma])}$.
  By IH, $\mathcal{E}(\tau' \to \tau)(f[\gamma])$ and so $f[\gamma] \to^\ast \tabs{x}{\tau'}{e'}$ where $\mathcal{V}(\tau' \to \tau)(\tabs{x}{\tau'}{e'})$.
  Similarly, $a[\gamma] \to^\ast v$ where $\mathcal{V}(\tau')(v)$.
  But then we have $$e[\gamma] = \app{(f[\gamma])}{(a[\gamma])} \to^\ast \app{(\tabs{x}{\tau'}{e'})}{a} \to^\ast \app{(\tabs{x}{\tau'}{e'})}{a} \to e'[x \mapsto a]$$
  But since $\mathcal{V}(\tau' \to \tau)(\tabs{x}{\tau'}{e'})$ and $\mathcal{V}(\tau')(v)$, we get $\mathcal{E}(\tau)(e'[x \mapsto a])$, and so $e'[x \mapsto a] \to^\ast v'$ where $\mathcal{V}(\tau)(v')$.
  Thus, $$e[\gamma] \to^\ast v' \mathrel{\text{where}} \mathcal{V}(\tau)(v')$$ and so $\mathcal{E}(\tau)(e[\gamma])$, as desired.

  \noindent\textbf{Case \textsc{$\times$-I}:}
  In this case, $e = (e_1, e_2)$ where $\Gamma \vdash e_1 : \tau_1$ and $\Gamma \vdash e_2 : \tau_2$.
  Let $\gamma$ be such that $\mathcal{C}(\Gamma)(\gamma)$.
  Then by IH, $\mathcal{E}(\tau_1)(e_1[\gamma])$ and so $e_1[\gamma] \to^\ast v_1$ where $\mathcal{V}(\tau_1)(v_1)$.
  Similarly, $\mathcal{E}(\tau_2)(e_2[\gamma])$ and so $e_2[\gamma] \to^\ast v_2$ where $\mathcal{V}(\tau_2)(v_2)$.
  Hence $$e[\gamma] = (e_1, e_2)[\gamma] = (e_1[\gamma], e_2[\gamma]) \to^\ast (v_1, e_2[\gamma]) \to^\ast (v_1, v_2) \mathrel{\text{where}} \mathcal{V}(\tau_1)(v_1) \land \mathcal{V}(\tau_2)(v_2)$$
  and thus $\mathcal{E}(\tau_1 \times \tau_2)(e[\gamma])$ as desired.

  \noindent\textbf{Case \textsc{$\times$-E$_1$}:}
  In this case, $e = \projl{e'}$ where $\Gamma \vdash e' : \tau_1 \times \tau_2$.
  Let $\gamma$ be such that $\mathcal{C}(\Gamma)(\gamma)$.
  Then $e[\gamma] = \projl{(e'[\gamma])}$, and by IH $\mathcal{E}(\tau_1 \times \tau_2)(e'[\gamma])$
  But then $e'[\gamma] \to^\ast (v_1, v_2)$ where $\mathcal{V}(\tau_1)(v_1)$.
  Thus, $$e[\gamma] = \projl{(e'[\gamma])} \to^\ast \projl{(v_1, v_2)} \to v_1 \mathrel{\text{where}} \mathcal{V}(\tau_1)(v_1)$$
  and so $\mathcal{E}(\tau_1)(e[\gamma])$ as desired.

  \noindent\textbf{Case \textsc{$\times$-E$_2$}:}
  Very similar to the previous case.

  \noindent\textbf{Case \textsc{$+$-I$_1$}:}
  In this case, $e = \injl{e'}$ where $\Gamma \vdash e' : \tau_1$.
  Let $\gamma$ be such that $\mathcal{C}(\Gamma)(\gamma)$.
  Then by IH $e'[\gamma] \to v$ where $\mathcal{V}(\tau_1)(v)$.
  But then $$e[\gamma] = \injl{(e'[\gamma])} \to^\ast \injl{v} \mathrel{\text{where}} \mathcal{V}(\tau_1)(v)$$
  and so $\mathcal{E}(\sumtype{\tau_1}{\tau_2})(e[\gamma])$ as desired.

  \noindent\textbf{Case \textsc{$+$-I$_2$}:}
  Very similar to the previous case.

  \noindent\textbf{Case \textsc{$+$-E}:}
  In this case, $e = \case{e'}{x}{e_1}{y}{e_2}$ where $\Gamma \vdash e' : \sumtype{\tau_1}{\tau_2}$, $\Gamma, x : \tau_1 \vdash e_1 : \tau$, and $\Gamma, y : \tau_2 \vdash e_2 : \tau$.
  Let $\gamma$ be such that $\mathcal{C}(\Gamma)(\gamma)$.
  Then $e'[\gamma] \to^\ast v$ where $\mathcal{V}(\sumtype{\tau_1}{\tau_2})(v)$.
  Hence, either $v = \injl{v'}$ where $\mathcal{V}(\tau_1)(v')$ or $v =\injr{v'}$ where $\mathcal{V}(\tau_2)(v')$.
  In the first case, we get
  $$
  \begin{array}{ll}
    e[\gamma] = \case{(e'[\gamma])}{x}{e_1[\gamma]}{y}{e_2[\gamma]} &\to^\ast \case{(\injl{v'})}{x}{e_1[\gamma]}{y}{e_2[\gamma]}\\
                                                &\to e_1[\gamma][x \mapsto v'] = e_1[\gamma[x\mapsto v']]\\
                                                &\to^\ast v \mathrel{\text{where}} \mathcal{V}(\tau)(v)
  \end{array}
  $$
  where the last mutlistep comes from the IH along with the fact that $\mathcal{C}(\Gamma, x : \tau_1)(\gamma[x \mapsto v'])$.

  The second case is very similar.
  
\end{proof}

\begin{cor}[Termination]
  If $\vdash e : \tau$, then $e \to^\ast v$ where $v$ is a value.
\end{cor}
\begin{proof}
  Since $\vdash e : \tau$, by the fundamental theorem $\vDash e : \tau$.
  Note that $\mathcal{C}(\cdot)([\,])$, where $[\,]$ is the empty substitution.
  Thus, $\mathcal{E}(\tau)(e[\,]) = \mathcal{E}(\tau)(e)$.
  But this means that $e \to^\ast v$, as desired.
\end{proof}

\end{document}

%%% Local Variables:
%%% mode: latex
%%% TeX-master: t
%%% TeX-engine: luatex
%%% End:
