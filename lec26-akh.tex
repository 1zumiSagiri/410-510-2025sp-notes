\documentclass{lecturenotes}

\usepackage[colorlinks,urlcolor=blue]{hyperref}
\usepackage{doi}
\usepackage{xspace}
\usepackage{fontspec}
\usepackage{enumerate}
\usepackage{mathpartir}
\usepackage{pl-syntax/pl-syntax}
\usepackage{forest}
\usepackage{stmaryrd}
\usepackage{epigraph}
\usepackage{xspace}

\setsansfont{Fira Code}

\newcommand{\abs}[2]{\ensuremath{\lambda #1.\,#2}}
\newcommand{\tabs}[3]{\ensuremath{\lambda #1 \colon #2.\,#3}}
\newcommand{\dbabs}[1]{\ensuremath{\lambda.\,#1}}
\newcommand{\dbind}[1]{\ensuremath{\text{\textasciigrave}#1}}
\newcommand{\app}[2]{\ensuremath{#1\;#2}}
\newcommand{\utype}{\textsf{unit}\xspace}
\newcommand{\unit}{\ensuremath{\textsf{(}\mkern0.5mu\textsf{)}}}
\newcommand{\prodtype}[2]{\ensuremath{#1 \times #2}}
\newcommand{\pair}[2]{\ensuremath{(#1, #2)}}
\newcommand{\projl}[1]{\ensuremath{\pi_1\mkern2mu#1}}
\newcommand{\projr}[1]{\ensuremath{\pi_2\mkern3mu#1}}
\newcommand{\sumtype}[2]{\ensuremath{#1 + #2}}
\newcommand{\injl}[1]{\ensuremath{\textsf{inj}_1\mkern2mu#1}}
\newcommand{\injr}[1]{\ensuremath{\textsf{inj}_2\mkern3mu#1}}
\newcommand{\case}[5]{\ensuremath{\textsf{case}\mkern5mu#1\mkern5mu\textsf{of}\mkern5mu\injl{#2} \Rightarrow #3;\mkern5mu\injr{#4} \Rightarrow #5\mkern5mu\textsf{end}}}
\newcommand{\vtype}{\textsf{void}\xspace}
\newcommand{\vcase}[1]{\ensuremath{\textsf{case}\mkern5mu#1\mkern5mu\textsf{of}\mkern5mu\textsf{end}}}
\newcommand{\rectype}[2]{\ensuremath{\mu #1.\,#2}}
\newcommand{\roll}[1]{\textsf{roll}\mkern2mu#1}
\newcommand{\unroll}[1]{\textsf{unroll}\mkern2mu#1}
\newcommand{\fatype}[2]{\ensuremath{\forall #1.\,#2}}
\newcommand{\Abs}[2]{\Lambda #1.\,#2}
\newcommand{\App}[2]{#1\;[#2]}
\newcommand{\extype}[2]{\ensuremath{\exists #1.\,#2}}
\newcommand{\pack}[3]{\ensuremath{\textsf{pack}\mkern5mu#1\mathrel{\textsf{as}}#2\mathrel{\textsf{in}}#3}}
\newcommand{\unpack}[4]{\ensuremath{\textsf{unpack}\mkern5mu#1\mathrel{\textsf{as}} #2, #3 \mathrel{\textsf{in}} #4}}

\newcommand{\FV}{\text{FV}}
\newcommand{\BV}{\text{BV}}

\newcommand{\toform}[1]{\ensuremath{\lceil #1 \rceil}}
\newcommand{\totype}[1]{\ensuremath{\lfloor #1 \rfloor}}

\newcommand{\neutral}[1]{#1\;\text{ne}}
\newcommand{\nf}[1]{#1\;\text{nf}}

\newcommand{\subtype}{\ensuremath{\mathrel{\mathord{<}\mathord{:}}}}

\newcommand{\pub}{\text{public}}
\newcommand{\priv}{\text{secret}}

\title{Noninterference}
\coursenumber{CSE 410/510}
\coursename{Programming Language Theory}
\lecturenumber{26}
\semester{Spring 2025}
\professor{Professor Andrew K. Hirsch}

\begin{document}
\maketitle

\begin{itemize}
\item Last time, we talked about two programs as semantically equivalent when an observer cannot tell the difference between the two.
\item Our notion of an observer was any other program in the language.
\item We then came up with \emph{contextual equivalence}, which formalized this idea but was hard to work with.
\item By moving to a binary logical relation, we got something much easier to work with that was equivalent in power.
\item But what about other notions of observer?
\item Today, we'll be changing the observer in order to discuss \emph{information security}, specifically \emph{noninterference}.
\end{itemize}

\section{Information-Flow Security and Noninterference}
\label{sec:inform-flow-secur}

\begin{itemize}
\item Information-flow security (at least as we will see it today) is all about keeping confidential information from leaking into information shared with the public.
\item In particular, we want to mark some of our data as secret and the rest of our data as public.
  We won't share the secret data, but the public data could be shared with anyone.
\item For example, your social security number should probably be secret, but you probably don't mind the fact that you live in the state of New York being shared.
\item Then, a program that takes in your social security number and the state you live in and then tells \emph{you} how your income compares to the average income in the state while publicly displaying the average income of your state would be secure
\item However, if it displayed how your income compares to the average, it would not be---after all, I can figure out how much you make with that information.
\item In general, we're interest in a property called \emph{noninterference}: private inputs should not affect public outputs.
\item We state this by adding \emph{labels} to types.
  Today, we only consider two labels:  \pub{} and \priv{}. 
\item We write $\ell_1 \sqsubseteq \ell_2$ to mean ``data labeled with $\ell_1$ can be securely used to compute data labeled with $\ell_2$.''
  We read this ``$\ell_1$ flows to $\ell_2$.''
  With the two labels we have, this relation is defined as follows:
  \begin{mathpar}
    \infer{ }{\ell \sqsubseteq \ell} \and
    \infer{ }{\pub \sqsubseteq \priv}
  \end{mathpar}
  \begin{itemize}
  \item We can prove that this is a preoder.
  \item In this case, we can prove more: it's a total order!
  \end{itemize}
\item We write $\ell_1 \sqcup \ell_2$ for the least label that both $\ell_1$ and $\ell_2$ can be securely used to compute.
  In our simple setting, it's defined as:
  $$
  \ell_1 \sqcup \ell_2 = \left\{
    \begin{array}{ll}
      \pub & \ell_1 = \ell_2 = \pub\\
      \priv & \text{otherwise}
    \end{array}\right.
  $$
\end{itemize}

\section{A $\lambda$ Calculus for Noninterference}
\label{sec:lambda-calc-nonint}

\begin{syntax}
  \category[Label]{\ell} \alternative{\pub} \alternative{\priv}
  \category[Raw Types]{A} \alternative{\utype} \alternative{\prodtype{\tau_1}{\tau_2}} \alternative{\sumtype{\tau_1}{\tau_2}} \alternative{\tau_1 \to \tau_2}
  \category[Types]{\tau} \alternative{\tau^\ell}
  \category[Expression]{e} \alternative{\unit}\\
  \alternative{\pair{e_1}{e_2}} \alternative{\projl{e}} \alternative{\projr{e}}\\
  \alternative{\injl{e}} \alternative{\injr{e}} \alternative{\case{e}{x}{e_1}{y}{e_2}}\\
  \alternative{\tabs{x}{\tau}{e}} \alternative{\app{e_1}{e_2}}
\end{syntax}

\begin{mathpar}
  \infer*[left=Axiom]{x : \tau \in \Gamma}{\Gamma \vdash x : \tau}\and
  \infer*[left=UnitI]{ }{\Gamma \vdash \unit : \utype^\ell}\\
  \infer*[left=$\times$-I]{\Gamma \vdash e : \tau_1\\ \Gamma \vdash e : \tau_2}{\Gamma \vdash (e_1, e_2) : (\prodtype{\tau_1}{\tau_2})^\ell} \and
  \infer*[right=$\times\textsc{-E}_1$]{\Gamma \vdash e : (\prodtype{A^{\ell_1}}{\tau})^{\ell_2}}{\Gamma \vdash \projl{e} : A^{\ell_1 \sqcup \ell_2}} \and
  \infer*[right=$\times\textsc{-E}_2$]{\Gamma \vdash e : (\prodtype{\tau}{A^{\ell_1}})^{\ell_2}}{\Gamma \vdash \projr{e} : A^{\ell_1 \sqcup \ell_2}} \\
  \infer*[left=$+\textsc{-I}_1$]{\Gamma \vdash e : \tau_1}{\Gamma \vdash \injl{e} : (\sumtype{\tau_1}{\tau_2})^\ell}\and
  \infer*[left=$+\textsc{-I}_2$]{\Gamma \vdash e : \tau_2}{\Gamma \vdash \injr{e} : (\sumtype{\tau_1}{\tau_2})^\ell}\and
  \infer*[right=$+\textsc{-E}$]{\Gamma \vdash e : (\sumtype{\tau_1}{\tau_2})^{\ell_1} \\ \Gamma, x : \tau_1 \vdash e_1 : A^{\ell_2}\\ \Gamma, y : \tau_2 \vdash e_2 : A^{\ell_2}\\ \ell_1 \sqsubseteq \ell_2}{\Gamma \vdash \case{e}{x}{e_1}{y}{e_2} : A^{\ell_2}}\\
  \infer*[left=$\to\textsc{-I}$]{\Gamma, x : \tau_1 \vdash e : \tau_2}{\Gamma \vdash \tabs{x}{\tau_1}{e} : (\tau_1 \to \tau_2)^\ell}\and
  \infer*[right=$\to\textsc{-E}$]{\Gamma \vdash f : (A^{\ell_1} \to B^{\ell_2})^{\ell_3}\\ \Gamma \vdash a : A^{\ell_4}\\ \ell_4 \sqsubseteq \ell_1}{\Gamma \vdash \app{f}{a} : B^{\ell_2 \sqcup \ell_3}}
\end{mathpar}

\begin{itemize}
\item The syntax and type system of a $\lambda$ calculus with information-flow labels is above.
\item The operational semantics is exactly the same as what you're used to; the only difference is in the type system.
\item For the most part, we don't change anything.
  However, there are a few notes:
  \begin{itemize}
  \item First, every introduction rule creates an arbitrary label on the outside of the type.
    Intuitively, you as the programmer get to decide how secret the data you create is.
  \item Second, every elimination rule has to ensure that all of the relevant inputs are no-more-secret than the outputs.
    For instance, the $+\textsc{-E}$ rule requires that the choice of the left or right is no more secret than the output.
  \item To see why this is required, consider the following program:
    $$
    \case{\injl{\unit}}{\_}{3}{\_}{5}
    $$
    Note that the choice of left or right is the only relevant information here, and it determines the output.
    If that choice is secret, then labeling the output public would leak that secret to the public!
    This is an \emph{indirect flow}.
  \end{itemize}
\item Our goal today is to prove noninterference:\\ if $x : \tau^\priv \vdash e : \mathbb{B}^\pub$ and $\vdash v_1, v_2 : \tau^\priv$, then $e[x \mapsto v_1] \cong_{\beta} e[x \mapsto v_2]$.
\end{itemize}

\section{A Binary Logical Relation}
\label{sec:binary-logic-relat}

\begin{itemize}
\item The idea is to build a binary logical relation representing what an adversary can see.
\item However, there's more than one possible adversary.
  For any label, we can imagine an adversary who can see at most that label.
\item Thus, we write $\mathcal{V}(\tau)(v_1 \sim^\ell v_2)$ to mean ``an adversary who can see at most $\ell$-labeled data cannot tell $v_1$ and $v_2$ apart.''
\item Note that the recursion on types will look at both $\tau$s and $A$s.
\end{itemize}

$$
\begin{array}{r@{\;\triangleq\;}l}
  \mathcal{V}(\utype)(v_1 \sim_\ell v_2) & v_1 = v_2 = \unit\\
  \mathcal{V}(\prodtype{\tau_1}{\tau_2})(v_1 \sim_\ell v_2) & \exists v_{11}, v_{12}, v_{21}, v_{22}.\, \hspace{-1.5em}\begin{array}[t]{l}\hspace{1.15em}v_1 = (v_{11}, v_{12})\\ {} \land v_2 = (v_{21}, v_{22})\\ {} \land \mathcal{V}(\tau_1)(v_{11} \sim_\ell v_{21})\\ {} \land \mathcal{V}(\tau_2)(v_{12} \sim_\ell v_{22})\end{array}\\[1em]
  \mathcal{V}(\sumtype{\tau_1}{\tau_2})(v_1 \sim_\ell v_2) & \hspace{-1.5em}\begin{array}[t]{l}\hspace{1.15em}(\exists v'_1, v'_2.\, v_1 = \injl{v'_1} \land v_2 = \injl{v'_2} \land \mathcal{V}(\tau_1)(v'_1 \sim_\ell v'_2))\\ {} \lor (\exists v'_1, v'_2.\,v_1 = \injr{v'_1} \land v_2 = \injr{v'_2} \land \mathcal{V}(\tau_2)(v'_1 \sim_\ell v'_2)\end{array}\\
  \mathcal{V}(\tau_1 \to \tau_2)(v_1 \sim_\ell v_2) & \exists x, e_1, y, e_2.\, \hspace{-1.5em}\begin{array}[t]{l}\hspace{1.15em}v_1 = \tabs{x}{\tau_1}{e_1}\\{} \land v_2 = \tabs{y}{\tau_2}{e_2}\\{} \land (\forall a_1, a_2.\, \mathcal{V}(\tau_1)(a_1 \sim_\ell a_2) \to \mathcal{E}(\tau_2)(e_1[x \mapsto a_1] \sim_\ell e_2[y \mapsto a_2])\end{array}\\
  \mathcal{V}(A^{\ell_1})(v_1 \sim_{\ell_2} v_2) & \ell_1 \not\sqsubseteq \ell_2 \lor \mathcal{V}(A)(v_1 \sim_{\ell_2} v_2)\\
  \mathcal{E}(\tau)(e_1 \sim_\ell e_2) & \exists v_1, v_2.\, e_1 \to^\ast v_1 \land e_2 \to^\ast v_2 \land \mathcal{V}(\tau)(v_1 \sim_\ell v_2)
\end{array}
$$

\begin{itemize}
\item Note that this looks almost exactly the binary logical relation from last time.
\item However, when we encounter a label in a type, we first check whether the adversary can see information with that label.
  If not, we just give up, and stop checking any further.
  Intuitively, the adversary cannot tell anything with that type apart.
\item Note that the adversary who can see \priv{}-labeled data can see everything!
\item We then can prove the fundamental lemma as normal.
  Note that the definition of semantic typing is exactly the one we saw last time.
\end{itemize}

\begin{thm}[Fundamental Theorem]
  If $\Gamma \vdash e : \tau$, then $\Gamma \vDash e \sim_\ell e : \tau$ for every label~$\ell$.
\end{thm}

\section{Noninterference, Proven}
\label{sec:nonint-prov}

\begin{itemize}
\item We can now prove our noninterference theorem as a corollary of the fundamental theorem!
\end{itemize}

\begin{cor}[Noninterference]
  If $x : A^\priv \vdash e : \mathbb{B}^\pub$ and $\vdash v_1, v_2 : A^\priv$, then $e[x \mapsto v_1] \cong_{\beta} e[x \mapsto v_2]$.
\end{cor}
\begin{proof}
  Note that no matter what $v_1$ and $v_2$ are, $\mathcal{V}(A^\priv)(v_1 \sim_\pub v_2)$.
  Thus, $$\mathcal{K}(x : A^\priv)([x \mapsto v_1] \sim_\pub [x \mapsto v_2])$$ and so by the fundamental theorem $$\mathcal{E}(\mathbb{B}^\pub)(e[x \mapsto v_1] \sim_\pub e[x \mapsto v_2])$$
  This means that there are a $v'_1, v'_2$ such that all three of the following hold:
  \begin{mathpar}
    e[x \mapsto v_1] \to^\ast v'_1 \and e[x \mapsto v_2] \to^\ast v'_2 \and \mathcal{V}(\mathbb{B}^\pub)(v'_1 \sim_\pub v'_2)
  \end{mathpar}
  Since $\pub \sqsubseteq \pub$, we must further have that $\mathcal{V}(\mathbb{B})(v'_1 \sim_\pub v'_2)$.
  But this is only possible if $v'_1 = v'_2$, which means $e[x \mapsto v_1] \cong_\beta e[x \mapsto v_2]$ as desired.
\end{proof}


\end{document}

%%% Local Variables:
%%% mode: latex
%%% TeX-master: t
%%% TeX-engine: luatex
%%% End:
