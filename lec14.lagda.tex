\documentclass{lecturenotes}

\usepackage[colorlinks,urlcolor=blue]{hyperref}
\usepackage{doi}
\usepackage{xspace}
\usepackage{agda}
\usepackage{fontspec}
\usepackage{enumerate}
\usepackage{forest}
\setsansfont{Fira Code}
\usepackage{newunicodechar}
\newunicodechar{∣}{\ensuremath{\mid}}
\newunicodechar{σ}{\ensuremath{\sigma}}
\newunicodechar{ξ}{\ensuremath{\xi}}
\newunicodechar{⇒}{\ensuremath{\Rightarrow}}

\newcommand{\agdanats}{\textsf{ℕ}\xspace}

\title{De Bruijn Indices}
\coursenumber{CSE 410/510}
\coursename{Programming Language Theory}
\lecturenumber{14}
\semester{Spring 2025}
\professor{Professor Andrew K. Hirsch}

\begin{document}
\maketitle

\begin{code}
module lec14 where
open import Data.Nat using (ℕ; zero; suc)
open import Function using (_∘_)

data Expr : Set where
  `_ : ℕ → Expr
  λ⇒_ : Expr → Expr
  _∙_ : Expr → Expr → Expr
\end{code}

\section{The Barendregt convention}
The Barendregt Convention in lambda calculus states that bound variables in lambda terms are always chosen to be distinct from free variables. 

\noindent In other words, it is a practice of picking a representative term where all bound variables are \emph{unique} (so, no other $\lambda$ uses those variables) and are \emph{fresh} (not bound by other $\lambda$'s).

\section{De Bruijn Indices}
De Bruijn indices are a variable-free notation for lambda calculus where variables are replaced by numbers representing their depth in the nested scope. 

\noindent To convert variables into De Bruijn Indices, turn them into numbers that represent the amount of $\lambda$'s between the bound variable and its binding $\lambda$.

\noindent As an AST, $\lambda x. \lambda y. (xy) x$ then converted into an AST with De Bruijn indices would look like this:

\[
\begin{array}{llcrr}
\begin{forest}
    [$\lambda x$
    [$\lambda y$
      [@
        [@
            [$x$]
            [$y$]
        ]
        [$x$]
      ]
    ]
  ]
\end{forest}
&
\rightarrow
&
\begin{forest}
    [$\lambda$
    [$\lambda$
      [@
        [@
            [$1$]
            [$0$]
        ]
        [$1$]
      ]
    ]
  ]
\end{forest}
\end{array}
\]

\noindent Another example would be with a function  $\lambda x. \lambda y. ((\lambda x.x) x)y$:

\[
\begin{array}{lcr}
    \begin{forest}
        [$\lambda x$
        [$\lambda y$
          [@
            [@
                [$\lambda x$
                    [$x$]
                ]
                [$x$]
            ]
            [$y$]
          ]
        ]
      ]
    \end{forest}
    &
    \rightarrow
    &
    \begin{forest}
        [$\lambda$
        [$\lambda$
          [@
            [@
                [$\lambda$
                    [$0$]
                ]
                [$1$]
            ]
            [$0$]
          ]
        ]
      ]
    \end{forest}
\end{array}
\]



\noindent Note: Every \alpha -equivalent tree will have a shape similar to this.



\section{Renaming}
Renaming refers to the process of changing a bound variable's name while keeping the meaning of the expression intact. 

\noindent The primary goal in renaming is to avoid capture (where a variable accidentally gets bound to a different lambda due to name clashes).

\noindent Renaming is defined as follows:

\begin{code}
Renaming : Set
Renaming = ℕ → ℕ

↑r : Renaming → Renaming
↑r ξ zero = zero
↑r ξ (suc x) = suc(ξ x)

_⟨_⟩ : Expr → Renaming → Expr
(` x) ⟨ ξ ⟩ = ` (ξ x)
(λ⇒ e) ⟨ ξ ⟩ = λ⇒ (e ⟨ ↑r ξ ⟩)
(e₁ ∙ e₂) ⟨ ξ ⟩ = (e₁ ⟨ ξ ⟩) ∙ (e₂ ⟨ ξ ⟩)

\end{code}


\section{Substitution}

The substitution of a variable refers to the process of replacing a variable in a term with another term.

\begin{code}

Substitution : Set
Substitution = ℕ → Expr

↑ : Substitution → Substitution
↑ σ zero = ` zero
↑ σ (suc x) = (σ x) 

_[_] : Expr → Substitution → Expr
(` x) [ σ ] = σ x
(λ⇒ e)[ σ ] = λ⇒ (e [ ↑ σ ] )
(e₁ ∙ e₂)[ σ ] = (e₁ [ σ ]) ∙ (e₂ [ σ ])

\end{code}



\end{document}

%%% Local Variables:
%%% mode: latex
%%% TeX-master: t
%%% TeX-engine: luatex
%%% TeX-command-default: "Make"
%%% End: