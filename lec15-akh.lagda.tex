\documentclass{lecturenotes}

\usepackage[colorlinks,urlcolor=blue]{hyperref}
\usepackage{doi}
\usepackage{xspace}
\usepackage{agda}
\usepackage{fontspec}
\usepackage{enumerate}
\usepackage{mathpartir}
\usepackage{pl-syntax/pl-syntax}
\usepackage{forest}
\usepackage{stmaryrd}
\usepackage{epigraph}

\setsansfont{Fira Code}
\usepackage{newunicodechar}
\newunicodechar{∣}{\ensuremath{\mid}}
\newunicodechar{′}{\ensuremath{{}^\prime}}
\newunicodechar{ˡ}{\ensuremath{{}^{\textsf{l}}}}
\newunicodechar{ʳ}{\ensuremath{{}^{\textsf{r}}}}
\newunicodechar{≤}{\ensuremath{\mathord{\leq}}}
\newunicodechar{≡}{\ensuremath{\mathord{\equiv}}}
\newunicodechar{≐}{\ensuremath{\mathord{\doteq}}}
\newunicodechar{∘}{\ensuremath{\circ}}
\newunicodechar{≃}{\ensuremath{\simeq}}
\newunicodechar{≲}{\ensuremath{\precsim}}
\newunicodechar{⊎}{\ensuremath{\uplus}}
\newunicodechar{≟}{\ensuremath{\stackrel{?}{=}}}
\newunicodechar{̬}{\ensuremath{{}_{\textsf{v}}}}
\newunicodechar{ₐ}{\ensuremath{{}_{\textsf{a}}}}
\newunicodechar{ₜ}{\ensuremath{{}_{\textsf{t}}}}
\newunicodechar{ₖ}{\ensuremath{{}_{\textsf{k}}}}
\newunicodechar{ᵣ}{\ensuremath{{}_{\textsf{r}}}}
\newunicodechar{₁}{\ensuremath{{}_{1}}}
\newunicodechar{₂}{\ensuremath{{}_{2}}}
\newunicodechar{₃}{\ensuremath{{}_{3}}}
\newunicodechar{⊕}{\ensuremath{\oplus}}
\newunicodechar{⊗}{\ensuremath{\otimes}}
\newunicodechar{σ}{\ensuremath{\sigma}}
\newunicodechar{∸}{\ensuremath{\stackrel{.}{-}}}
\newunicodechar{≮}{\ensuremath{\not<}}
\newunicodechar{⋆}{\ensuremath{{}^{\ast}}}
\newunicodechar{⇓}{\ensuremath{\Downarrow}}
\newunicodechar{⇒}{\ensuremath{\implies}}
\newunicodechar{‵}{\ensuremath{`}}
\newunicodechar{ƛ}{\ensuremath{\mkern0.75mu-\mkern -12mu\lambda}}
\newunicodechar{↦}{\ensuremath{\mapsto}}
\newunicodechar{ξ}{\ensuremath{\xi}}
\newunicodechar{τ}{\ensuremath{\tau}}

\newcommand{\abs}[2]{\ensuremath{\lambda #1.\,#2}}
\newcommand{\tabs}[3]{\ensuremath{\lambda #1 \colon #2.\,#3}}
\newcommand{\dbabs}[1]{\ensuremath{\lambda.\,#1}}
\newcommand{\dbind}[1]{\ensuremath{\text{\textasciigrave}#1}}
\newcommand{\app}[2]{\ensuremath{#1\;#2}}
\newcommand{\FV}{\text{FV}}
\newcommand{\BV}{\text{BV}}

\title{Simply Typed $\lambda$~Calculus}
\coursenumber{CSE 410/510}
\coursename{Programming Language Theory}
\lecturenumber{15}
\semester{Spring 2025}
\professor{Professor Andrew K. Hirsch}

\begin{document}
\maketitle

\begin{code}[hide]
module lec15-akh where

open import Data.Nat using (ℕ; zero; suc)
open import Data.List using ([]; _∷_; _++_)
open import Data.List.Membership.Propositional using(_∈_)
open import Relation.Binary.PropositionalEquality using (_≡_; refl; sym; trans; cong)
open import Data.Product using (Σ; ∃; Σ-syntax; ∃-syntax; _×_) renaming (_,_ to ⟨_,_⟩)
open import Data.Sum using (_⊎_; inj₁; inj₂)

open import Lambda
\end{code}  

\begin{itemize}
\item In this lecture, we begin to explore \emph{type systems}, a theme we will return to for the rest of the semester.
\item The purpose of a (data) type system is to constrain the shape of data.
  \begin{itemize}
  \item For instance, data with type $\mathbb{N}$ should have the ``shape'' of a number, data with type $t \times u$ should have the ``shape'' of a pair, et cetera.
  \end{itemize}
\item By constraining the ``shape'' of the data, we ensure that a programmer has not made the mistake of promising data of one shape while providing data of a different shape.
\item This means that other parts of the program which rely on the data having that shape will not be disappointed.
\end{itemize}

\section{The Syntax of Types}
\label{sec:syntax-types}

\begin{itemize}
\item We start by considering what the possible types of $\lambda$~calculus would be.
\item In $\lambda$~calculus, every term represents a function.
  There's nothing else it could represent!
\item Therefore, it's tempting to try to use the following syntax for types:
  \begin{syntax}
    \category[Types]{\tau} \alternative{\tau_1\to\tau_2}
  \end{syntax}
\item However, this cannot work.
  \begin{itemize}
  \item \textbf{Stop and Think:} Why not?
  \item Just try writing down a type in this syntax!
  \end{itemize}
\item Instead, we create a ``dummy'' \emph{base} type~$A$, which will allow us to actually write down types.
  \begin{syntax}
    \category[Types]{\tau} \alternative{A} \alternative{\tau_1\to\tau_2}
  \end{syntax}
\item Here, $A$ does not represent any data.
\item In other words, \emph{there are no closed terms that have type $A$}.
  We will not be able to prove this today, but you should be able to convince yourself that it's true.
\end{itemize}

\section{Contexts}
\label{sec:contexts}

\begin{itemize}
\item One of the most important notions in programming languages is that of \emph{scope}.
  You've run into many times before.
\item In a type system, the type of a term can depend on the variables that are in scope.
\item For instance,
  \begin{itemize}
  \item The variable~$y$ \emph{does not have a type} if it is not in scope.
  \item The term $x + 1$ has type $\mathbb{N}$ if the variable~$x$ has type $\mathbb{N}$ in the current scope
  \item etc
  \end{itemize}
\item In type systems, we use \emph{contexts} to keep track of all of the variables in scope.
\item A context is just a list of variable-type pairs written $x \colon \tau$, meaning ``variable~$x$ has type~$\tau$ in the current scope.''
  \begin{syntax}
    \category[Contexts]{\Gamma} \alternative{\cdot} \alternative{\Gamma, x \colon \tau}
  \end{syntax}
  \begin{itemize}
  \item Note that this is different then an \emph{evaluation context}, which is a term with a hole.
  \end{itemize}
\end{itemize}

\section{The Type System}
\label{sec:type-system}

\begin{itemize}
\item A type system is defined as an inductive relation.
\item That relation is \emph{trinary} i.e., it has $3$ parts.
\item We write $\Gamma \vdash e \colon \tau$ to mean ``in context~$\Gamma$ (describing what variables are in scope), the expression $e$ has type $\tau$.''
\item We call the relation $\_\vdash\_\colon\_$ a \emph{typing judgment}.
\item For $\lambda$~calculus, we can define the typing system as follows:
\end{itemize}

\begin{mathpar}
  \infer*[left=var]{x \colon \tau \in \Gamma}{\Gamma \vdash x \colon \tau}\and
  \infer*[left=abs]{\Gamma, x \colon \tau_1 \vdash e \colon \tau_2}{\Gamma \vdash \abs{x}{e} \colon \tau_1 \to \tau_2} \and
  \infer*[left=app]{\Gamma \vdash t \colon \tau_1 \to \tau_2\\ \Gamma \vdash u \colon \tau_2}{\Gamma \vdash \app{t}{u} \colon \tau_2}
\end{mathpar}

\begin{itemize}
\item A variable~$x$ has type~$\tau$ if the current context contains~$x \colon \tau$.
\item A function has an arrow type if the body of the function is well typed after adding the parameter to the context.
\item A function call is well typed if the function has an arrow type \emph{and} the argument has the type to the left of the arrow.
\end{itemize}

\section{Two Views on Types}
\label{sec:two-views-types}

\begin{itemize}
\item So far, we've thought about types as being imposed on some preexisting untyped system.
  \begin{itemize}
  \item In particular, that system is untyped $\lambda$ calculus.
  \end{itemize}
\item This is Curry's view of types.
\item However, Church viewed types differently: he believed that we should make types an integral part of the language.
\item This leads to a slightly different formal development:
\end{itemize}
\begin{syntax}
  \category[Expressions]{e,t,u} \alternative{x} \alternative{\tabs{x}{\tau}{e}} \alternative{\app{t}{u}}
\end{syntax}
\begin{mathpar}
  \infer*[left=var]{x \colon \tau \in \Gamma}{\Gamma \vdash x \colon \tau}\and
  \infer*[left=abs]{\Gamma, x \colon \tau_1 \vdash e \colon \tau_2}{\Gamma \vdash \tabs{x}{\tau_1}{e} \colon \tau_1 \to \tau_2} \and
  \infer*[left=app]{\Gamma \vdash t \colon \tau_1 \to \tau_2\\ \Gamma \vdash u \colon \tau_2}{\Gamma \vdash \app{t}{u} \colon \tau_2}
\end{mathpar}
\begin{itemize}
\item In the first, we have the language of untyped $\lambda$~calculus and then impose a typing discipline on top.
\item In the second, we have incorporated the types directly into the syntax of terms.
  By adding a typing annotation, we have made it easier to check the term, but less flexible.
  \begin{itemize}
  \item For instance, in Curry's style $\abs{x}{x}$ can be given type $\tau \to \tau$ for any type $\tau$.
  \item However, in Church's style $\tabs{x}{\tau}{x}$ bakes in the choice of $\tau$.
  \item Even in Curry's style, once one instance of $\abs{x}{x}$ has been given type $\tau \to \tau$, that choice is irrevertable: no other choice of type can replace $\tau$.
  \end{itemize}
\item Curry-style typing is sometimes known as \emph{extensional} typing, while Church-style typing is sometimes known as \emph{intensional} typing.
\item For simple types, they are equivalent: you can use Hindley-Milner type inference to recover one from the other.
\item However, in more-complicated type systems, they may not be equivalent.
  For instance, in dependent types extensional type systems have undecidable type checking.
  However, they also allow for function extensionality by default, without an axiom.
\item Personally, my view is closer to Church's
\end{itemize}
\subsection{Historical Context}
\label{sec:historical-context}

\begin{itemize}
\item Haskell Curry was one of the two inventors of typed $\lambda$ calculus.
\item He was an American who was one of the founders of \emph{combinatory logic}, an area of logic that was very popular in the 1930s and '40s.
\item Two of Alonzo Church's students, Stephen Kleene and John Rosser, found that both $\lambda$ calculus and combinatory logic were paradoxical in the same way.
\item This lead to the development of typed $\lambda$~calculus by both Curry and Church independently, though each with their own style.
\item Curry said that we should not ``run away from paradox,'' while Church said that a term without a type may be meaningless.
\item This debate continues to this day.
  Robert Harper famously argues that untyped systems are actually ``unityped systems'' (though we'll get into this more later).
  Others, including many in the gradual typing community, view typed terms as having been checked for errors, while untyped terms simply haven't been checked yet.
\end{itemize}

\section{Type Soundness}
\label{sec:type-soundness}

\epigraph{Well-typed programs cannot ``go wrong.''}{Robin Milner}

\begin{itemize}
\item The point of using a type discipline is to make sure that programs are ``meaningful'' and ``free of paradoxes.''
\item What does that mean for us?
\item We would like to prove that well-typed programs never ``get stuck,'' where there is no next step they can take in CBV~reduction.
\item In order to ensure that, we take the approach of Felleisen and Wright and divide our proof into two parts: ``progress'' and ``preservation.''
\end{itemize}

\subsection{Substitution and Types}
\label{sec:substitution-types}

\begin{itemize}
\item Substitution is the key operation on terms, and it is used to define $\beta$ equivalence.
\item In order to know how types interact with steps, we need to know how they interact with substitution.
\item Intuitively, if $x \colon \tau$ is in scope in $e$, then replacing $x$ in $e$ with any $e' \colon \tau$ should be okay.
  In other words, $x$ stands for some arbitrary~$\tau$, and replacing it with whatever~$\tau$ it stands for should be okay.
\item We formalize this in a theorem which you will be asked to prove in your homework using the Barendregt convention:
\end{itemize}
\begin{thm}[Substitution]
  If $\Gamma \vdash e \colon \tau$ and $\Gamma, x \colon \tau \vdash e' \colon \tau'$, then $\Gamma \vdash e'[x \mapsto e] \vdash \tau'$.
\end{thm}

\subsection{Type Preservation}
\label{sec:type-preservation}

Type preservation says that taking a CBV~reduction step does not change the type of a term.

\begin{thm}[Preservation]
  If $\Gamma \vdash e \colon \tau$ and $e \to e'$, then $\Gamma \vdash e' \colon \tau$.
\end{thm}
\begin{proof}
  By induction on the proof that $\Gamma \vdash e \colon \tau$.

  \noindent\textbf{Case \textsc{var}:}
  In this case, we know that $e = x$.
  But then there is no possible step $x \to e'$, and so this case works vacuously.

  \noindent\textbf{Case \textsc{abs}:}
  In this case, $e$ is a value and there is no $e'$ such that $e \to e'$, so this case works vacuously.

  \noindent\textbf{Case \textsc{app}:}
  In this case, we know that $e = \app{e_1}{e_2}$ where $\Gamma \vdash e_1 \colon \tau_1 \to \tau$ and $\Gamma \vdash e_2 \colon \tau_1$.
  There are three possibilities for the step: either $e_1 \to e_1'$, $e_2 \to e_2'$, or $e_1 = \abs{x}{e_1'}$ and $e' = e_1'[x \mapsto e_2]$.
  In the first two cases, the induction hypothesis gives exactly the information needed to construct the desired type derivation.
  In the last case, we can appeal to the Substitution theorem to get the desired result.
\end{proof}

\begin{cor}[Multistep Preservation]
  If $\Gamma \vdash e \colon \tau$ and $e \to^\ast e'$, then $\Gamma \vdash e' : \tau$.
\end{cor}
\begin{proof}
  By induction on the proof of $e \to^\ast e'$.

  \noindent\textbf{Case \textsc{Refl}:} In this case, $e = e'$ and we're done.

  \noindent\textbf{Case \textsc{Step}:} In this case, there is an $e''$ such that $e \to e''$ and $e'' \to^\ast e'$.
  By Preservation, $\Gamma \vdash e'' \colon \tau$.
  Then, by IH, $\Gamma \vdash e' \colon \tau$ as desired.
\end{proof}

\subsection{Progress}
\label{sec:progress}

Progress says that a well-typed closed term can always take a step unless it is already a value.
Note two things.
First, that a closed term is exactly a term that can be typed in an empty context---i.e., with no variables in scope.
Second, that a value can never take a step.

\begin{thm}[Progress]
  If $\vdash e \colon \tau$, then either $e$ is a value or there is an $e'$ such that $e \to e'$.
\end{thm}
\begin{proof}
  By induction on the proof of $\vdash e \colon \tau$.

  \noindent\textbf{Case \textsc{var}:} Since $e$ is closed this case is impossible, and so works vacuously.

  \noindent\textbf{Case \textsc{abs}:} In this case, $e$ is already a value and we are done.

  \noindent\textbf{Case \textsc{app}:} In this case, $e = \app{e_1}{e_2}$.
  By IH, either $e_1$ is a value or there is an $e_1'$ such that $e_1 \to e_1'$.
  In the latter case, $e \to \app{e_1'}{e_2}$ and we are done, so assume that $e_1$ is a value.
  In this case, $e_1$ must have the form $\abs{x}{e_1'}$.
  By IH again, either $e_2$ is a value or there is an $e_2'$ such that $e_2 \to e_2'$.
  In the latter case $e \to \app{e_1}{e_2'}$, so assume that $e_2$ is a value.
  But then because $e_1 = \abs{x}{e_1'}$, we get $$e = \app{(\abs{x}{e_1'})}{e_2} \to e_1'[x \mapsto e_2]$$
\end{proof}

\subsection{Putting it all Together}
\label{sec:putting-it-all}

\begin{thm}[Type Safety]
  If $\vdash e \colon \tau$ and $e \to^\ast e'$, then either $e'$ is a value or there is a $e''$ such that $e' \to e''$.
\end{thm}
\begin{proof}
  By Multistep Preservation, we know that $\vdash e' : \tau$.
  But then progress exactly matches the rest of the theorem statement.
\end{proof}

\begin{itemize}
\item Thus, any well-typed closed term cannot ``get stuck''.
\item After all, it can always take a step unless it is a value.
\item Moreover, if you take a few steps, you can take more steps unless you have hit a value.
\item In other words, our type system has done its job and kept us from writing misbehaving programs.
\end{itemize}

\section{Type Systems in de~Bruijn Indices}
\label{sec:type-systems-de}

\begin{itemize}
\item With de~Bruijn indices we use the same trick for type systems that we did for substitutions: we type with an infinite context.
\item This leads to the following code:
\begin{code}
-- again, using long arrow \-->
infixl 4 _⟶_
data Type : Set where
  A : Type
  _⟶_ : Type → Type → Type

Ctxt = ℕ → Type

upctxt : Ctxt → Type → Ctxt
upctxt Γ τ zero = τ
upctxt Γ τ (suc n) = Γ n

infix 3 _⊢_∷_
data _⊢_∷_ : Ctxt → Expr → Type → Set where
  var : ∀ {Γ : Ctxt} {x : ℕ} {τ : Type} →
     Γ x ≡ τ →
     -- Why the indirection? I want this to be a proof goal, not a unification problem
     ------------
     Γ ⊢ ‵ x ∷ τ
  arr : ∀ {Γ : Ctxt} {e : Expr} {τ₁ τ₂ : Type} →
    upctxt Γ τ₁ ⊢ e ∷ τ₂ →
    Γ ⊢ ƛ⇒ e ∷ τ₁ ⟶ τ₂
  app : ∀ {Γ : Ctxt} {e₁ e₂ : Expr} {τ₁ τ₂ : Type} →
    Γ ⊢ e₁ ∷ τ₁ ⟶ τ₂ →
    Γ ⊢ e₂ ∷ τ₁ →
    Γ ⊢ e₁ · e₂ ∷ τ₂

typing-ext : ∀ {Γ Δ : Ctxt} {e : Expr} {τ : Type} →
  (∀ n → Γ n ≡ Δ n) →
  Γ ⊢ e ∷ τ →
  Δ ⊢ e ∷ τ
typing-ext exteq (var {x = x} eq) = var (trans (sym (exteq x)) eq)
typing-ext {Γ} {Δ} exteq (arr typ) =
  arr (typing-ext (λ {zero → refl ; (suc n) → exteq n }) typ)
typing-ext exteq (app typ₁ typ₂) =
  app (typing-ext exteq typ₁) (typing-ext exteq typ₂)
\end{code}
\item To connect this with substitutions, we need the notion of typing a substitution.
\item We write $\Gamma \vdash \sigma \dashv \Delta$ to mean ``the substitution~$\sigma$ moves terms from context~$\Gamma$ to context~$\Delta$.
\item Formally, this works if for every $x \colon \tau \in \Gamma$, $\Delta \vdash \sigma(x) \colon \tau$.
\begin{code}
_⊢_⊣_ : Ctxt → Substitution → Ctxt → Set
Γ ⊢ σ ⊣ Δ = ∀ (x : ℕ) → Δ ⊢ (σ x) ∷ Γ x
\end{code}
\item Renamings, in the meantime, merely ``shuffle around'' the free variables.
  We need to ``shuffle around'' the context the same way.
\begin{code}
renaming-typing : ∀ {Γ : Ctxt} {ξ : Renaming} {e : Expr} {τ : Type} →
  (λ x → Γ (ξ x)) ⊢ e ∷ τ → Γ ⊢ e ⟨ ξ ⟩ ∷ τ
renaming-typing {Γ} {ξ} {e} {τ} (var eq) = var eq
renaming-typing {Γ} {ξ} {ƛ⇒ e} {τ₁ ⟶ τ₂} (arr {_} {e} {τ₁} {τ₂} typ) =
  arr (renaming-typing
        (typing-ext {upctxt (λ n → Γ (ξ n)) τ₁}
                    {λ n → (upctxt Γ τ₁) (↑ᵣ ξ n)}
                    (λ {zero → refl; (suc n) → refl})
                    typ))
renaming-typing (app typ₁ typ₂) = app (renaming-typing typ₁) (renaming-typing typ₂)
\end{code}
\newpage
\item We can then show the central theorem: if $\Gamma \vdash \sigma \dashv \Delta$ and $\Gamma \vdash e \colon \tau$, then $\Delta \vdash e[\sigma] \colon \tau$.
\begin{code}
uptyping : ∀ {Γ Δ : Ctxt} {σ : Substitution} {τ : Type} →
  Γ ⊢ σ ⊣ Δ → upctxt Γ τ ⊢ ↑ σ ⊣ upctxt Δ τ
uptyping {Γ} {Δ} {σ} {τ} styp zero = var refl
uptyping {Γ} {Δ} {σ} {τ} styp (suc n) = renaming-typing (styp n)

subst-typing : ∀ {Γ Δ : Ctxt} {e : Expr} {τ : Type} {σ : Substitution} →
  Γ ⊢ e ∷ τ → Γ ⊢ σ ⊣ Δ → Δ ⊢ (e [ σ ]) ∷ τ
subst-typing {e = ‵ x} (var eq) styp rewrite sym eq = styp x
subst-typing (arr typ) styp =
  arr (subst-typing typ (uptyping styp))
subst-typing (app typ₁ typ₂) styp =
  app (subst-typing typ₁ styp) (subst-typing typ₂ styp)

zsubst-typing : ∀ {Γ : Ctxt} {e : Expr} {τ : Type} →
  Γ ⊢ e ∷ τ →
  upctxt Γ τ ⊢ zsubst e ⊣ Γ
zsubst-typing typ zero = typ
zsubst-typing typ (suc n) = var refl
\end{code}
\end{itemize}
\end{document}

%%% Local Variables:
%%% mode: latex
%%% TeX-master: t
%%% TeX-engine: luatex
%%% TeX-command-default: "Make"
%%% End:
