\documentclass{lecturenotes}

\usepackage[colorlinks,urlcolor=blue]{hyperref}
\usepackage{doi}
\usepackage{xspace}
\usepackage{agda}
\usepackage{fontspec}
\usepackage{enumerate}
\usepackage{amsmath}
\usepackage{mathpartir}
\setsansfont{Fira Code}
\usepackage{newunicodechar}
\newunicodechar{∣}{\ensuremath{\mid}}
\newunicodechar{′}{\ensuremath{{}^\prime}}
\newunicodechar{⊎}{\ensuremath{\uplus}}
\newunicodechar{₁}{\ensuremath{{}_1}}
\newunicodechar{₂}{\ensuremath{{}_2}}
\newunicodechar{ˡ}{\ensuremath{{}^{\textsf{l}}}}
\newunicodechar{ʳ}{\ensuremath{{}^{\textsf{r}}}}
\newunicodechar{≤}{\ensuremath{\mathord{\leq}}}
\newunicodechar{≡}{\ensuremath{\mathord{\equiv}}}
\newunicodechar{≐}{\ensuremath{\mathord{\doteq}}}
\newunicodechar{∘}{\ensuremath{\circ}}
\newunicodechar{≃}{\ensuremath{\simeq}}
\newunicodechar{≲}{\ensuremath{\precsim}}


\newcommand{\agdanats}{\textsf{ℕ}\xspace}
\newcommand{\agdaempty}{\textsf{⊥}\xspace}

\title{Quantifiers}
\coursenumber{CSE 410/510}
\coursename{Programming Language Theory}
\lecturenumber{8}
\semester{Spring 2025}
\professor{Professor Andrew K. Hirsch}

\begin{document}
\maketitle

\begin{code}
module lec08 where

import Relation.Binary.PropositionalEquality as Eq
open Eq using (_≡_; refl)
open import Data.Nat using (ℕ; zero; suc; _+_; _*_; _≤_; z≤n)
open import Isomorphism using (_≃_; _≲_; extensionality; _⇔_)
open import Data.Empty using (⊥; ⊥-elim)
open import Relation.Nullary using (¬_)
open import Data.Product using (_×_; proj₁; proj₂) renaming (_,_ to ⟨_,_⟩)
open import Data.Sum using (_⊎_; inj₁; inj₂)
open import Function using (_∘_)
\end{code}

\section{Quantifiers}
\label{sec:quantifiers}

In logic we use two quantifiers:
\begin{itemize}
  \item The \textbf{universal quantifier} ($\forall$) states that a property holds for every element of a set.
  \item The \textbf{existential quantifier} ($\exists$) asserts that there is at least one element in the set for which the property holds.
\end{itemize}
In Agda these are represented by:
\begin{itemize}
  \item $\forall$ is just a function type, or a Pi-type.
  \item $\exists$ is represented by a Sigma-type, which is a pair of an element of the set and a proof that the element satisfies the property.
\end{itemize}

\section{Universals}
\label{sec:universals}

The statement $\forall x : A.P(x)$ means that for every $x$ in $A$, the property $P(x)$ holds. In Agda we write this as a function type:
\begin{code}
_ : ∀ (x : ℕ) → 0 ≤ x
_ = λ x → _≤_.z≤n
\end{code}
Here the input $x$ is explicitly named so that it appears in the output type (i.e. the proof that $0 ≤ x$). Compare this with a function where the input name is implicit:
\begin{code}
_ : ℕ → ℕ
_ = λ x → x + 510
\end{code}
Even though both examples define functions, in the first case we use an explicit name in the type (with $\forall$) to indicate that the result depends on that particular $x$.

\newpage

Here we have an elimination rule for the universal quantifier:
\begin{code}
∀-elim : ∀ {A : Set} {B : A → Set} →
  (f : ∀ (x : A) → B x) →
      (a : A) →
  -----------------
        B a

∀-elim f a = f a
\end{code}
This tells us that if we have a function $f$ giving a proof for every $x$ (i.e. $B(x)$), then for any specific $a$ we obtain a proof of $B(a)$ by applying $f$ to $a$.
\newline \textbf{Note:} The usual function arrow ($\rightarrow$) is just a forall type:
\[
A \rightarrow B \equiv \forall (\_ : A) \rightarrow B.
\]
We also have function extensionality:
\begin{code}
postulate
  ∀-extensionality : {A : Set} {B : A → Set} {f g : ∀ (x : A) → B x}
    → (∀ (x : A) → f x ≡ g x)
    -------------------------
            → f ≡ g

-- Note: This version of function extensionality is more general than the one imported from Isomorphism.

lecture-extensionality : ∀ {A B : Set}{f g : A → B}
  → (∀ (x : A) → f x ≡ g x)
  --------------------------
          → f ≡ g

lecture-extensionality = ∀-extensionality
\end{code}
This tells us that if two functions give equal outputs for all inputs, then the functions themselves are equal.
\newline \newline
Here we see that universal quantification distributes over conjunction, 
meaning a universal statement about a product is equivalent to a product of universal statements:
\begin{code}
∀-distrib-× : ∀ {A : Set} {B C : A → Set} →
  (∀ (x : A) → B x × C x) ≃ (∀ (x : A) → B x) × (∀ (x : A) → C x)

∀-distrib-× =
  record
  {
     to      = λ f → ⟨ (λ x → proj₁ (f x)), (λ x → proj₂ (f x)) ⟩
   ; from    = λ { ⟨ f , g ⟩ → λ x → ⟨ f x , g x ⟩ }
   ; from∘to = λ f → refl
   ; to∘from = λ f → refl
   }
\end{code}
This shows that giving a pair for each $x$ is equivalent to giving two functions, one for each component.
\newpage

\section{Existentials}
\label{sec:existentials}

The statement $\exists x : A.\; P(x)$ means that there exists some $x$ in $A$ for which $P(x)$ holds. In Agda we represent this with a $\Sigma$-type:
\begin{code}
record Σ (A : Set) (B : A → Set) : Set where
  constructor ⟨_,_⟩
  field
    proj₁ : A
    proj₂ : B proj₁
\end{code}
This record pairs an element of $A$ with a proof that the element satisfies the given condition. 
Meaning that type $B$ specifies the condition that must hold for the element.
\newline \newline
We define a custom syntax for $\Sigma$, which is simply syntactic sugar for improved readability:
\begin{code}
Σ-syntax = Σ
infix 2 Σ-syntax
syntax Σ-syntax A (λ x → Bx) = Σ[ x ∈ A ] Bx

∃ : ∀ {A : Set} (B : A → Set) → Set
∃ {A} B = Σ A B

∃-syntax = ∃
syntax ∃-syntax (λ x → B) = ∃[ x ] B
\end{code}
Writing $\exists[ x ] B$ is just a shorthand for the corresponding $\Sigma$-type.
\newline \newline
To “use” an existential proof, we have the elimination rule:
\begin{code}
∃-elim : ∀ {A : Set} {B : A → Set} {C : Set} →
  (∀ x → B x → C) →
    ∃[ x ] B x →
  -----------------
        C

∃-elim f ⟨ x , y ⟩ = f x y
\end{code}
This means that if, for every $x$ and that $B x$ implies $C$, 
and also for some $x$ of type $A$ that $B x$ holds then we can conclude that $C$ holds.
\newline 
We can see that the converse also holds, and the two together form an isomorphism:
\begin{code}
∀∃-currying : ∀ {A : Set} {B : A → Set} {C : Set}
  → (∀ x → B x → C) ≃ (∃[ x ] B x → C)
∀∃-currying =
  record
  {
     to      = λ {f → λ { ⟨ x , y ⟩ → f x y } }
   ; from    = λ {g → λ x → λ y → g ⟨ x , y ⟩ }
   ; from∘to = λ f → refl
   ; to∘from = λ g → refl
   }
\end{code}
This shows that a function taking two arguments (an $x$ and a proof of $B(x)$) is equivalent to a function taking a single $\Sigma$-pair.
\textbf{Note:} In the finite case, the size of a $\Pi$-type (product) is given by the product of the sizes of each $B(x)$, 
while the size of a $\Sigma$-type (sum) is given by the sum of the sizes, hence the names.
\subsection{Existential Example}
Here's an example of how we can use existentials.
We can define even and odd numbers as follows:
\begin{code}
data even : ℕ → Set
data odd : ℕ → Set

data even where
  even-zero : even zero

  even-suc : ∀ {n : ℕ} →
        odd n →
     ------------
     even (suc n)

data odd where
  odd-suc : ∀ {n : ℕ} →
      even n →
   -------------
    odd (suc n)
\end{code}

We can then relate even/odd proofs to an existential statement. 
For example, we might express that an even number $n$ is of the form $2 \times m$:
\begin{code}
even-∃ : ∀ {n : ℕ} → even n → ∃[ m ] (m * 2 ≡ n)
odd-∃ : ∀ {n : ℕ} → odd n → ∃[ m ] (1 + m * 2 ≡ n)

even-∃ even-zero = ⟨ 0 , refl ⟩
even-∃ (even-suc on) with odd-∃ on
... | ⟨ m , refl ⟩ = ⟨ suc m , refl ⟩

odd-∃ (odd-suc en) with even-∃ en
... | ⟨ m , refl ⟩ = ⟨ m , refl ⟩
\end{code}
\begin{itemize}
  \item If the number is even because it is zero, then return the pair.
  \item If the number is even because it is one more than an odd number, 
  then apply the induction hypothesis to obtain a number $m$ and evidence that $1 + m \times 2 \equiv n$. Then return the pair.
  \item If the number is odd because it is the successor of an even number, 
  then apply the induction hypothesis to obtain a number $m$ and evidence that $m \times 2 \equiv n$. Then return the pair.
\end{itemize}
\newpage
\section{Existentials, Universals, and Negation}
\label{sec:additional}

\subsection{Negation and Quantifiers}
We now consider relationships between negation and quantifiers.
One useful equivalence is between the negation of an existential and a universal of a negation,
which we see here that they are isomorphic:
\begin{code}
¬∃≃∀¬ : ∀ {A : Set} {B : A → Set} →
  (¬ ∃[ x ] B x ≃ ∀ x → ¬ B x)
¬∃≃∀¬ =
  record
  {
    to      = λ {f → λ x → λ y → f ⟨ x , y ⟩ }
   ; from    = λ {g → λ { ⟨ x , y ⟩ → g x y } }
   ; from∘to = λ f → refl
   ; to∘from = λ g → refl
   }
\end{code}
Additionally, if there exists an $x$ for which $B(x)$ fails then it cannot be that $B(x)$ holds for every $x$:
\begin{code}
∃¬-implies-¬∀ : ∀ {A : Set}{B : A → Set} →
  ∃[ x ] (¬ B x) →
  -----------------
  ¬ (∀ x → B x)

∃¬-implies-¬∀ ⟨ x , ¬Bx ⟩ = λ f → ¬Bx (f x)
\end{code}
Finally, we have two related equivalences involving negation of products and sums:
\begin{code}
postulate
  ¬×≃+¬ : ∀ {A B : Set} → ¬ (A × B) ≃ ¬ A ⊎ ¬ B
  +¬≃¬× : ∀ {A B : Set} → ¬ (A ⊎ B) ≃ ¬ A × ¬ B
\end{code}
\textbf{Note:} In Agda the equivalence 
\[
\neg (A \uplus B) \simeq \neg A \times \neg B
\]
holds as a full isomorphism, while the equivalence 
\[
\neg (A \times B) \simeq \neg A \uplus \neg B
\]
only holds in one direction (i.e. it is an embedding rather than a full isomorphism). 
This shows the differences in how negation interacts with product types versus sum types.

\end{document}

%%% Local Variables:
%%% mode: latex
%%% TeX-master: t
%%% TeX-engine: luatex
%%% TeX-command-default: "Make"
%%% End:
