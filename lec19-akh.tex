\documentclass{lecturenotes}

\usepackage[colorlinks,urlcolor=blue]{hyperref}
\usepackage{doi}
\usepackage{xspace}
\usepackage{fontspec}
\usepackage{enumerate}
\usepackage{mathpartir}
\usepackage{pl-syntax/pl-syntax}
\usepackage{forest}
\usepackage{stmaryrd}
\usepackage{epigraph}
\usepackage{xspace}

\setsansfont{Fira Code}

\newcommand{\abs}[2]{\ensuremath{\lambda #1.\,#2}}
\newcommand{\tabs}[3]{\ensuremath{\lambda #1 \colon #2.\,#3}}
\newcommand{\dbabs}[1]{\ensuremath{\lambda.\,#1}}
\newcommand{\dbind}[1]{\ensuremath{\text{\textasciigrave}#1}}
\newcommand{\app}[2]{\ensuremath{#1\;#2}}
\newcommand{\utype}{\textsf{unit}\xspace}
\newcommand{\unit}{\ensuremath{\textsf{(}\mkern0.5mu\textsf{)}}}
\newcommand{\prodtype}[2]{\ensuremath{#1 \times #2}}
\newcommand{\pair}[2]{\ensuremath{(#1, #2)}}
\newcommand{\projl}[1]{\ensuremath{\pi_1\mkern2mu#1}}
\newcommand{\projr}[1]{\ensuremath{\pi_2\mkern3mu#1}}
\newcommand{\sumtype}[2]{\ensuremath{#1 + #2}}
\newcommand{\injl}[1]{\ensuremath{\textsf{inj}_1\mkern2mu#1}}
\newcommand{\injr}[1]{\ensuremath{\textsf{inj}_2\mkern3mu#1}}
\newcommand{\case}[5]{\ensuremath{\textsf{case}\mkern5mu#1\mkern5mu\textsf{of}\mkern5mu\injl{#2} \Rightarrow #3;\mkern5mu\injr{#4} \Rightarrow #5\mkern5mu\textsf{end}}}
\newcommand{\vtype}{\textsf{void}\xspace}
\newcommand{\vcase}[1]{\ensuremath{\textsf{case}\mkern5mu#1\mkern5mu\textsf{of}\mkern5mu\textsf{end}}}
\newcommand{\rectype}[2]{\ensuremath{\mu #1.\,#2}}
\newcommand{\roll}[1]{\textsf{roll}\mkern2mu#1}
\newcommand{\unroll}[1]{\textsf{unroll}\mkern2mu#1}

\newcommand{\FV}{\text{FV}}
\newcommand{\BV}{\text{BV}}

\newcommand{\toform}[1]{\ensuremath{\lceil #1 \rceil}}
\newcommand{\totype}[1]{\ensuremath{\lfloor #1 \rfloor}}

\newcommand{\neutral}[1]{#1\;\text{ne}}
\newcommand{\nf}[1]{#1\;\text{nf}}

\title{Recursive Types}
\coursenumber{CSE 410/510}
\coursename{Programming Language Theory}
\lecturenumber{19}
\semester{Spring 2025}
\professor{Professor Andrew K. Hirsch}

\begin{document}
\maketitle

\begin{itemize}
\item After several lectures spent studying the same four type constructors, we finally get to move to discussing something new.
\item So far, all of the types we have seen are \emph{finite}---there are a finite number of values that have that type.
\item However, real programs often need infinite types, such as data structures that can store an arbitrary amount of data.
\item In this lecture, we introduce one construct that allows for us to write all of the infinite types that you can think of---recursive types.
\end{itemize}

\section{Recursive Types by Example}
\label{sec:recurs-types-example}

\begin{itemize}
\item Think about one of the first infinite types you learned---the list.
\item In OCaml, we write the list type as such:
  \begin{center}
    {\sffamily
      type 'a list = nil | cons of 'a * 'a list
    }
  \end{center}
\item There are several aspects of this type.
\item First, note that the line represents a sum---it can be either a nil or a cons.
\item Nil contains nothing interesting, so we can represent it as a unit.
\item All together, we could write this as
  \begin{center}
    {\sffamily
      type 'a list = \sumtype{\utype}{\prodtype{\textsf{'a}}{\textsf{'a list}}}
    }
  \end{center}
\item Now this type is \emph{parameterized}, so the list can store any type of data.
  We will study parameterized types in more detail after Spring break; but for now we will specialize lists to a particular type.
  For our purposes, that type will be \utype.
  \begin{center}
    {\sffamily
      type list = \sumtype{\utype}{\prodtype{\utype}{\textsf{list}}}
    }
  \end{center}
  \begin{itemize}
  \item \textbf{Stop and Think:} How can you describe the values of this type?
  \item It's any collection of one or more $\unit$s!
  \end{itemize}
\item Finally, this type is \emph{recursive}: we use \textsf{list} in the definition of \textsf{list}.
\item In OCaml, in order to express this type you must use a type declaration, like we did above.
\item In order to express this type without a type declaration, you need a way to describe recursion without a type declaration.
\item We do that with a \emph{fixed point combinator}~$\mu$.
\item We can write $\rectype{\textsf{list}}{\sumtype{\utype}\prodtype{\utype}{\textsf{list}}}$.
\item Our aim today is to understand this combinator.
\end{itemize}

\newpage
\section{Recursive Types, Formally}
\label{sec:recurs-types-form}

\begin{syntax}
  \abstractCategory[Type Variables]{X, Y, \dots}
  \category[Types]{\tau} \alternative{\ldots} \alternative{X} \alternative{\rectype{X}{\tau}}
  \category[Expressions]{e} \alternative{\ldots} \alternative{\roll{e}} \alternative{\unroll{e}}
\end{syntax}

\begin{mathpar}
  \infer*[left=$\mu$-I]{\Gamma \vdash e : \tau[X \mapsto \rectype{X}{\tau}]}{\Gamma \vdash \roll{e} : \rectype{X}{\tau}}\and
  \infer*[left=$\mu$-E]{\Gamma \vdash e : \rectype{X}{\tau}}{\Gamma \vdash \unroll{e} : \tau[X \mapsto \rectype{X}{\tau}]}
\end{mathpar}

\begin{mathpar}
  \infer{ }{\unroll{(\roll{e})} \to e}\and
  \roll{(\unroll{e})} \cong_\eta e
\end{mathpar}

\begin{itemize}
\item We can see above that we are not adding too much to our system.
\item Types can now contain \emph{type variables}, which stand for other types.
  We can define substitution on types where $\mu$ is the only binder.
\item Types now also contain recursive types, written with a $\mu$.
  If we think of a type $\tau$ that contains a free variable $X$ as a function from types to types (so $\tau$ sends a type $\sigma$ to $\tau[X \mapsto \sigma]$), $\rectype{X}{\tau}$ is the least fixed point of $\tau$.
  \begin{itemize}
  \item Least with respect to the following ordering: $\tau \leq \sigma$ if every value of $\tau$ is also a value of $\sigma$.
    So far, we don't have a good name for this ordering; we will talk more about it next week.
  \item Remember, a fixed point of a function $f : X \to X$ is an $x \in X$ such that $f(x) = x$.
  \end{itemize}
\item In our case, we're working with types so we only work up to isomorphism: $\rectype{X}{\tau} \cong \tau[X \mapsto \rectype{X}{\tau}]$
\item The two functions that witness this isomorphism are $\textsf{roll} : \tau[X \mapsto \rectype{X}{\tau}] \to \rectype{X}{\tau}$ and\\ \mbox{$\textsf{unroll} : \rectype{X}{\tau} \to \tau[X \mapsto \rectype{X}{\tau}]$}.
\item They're not exactly functions, however, because then they'd only work for some particular $\tau$; instead, we give that \textsc{$\mu$-I} and \textsc{$\mu$-E} rules above.
\item The fact that these form an isomorphism is given by the $\beta$ and $\eta$ laws, also given above.
\end{itemize}

\section{Type Well-Formedness}
\label{sec:type-well-formedness}

\begin{itemize}
\item Let's look a bit deeper at type variables.
\item A type variable~$X$ acts a lot like the base type we had in our original formulation of STLC---there are no closed terms with the type $X$.
\item Moreover, you can form closed programs like $\vdash \tabs{x}{X}{x} : X \to X$ that don't have a clear meaning.
\item It might be tempting to say that this is a polymorphic function like you would in OCaml, but $\app{(\tabs{x}{X}{x})}{\unit}$ is not well typed, because $\unit$ does not have type $X$.
\item What's going on here?
  Well, we have a (type) variable~$X$ that is not in scope, and is therefore not meaningful.
\item We need to keep track of scope in our type system!
\item We can do that using a context.
  However, our types do not themselves have types, and so our context will just be the set of types in scope.
\item We start by defining a new relation $\Delta \vdash \tau$ which means ``when the type variables in $\Delta$ are in scope, $\tau$ is \emph{well formed}''.
  Here, a well-formed type is one that only references in-scope variables.
\end{itemize}

\begin{mathpar}
  \infer*[left=Axiom]{X \in \Delta}{\Delta \vdash X}\and
  \infer*[left=Unit]{ }{\Delta \vdash \utype}\\
  \infer*[left=Prod]{\Delta \vdash \tau_1\\ \Delta \vdash \tau_2}{\Delta \vdash \prodtype{\tau_1}{\tau_2}}\and
  \infer*[left=Sum]{\Delta \vdash \tau_1\\ \Delta \vdash \tau_2}{\Delta \vdash \sumtype{\tau_1}{\tau_2}}\and
  \infer*[left=Arrow]{\Delta \vdash \tau_1\\ \Delta \vdash \tau_2}{\Delta \vdash \tau_1 \to \tau_2}\and
  \infer*[left=Rec]{\Delta, X \vdash \tau}{\Delta \vdash \rectype{X}{\tau}}x
\end{mathpar}

\begin{itemize}
\item We say that a typing context $\Gamma$ is well-formed if every type in $\Gamma$ is well-formed.
\end{itemize}

\begin{mathpar}
  \infer*[left=Empty]{ }{\Delta \vdash\cdot} \and
  \infer*[left=Cons]{\Delta \vdash \Gamma\\ \Delta \vdash \tau}{\Delta \vdash \Gamma, x : \tau}
\end{mathpar}

\begin{itemize}
\item We then need to modify our typing system to ensure that every type is well formed.
\item In particular only two rules change: the \textsc{Axiom} and \textsc{Unit-I} rules must require that the context be well-formed in the empty type-variable context
\end{itemize}

\begin{mathpar}
  \infer*[left=Axiom]{\vdash \Gamma\\ x : \tau \in \Gamma}{\Gamma \vdash x : \tau} \and
  \infer*[left=Unit-I]{\vdash \Gamma}{\Gamma \vdash \unit : \utype}
\end{mathpar}

\begin{itemize}
\item These changes allow us to prove the following theorem:
\end{itemize}

\begin{thm}[Type Well-Formedness]
  If $\Gamma \vdash e : \tau$, then $\vdash \Gamma$ and $\vdash \tau$.
\end{thm}
\begin{proof}
  By induction on the proof of $\Gamma \vdash e : \tau$.
  For the axiom rule, because $\vdash \Gamma$ by fiat and $x : \tau \in \Gamma$, $\vdash \tau$ holds.
  For the unit rule, $\vdash \Gamma$ by fiat, and $\vdash \utype$ is always true.
  Every other rule has a recursive case that allows us to get the information we need out of the IH.
\end{proof}

\section{More Examples}
\label{sec:more-examples}

\subsection{Lists}
\label{sec:lists}

\begin{itemize}
\item $\textsf{list}\,A = \rectype{X}{\sumtype{\utype}{\prodtype{A}{X}}}$
\item $\textsf{nil} = \roll{(\injl{\unit})}$
\item $\textsf{cons} = \tabs{x}{A}{\tabs{xs}{\textsf{list}\,A}{\roll{(\injr{\pair{x}{xs}})}}}$
\item $\textsf{case}\,xs \mathrel{\textsf{of}} \textsf{nil} \Rightarrow e_1; \app{\app{\textsf{cons}}{x}}{xs} \Rightarrow e_2 =$\\
  $\case{(\unroll{xs})}{\_}{e_1}{y}{e_2[x \mapsto \projl{y}, xs \mapsto \projr{y}]}$
\end{itemize}

\subsection{Nats}
\label{sec:nats}

\begin{itemize}
\item $\mathbb{N} = \rectype{X}{\sumtype{\unit}{X}}$
\item $0 = \roll{(\injl{\unit})}$
\item $\textsf{suc} = \tabs{x}{\mathbb{N}}{\roll{(\injr{x})}}$
\item $\textsf{case}\,n \mathrel{\textsf{of}} 0 \Rightarrow e_1; \app{\textsf{suc}}{m} \Rightarrow e_2 = $\\
  $\case{(\unroll{n})}{\_}{e_1}{m}{e_2}$
\end{itemize}

\subsection{Trees}
\label{sec:trees}

\begin{itemize}
\item $\textsf{tree}\,A = \rectype{X}{\sumtype{\utype}{\prodtype{X}{\prodtype{A}{X}}}}$
\item $\textsf{leaf} = \roll{(\injl{\unit})}$
\item $\textsf{branch} = \tabs{l}{\textsf{tree}\,A}{\tabs{a}{A}{\tabs{r}{\textsf{tree}\,A}{\roll{(\injr{\pair{l}{\pair{a}{r}}})}}}}$
\item $\textsf{case}\,t \mathrel{\textsf{of}} \textsf{leaf} \Rightarrow e_1; \app{\app{\app{\textsf{branch}}{l}}{a}}{r} \Rightarrow e_2 =$\\
  $\case{\unroll{t}}{\_}{e_1}{x}{e_2[l \mapsto \projl{x}, a \mapsto \projl{(\projr{x})}, r \mapsto \projr{(\projr{x})}]}$
\end{itemize}

\section{The Return of Untyped $\lambda$~Calculus}
\label{sec:return-untyp-lambda}

\begin{itemize}
\item Consider the term $\omega_f = \abs{x}{\app{f}{(\app{(\unroll{x})}{x})}}$.
\item It's easy to see that $f : \tau \to \tau \vdash \omega_f : (\rectype{X}{X \to \tau}) \to \tau$.
\item Thus, $\textsf{fix} \triangleq \tabs{f}{\tau \to \tau}{\omega_f} : (\tau \to \tau) \to \tau$.
\item This is a least-fixed-point combinator for $\lambda$ terms: it runs a loop.
\item For instance, we can write:
  $$\app{\textsf{fix}}{\left(\tabs{f}{\mathbb{N} \to \mathbb{N}}{\tabs{x}{\mathbb{N}}{\textsf{case}\,n \mathrel{\textsf{of}} 0 \Rightarrow 1; \app{\textsf{suc}}{0} \Rightarrow 1; \app{\textsf{suc}}{(\app{\textsf{suc}}{m})} \Rightarrow \app{\app{\textsf{plus}}}{(\app{f}{m})}{(\app{f}{(\app{\textsf{suc}}{m})})}}}\right)}$$
  \begin{itemize}
  \item Assume that $f$ has its normal meaning, as does the three-way pattern matching here.
  \item \textbf{Stop and Think:} What is this function?
  \item Answer: the fibonacci function!
  \end{itemize}
\item So we have the ability to write recursive functions now.
\item Importantly, we also have $$\app{\textsf{fix}}{(\tabs{f}{\utype \to \utype}{\tabs{x}{\utype}{\app{f}{x}}})}$$ which is the function which just runs an infinite loop.
\item So, did we get the full power of untyped $\lambda$~calculus?
  Yes!
\item Define $D \triangleq \rectype{X}{(X \to X)}$.
\item Then, we can perform the following translation from untyped $\lambda$~calculus to typed $\lambda$~calculus:
  $$\llbracket e \rrbracket = \left\{\begin{array}{ll}
    x & \textsf{if}~e = x\\
    \roll{(\tabs{x}{D}{\llbracket e' \rrbracket})} & \textsf{if}~e = \abs{x}{e'}\\
    \app{(\unroll{e_1})}{e_2} & \textsf{if}~e = \app{e_1}{e_2}
  \end{array}\right.
$$
\item You will be asked to prove the following theorems in your homework:
\end{itemize}

\begin{thm}
  Let $\Gamma_S = \{x : D \mid x \in X\}$.
  Then $\Gamma_{\FV(e)} \vdash \llbracket e \rrbracket  : D$ for every untyped $\lambda$~calculus term~$e$.
\end{thm}

\begin{thm}
  If $e \to e'$ in call-by-value untyped $\lambda$~calculus, then $\llbracket e \rrbracket \to \llbracket e' \rrbracket$.
\end{thm}

\begin{thm}
  If $\llbracket e \rrbracket \to \llbracket e' \rrbracket$, then $e \to e'$ in call-by-value untyped $\lambda$~calculus.
\end{thm}

\end{document}

%%% Local Variables:
%%% mode: latex
%%% TeX-master: t
%%% TeX-engine: luatex
%%% End:
