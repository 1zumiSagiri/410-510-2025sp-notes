\documentclass{lecturenotes}

\usepackage[colorlinks,urlcolor=blue]{hyperref}
\usepackage{doi}
\usepackage{xspace}
\usepackage{agda}
\usepackage{fontspec}
\usepackage{enumerate}
\setsansfont{Fira Code}
\usepackage{newunicodechar}
\newunicodechar{∣}{\ensuremath{\mid}}

\newcommand{\agdanats}{\textsf{ℕ}\xspace}
\newcommand{\agdabool}{\textsf{𝔹}\xspace}
\newcommand{\agdacons}{\textsf{∷}\xspace}


\title{Inductive Data Types}
\coursenumber{CSE 410/510}
\coursename{Programming Language Theory}
\lecturenumber{3}
\semester{Spring 2025}
\professor{Professor Andrew K. Hirsch}


\begin{document}
\maketitle

As we did in lecture 02, we start with importing some things from the standard library. 

\begin{center}
    \begin{code}
        module lec03 where
        
        import Relation.Binary.PropositionalEquality as Eq
        open Eq using (_≡_; cong; sym; refl)
        open Eq.≡-Reasoning using (begin_; step-≡-∣; step-≡-⟩; _∎)
        open import Data.Nat using (ℕ;_+_;_*_;suc;zero)
    \end{code}
\end{center}


\section{Boolean Data Types}
\label{sec:boolean-data-types}

Those used to OCaml likely are familiar with the concept of variants/algebraic data types.
For instance, in OCaml we might define:

\begin{center}
    \textsf{type bool = true | false}
\end{center}

The above denotes a new type, bool, which should be immediately familiar.
As expected, this type carries two possible values, true and false.
One might wonder, however, how this would translate to Agda.
In Agda, this definition would take the form:

\begin{center}
    \begin{code}
        data 𝔹 : Set where
            true : 𝔹
            false : 𝔹
    \end{code}
\end{center}

Here, we are defining an algebraic data type called \agdabool which represents the type of boolean.
As we did with \agdanats in the previous lecture, we can begin writing code which acts on this new data type.
For instance, implementations for logical AND and logical OR would take the form:

\begin{center}
    \begin{code}
        _&&_ : 𝔹 -> 𝔹 -> 𝔹
        true && b' = b'
        false && b' = false 

        _||_ : 𝔹 -> 𝔹 -> 𝔹 
        true || b' = true 
        false || b' = b' 
    \end{code}
\end{center}

Certainly, this is interesting. However, the code above could also be implemented in OCaml.
As we discussed last class, Agda's infix is significantly more powerful than OCaml's.
To demonstrate this, let's implement our own version of conditional branching, which would take the form:

\begin{center}
    \begin{code}
        if_then_else_ : ∀ {A : Set} -> 𝔹 -> A -> A -> A
        if true then x else y = x 
        if false then x else y = y
    \end{code}
\end{center}

\pagebreak

The type signature above indicates that for any type A, our function takes a boolean condition and two values of type A, returning a value of type A.
Here, we begin to see the power of Agda.

\vspace{0.2in}

\noindent Stop and Think: How do we prove something about code which takes a boolean as an input?

\noindent Answer: Case analysis. Prove each case separately until each case has been exhausted and the proof is complete. 

\vspace{0.2in}

Consider the following simple example:

\begin{center}
    \begin{code}
        &&-comm : ∀ (b1 b2 : 𝔹) -> b1 && b2 ≡ b2 && b1
        &&-comm true true = refl
        &&-comm true false = refl
        &&-comm false true = refl
        &&-comm false false = refl
    \end{code}
\end{center}

Above, we have proven that the \textsf{\&\&} infix operator is commutative. 
Note that this operator takes two boolean arguments, which results in a total of four possible cases.
Each case is proven individually, each requiring only simple computations allowing us to simply use the \textsf{refl} keyword. 
Despite the simplicity of this approach, the proof is complete and type checks in Agda without error. 
Now that we have seen a simple proof, let's turn to inductive reasoning.

\vspace{0.2in}

\noindent Stop and Think: What is the induction principle for booleans?

\noindent Answer: The induction principle consists of two base cases. Specifically: 

\begin{center}
    \begin{code}
        bool-ind : ∀ (P : 𝔹 -> Set) -> P true -> P false -> ∀ (x : 𝔹) -> P(x)
        bool-ind P Ptrue Pfalse true = Ptrue 
        bool-ind P Ptrue Pfalse false = Pfalse
    \end{code}
\end{center}

\vspace{0.2in}

\noindent Let's compare this to the induction principle we saw for the natural numbers in the previous lecture: 

\begin{center}
    \begin{code}
        nat-ind : ∀ (P : ℕ -> Set) -> P zero -> (∀ n -> P n -> P (suc n)) -> ∀ n -> P n
        nat-ind P Pz Ps zero = Pz
        nat-ind P Pz Ps (suc n) = Ps n (nat-ind P Pz Ps n)
    \end{code}
\end{center}

Why do these two induction principles differ? 
Well, the way the data types are constructed differ.
This requires that we also change the way we provide evidence.
This is the secret sauce of induction!

\vspace{0.2in}

\section{Lists}
\label{sec:list-types}

What about a more complicated data type? Lists.
In Ocaml, we would say \textsf{'a list}. With Agda, however, we say:

\begin{center}
    \begin{code}
        data List (A : Set) : Set where
            [] : List A
            _∷_ : A -> List A -> List A

        infixr 5 _∷_
    \end{code}
\end{center}

\pagebreak

Be careful, the character (\agdacons) is not the same as two colons (::).
Try not to get the two confused, as this is a frequent source of errors.

\vspace{0.2in}

Note that the definition above is actually just a function from set to set. 
We can write the code differently to make this more clear:

\begin{center}
    \begin{code}
        data List' : Set -> Set₁ where
            [] : ∀ {A : Set} -> List' A
            _∷_ : ∀ {A : Set} -> List' A -> List' A
    \end{code}
\end{center}

\noindent While this new definition certainly looks different, it is equivalent to the first. 

\vspace{0.2in}

Similar to the natural numbers, Lists are important enough to have their own builtin pragma that makes things more efficient:

\begin{center}
    \begin{code}
        {-# BUILTIN LIST List #-}
    \end{code}
\end{center}

But how does one work with lists?
At the most basic level, we can write code such as: 

\begin{center}
    \begin{code}
        _ : List ℕ
        _ = 0 ∷ 1 ∷ 2 ∷ []
    \end{code}
\end{center}

This works, but is tedious to write. 
Likely, we will not want to write this much to define a list of any significant length.
Instead, we can make use of patterns:

\begin{center}
    \begin{code}
        pattern [_] z = z ∷ []
        pattern [_,_] y z = y ∷ z ∷ []
        pattern [_,_,_] x y z = x ∷ y ∷ z ∷ []
        pattern [_,_,_,_] w x y z = w ∷ x ∷ y ∷ z ∷ []

        _ : List ℕ
        _ = [ 0 , 1 , 2 ]
    \end{code}
\end{center}

Patterns make it much easier to work with lists.
Pay special attention to the whitespace in the example above, however, as it is needed for the code to type check properly.

\vspace{0.2in}

\noindent Can I do things with lists? Sure! Lets look at concatenation and reverse.

\begin{center}
    \begin{code}
        infixr 5 _++_

        _++_ : ∀ {A : Set} -> List A -> List A -> List A
        [] ++ ys = ys 
        (x ∷ xs) ++ ys = x ∷ (xs ++ ys)

        reverse : ∀ {A : Set} -> List A -> List A
        reverse [] = [] 
        reverse (x ∷ xs) = reverse xs ++ [ x ]
    \end{code}
\end{center}

\pagebreak

How can we prove things about lists? 
As with booleans and natural numbers, we will need an induction princple. 
Let's work through an example to see how induction with lists works.
Consider the following definition of length:

\begin{center}
    \begin{code}
        length : ∀ {A : Set} -> List A -> ℕ
        length [] = 0
        length (x ∷ xs) = suc (length xs)
    \end{code}
\end{center}

Note that this is code is recursive.
Since we do induction over recursion, we can do induction over a list. 
The following proof template provides a formula you can follow to complete an induction proof on a list:

\vspace{0.2in}

\noindent Theorem: For every list \textsf{l} of type \textsf{A}, P(\textsf{l})

\noindent Proof: By (structural) induction on \textsf{l} 

\vspace{0.1in}

\noindent Base Case: \textsf{l} = \textsf{[]} 

\noindent Show: P(\textsf{[]}) 

\vspace{0.1in}

\noindent Inductive Case: \textsf{l} = \textsf{x} \agdacons \textsf{xs} 

\noindent Inductive Hypothesis: P(\noindent{xs}) 

\noindent Show: P(\textsf{x} \agdacons \textsf{xs}) 

\vspace{0.1in}

\noindent Therefore P(\noindent{l}) holds for any list \noindent{l}.

\noindent QED

\vspace{0.2in}

Where does this proof template come from? From the constructors!
Consider the constructor we defined for lists. 
First, I have the empty list - this is the base case. 
Next, I have cons, which takes a list A as an argument. 
This creates my inductive case and provides me with an inductive hypothesis. 

\vspace{0.2in}

\noindent Can we write some proofs with this?
Sure! Let's prove \textsf{++} is associative:

\begin{center}
    \begin{code}
        ++-assoc : ∀ {A : Set} (xs ys zs : List A) -> xs ++ (ys ++ zs) ≡ (xs ++ ys) ++ zs
        ++-assoc [] ys zs = refl
        ++-assoc (x ∷ xs) ys zs =
            begin 
                (x ∷ xs) ++ (ys ++ zs)
            ≡⟨⟩
                x ∷ (xs ++ (ys ++ zs))
            ≡⟨ cong (x ∷_) (++-assoc xs ys zs) ⟩
                x ∷ (xs ++ ys) ++ zs
            ∎
    \end{code}
\end{center}

\vspace{0.2in}

\section{Trees}
\label{sec:tree-types}

Quickly, there is one last data structure to cover today: trees.

\begin{center}
    \begin{code}
        data Tree (A : Set) : Set where
            Leaf : Tree A 
            Node : Tree A -> A -> Tree A -> Tree A
    \end{code}
\end{center}

\pagebreak

How do I do induction on a Tree? 
As we have said for each of the data structures before, it depends on my constructor. 
Note the fundamental differences comparing the constructor we just defined for Tree to that for List. 
With a list, we either had nil or cons, where cons contained a single additional List component.
Here, however, I have a Leaf and a Node.
Note that Node consists of not one, but two additional Node components.

\vspace{0.2in}

\noindent Stop and Think: How does this change my approach to induction?

\noindent Answer: I get not one IH, but two! Why? Because I have two components to the node structure. 
There are two Tree arguments to node. For every argument, we get an inductive hypothesis. 

\vspace{0.2in}

\begin{center}
    \begin{code}
        tree-ind : ∀ {A : Set} (P : Tree A -> Set) ->
            P Leaf -> 
            (∀ (l r : Tree A) (x : A) -> P l -> P r -> P (Node l x r)) ->
            (t : Tree A) -> P t
        tree-ind P Pleaf Pnode Leaf = Pleaf 
        tree-ind P Pleaf Pnode (Node l x r) = Pnode l r x (tree-ind P Pleaf Pnode l) (tree-ind P Pleaf Pnode r)
    \end{code}
\end{center}

\noindent Big takeaways: Whenever you have a data type defined by this data keyword (called inductive types), you can reason about the type using induction.
How? Simply read off the construction of the type. 
Second, if you  are comfortable with recursion, you should be confortable with induction.
Induction is just recursion in a pretty thin disguise. 

\end{document}

%%% Local Variables:
%%% mode: latex
%%% TeX-master: t
%%% TeX-engine: luatex
%%% TeX-command-default: "Make"
%%% End: