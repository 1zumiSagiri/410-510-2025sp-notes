\documentclass{lecturenotes}

\usepackage[colorlinks,urlcolor=blue]{hyperref}
\usepackage{doi}
\usepackage{xspace}
\usepackage{fontspec}
\usepackage{enumerate}
\usepackage{mathpartir}
\usepackage{pl-syntax/pl-syntax}
\usepackage{forest}
\usepackage{stmaryrd}
\usepackage{epigraph}
\usepackage{xspace}

\setsansfont{Fira Code}

\newcommand{\abs}[2]{\ensuremath{\lambda #1.\,#2}}
\newcommand{\tabs}[3]{\ensuremath{\lambda #1 \colon #2.\,#3}}
\newcommand{\dbabs}[1]{\ensuremath{\lambda.\,#1}}
\newcommand{\dbind}[1]{\ensuremath{\text{\textasciigrave}#1}}
\newcommand{\app}[2]{\ensuremath{#1\;#2}}
\newcommand{\utype}{\textsf{unit}\xspace}
\newcommand{\unit}{\ensuremath{\textsf{(}\mkern0.5mu\textsf{)}}}
\newcommand{\prodtype}[2]{\ensuremath{#1 \times #2}}
\newcommand{\pair}[2]{\ensuremath{(#1, #2)}}
\newcommand{\projl}[1]{\ensuremath{\pi_1\mkern2mu#1}}
\newcommand{\projr}[1]{\ensuremath{\pi_2\mkern3mu#1}}
\newcommand{\sumtype}[2]{\ensuremath{#1 + #2}}
\newcommand{\injl}[1]{\ensuremath{\textsf{inj}_1\mkern2mu#1}}
\newcommand{\injr}[1]{\ensuremath{\textsf{inj}_2\mkern3mu#1}}
\newcommand{\case}[5]{\ensuremath{\textsf{case}\mkern5mu#1\mkern5mu\textsf{of}\mkern5mu\injl{#2} \Rightarrow #3;\mkern5mu\injr{#4} \Rightarrow #5\mkern5mu\textsf{end}}}
\newcommand{\vtype}{\textsf{void}\xspace}
\newcommand{\vcase}[1]{\ensuremath{\textsf{case}\mkern5mu#1\mkern5mu\textsf{of}\mkern5mu\textsf{end}}}
\newcommand{\rectype}[2]{\ensuremath{\mu #1.\,#2}}
\newcommand{\roll}[1]{\textsf{roll}\mkern2mu#1}
\newcommand{\unroll}[1]{\textsf{unroll}\mkern2mu#1}

\newcommand{\FV}{\text{FV}}
\newcommand{\BV}{\text{BV}}

\newcommand{\toform}[1]{\ensuremath{\lceil #1 \rceil}}
\newcommand{\totype}[1]{\ensuremath{\lfloor #1 \rfloor}}

\newcommand{\neutral}[1]{#1\;\text{ne}}
\newcommand{\nf}[1]{#1\;\text{nf}}

\newcommand{\subtype}{\ensuremath{\mathrel{\mathord{<}\mathord{:}}}}

\title{Subtyping}
\coursenumber{CSE 410/510}
\coursename{Programming Language Theory}
\lecturenumber{20}
\semester{Spring 2025}
\professor{Professor Andrew K. Hirsch}

\begin{document}
\maketitle

\begin{itemize}
\item Up until now, all of our types have been very precise.
  In fact, every given term has exactly one type.
\item This week, we will be studying systems where one term might have multiple types.
\item We start with the most famous and commonly-used version: subtyping.
\item The idea is to develop a relation $\tau \subtype \tau'$ that means ``every term of type $\tau$ also has type $\tau'$.''
\item In order to use it, we start by introducing record types.
\end{itemize}

\section{Record Types}
\label{sec:record-types}

\begin{syntax}
  \abstractCategory[Labels]{\ell}
  \category[Types]{\tau} \alternative{\utype} \alternative{\tau_1 \to \tau_2} \alternative{\prodtype{\tau_1}{\tau_2}} \alternative{\sumtype{\tau_1}{\tau_2}} \alternative{\{\ell_1 : \tau_1; \dots; \ell_n : \tau_n\}}
  \category[Expressions]{e} \alternative{\dots} \alternative{\{\ell_1 = e_1; \dots; \ell_n = e_n\}} \alternative{e.\ell}
\end{syntax}

\begin{mathpar}
  \{\ell_1 = e_1; \dots; \ell_n = e_n\}.\ell_i \to_\beta e_i \and
  e \equiv_\eta \{\ell_1 = e.\ell_1; \dots; \ell_n = e.\ell_n\}
\end{mathpar}

\begin{mathpar}
  \infer*[left=Record-I]{\forall i \in \{1, \dots, n\}.\;\Gamma \vdash e_i : \tau_i}{\{\ell_1 = e_1; \dots; \ell_n = e_n\} : \{\ell_1 : \tau_1; \dots; \ell_n : \tau_n\}}\and
  \infer*[left=Record-E]{\Gamma \vdash e : \{\ell_1 : \tau_1; \dots; \ell_n : \tau_n\}}{\Gamma \vdash e.\ell_i : \tau_i}
\end{mathpar}

\begin{itemize}
\item You've seen records before in agda and OCaml.
\item They act like product types, but they don't care about order.
  Instead, the projections are named explicitly by labels $\ell$.
\item Thus, the type $\{x : \mathbb{N}, y : \mathbb{N}\}$ is the type of records with \emph{exactly} two natural numbers labeled by $x$ and $y$, respectively.
\item In Agda and OCaml, the order of elements does not matter for records.
\item However, here is matters: in \textsc{Record-I}, the expression and the type must have exactly the same labels in the same order.
\item In the next section, we will use subtyping to get rid of this requirement.
\end{itemize}

\section{Subtyping}
\label{sec:subtyping}

\begin{itemize}
\item Again, our goal is to build a relation $\tau \subtype \tau'$ that means ``every term of type $\tau$ also has type $\tau'$.''
\item The most important goal for this principle is the \emph{Liskov Substitution Principle}, named after Turing laureate Barbara Liskov:
\end{itemize}

\begin{defn}[Liskov Substitution Principle]
  If $\tau \subtype \tau'$ then in any context that demands a $\tau'$, a $\tau$ should be usable.
\end{defn}

\begin{itemize}
\item For example, if I need an animal in order to do something, and you provide me with a mammal, I should be okay with that.
\item For the type system we explored in the last section, we can define $\subtype$ as follows:
\end{itemize}

\begin{mathpar}
  \infer*[left=UnitRefl]{ }{\utype \subtype \utype} \and
  \infer*[left=ProdSub]{\tau_1 \subtype \tau'_1\\  \tau_2 \subtype \tau'_2}{\prodtype{\tau_1}{\tau_2} \subtype \prodtype{\tau'_1}{\tau'_2}} \and
  \infer*[left=SumSub]{\tau_1 \subtype \tau'_1\\  \tau_2 \subtype \tau'_2}{\sumtype{\tau_1}{\tau_2} \subtype \sumtype{\tau'_1}{\tau'_2}} \\
  \infer*[left=ArrSub]{\tau'_1 \subtype \tau_1\\  \tau_2 \subtype \tau'_2}{\tau_1 \to \tau_2 \subtype \tau'_1 \to \tau'_2} \\
  \infer*[left=RecordSub]{\exists f : \{1, \dots, m\} \hookrightarrow \{1, \dots, n\}. \forall i \in \{1, \dots, m\}.\; \ell_{f(i)} = \ell'_i \land \tau_{f(i)} \subtype \tau'_i}{\{\ell_1 : \tau_1; \dots; \ell_n : \tau_n\} \subtype \{\ell'_1: \tau'_1; \dots; \ell'_m : \tau'_m\}}
\end{mathpar}

\begin{itemize}
\item The first three rules in this definition are pretty obvious.
\item The rule for arrow types is less obvious: note that the subtyping relation for the arguments is reversed.
  \begin{itemize}
  \item To see why this is the correct thing to do, consider that a function that can work for any animal is also one that can work for any mammal.
    However, a function that can work for any mammal may not work for any animal; for instance, it may do something like return the chemical formulation for the milk of the mammal, but snakes don't have milk.
  \item This is in many ways ``just'' an application of the Liskov Substitution Principle: a more-precise function type promises to be able to do fewer things, which means it can take more things as a result!
  \end{itemize}
\item The rule for records is also potentially confusing: we require an injective mathematical function $f : \{1, \dots, n\} \hookrightarrow \{1, \dots, m\}$ and use that to match up labels.
  \begin{itemize}
  \item The notation $X \hookrightarrow Y$ just means ``injective function from $X$ to $Y$.''
  \item Because $f$ is injective, we know that $m < n$.
  \item Every label~$\ell_I$ in the subtype must exist in the supertype as $\ell_{f(i)}$, and the type $\tau_i$ must be a subtype of $\tau_{f(i)}$
  \end{itemize}
\item Our subtyping relation admits the following rules:
\end{itemize}

\begin{mathpar}
  \infer*[left=SubRefl]{ }{\tau \subtype \tau}\and
  \infer*[left=SubTrans]{\tau_1 \subtype \tau_2\\ \tau_2 \subtype \tau_3}{\tau_1 \subtype \tau_3}\\
  \infer*[left=RecordRearrange]{ }{\{\ell_1 : \tau_1; \dots; \ell_i : \tau_i; \ell_{i+1} : \tau_{i+1}; \dots; \ell_n : \tau_n\} \subtype \{\ell_1 : \tau_1; \dots; \ell_{i+1} : \tau_{i+1}; \ell_i : \tau_i; \dots; \ell_n : \tau_n\}} \and
  \infer*[left=RecordDepth]{\forall i \in \{1, \dots, n\}.\; \tau_i \subtype \tau'_i}{\{\ell_1 : \tau_1; \dots; \ell_n : \tau_n\} \subtype \{\ell_1 : \tau'_1; \dots; \ell_n : \tau'_n\}} \and
  \infer*[left=RecordBreadth]{ }{\{\ell_1 : \tau_1; \dots; \ell_n : \tau_n; \ell_{n+1} : \tau_{n+1}; \dots; \ell_m : \tau_m\} \subtype \{\ell_1 : \tau_1; \dots; \ell_n : \tau_n\}}
\end{mathpar}

\begin{itemize}
\item The first two say that subtyping is a preorder.
  \begin{itemize}
  \item It is \emph{not} a partial order: if $\tau_1 \subtype \tau_2$ and $\tau_2 \subtype \tau_1$ then they have the same values, but they are not necessarily (syntactically) the same type.
    We write $\tau_1 \equiv \tau_2$ if $\tau_1 \subtype \tau_2$ and $\tau_2 \subtype \tau_1$, and say that $\tau_1$ and $\tau_2$ are equivalent.
  \item   An example of two equivalent types are $\{x : \mathbb{N}; y : \mathbb{N}\} \equiv \{y : \mathbb{N}; x : \mathbb{N}\}$.
    In fact, any rearrangement of labels in a record type gives an equivalent record type.
  \item As with any preorder, $\equiv$ is an equivalence relation.
  \end{itemize}
\item The last three are all about records:
  \begin{itemize}
  \item The first allows us to rearrange the labels in a record type.
  \item The second allows us to go ``deeper'' inside of a record type at every label.
  \item The last one allows us to remove labels to a record type, ``hiding'' them from view.
    Think about it this way: there are more things that offer an~$x$ and a~$y$ then those that offer an~$x$, a~$y$, and a~$z$.
  \end{itemize}
\item Right now, our language does not follow the Liskov Substitution Principle.
\item After all, we cannot use record types wherever we want; they have to match precisely!
\item There are two ways to fix this.
\item First, we can just declare by fiat that we will allow subtypes.
\end{itemize}

\begin{mathpar}
  \infer*[left=Sub]{\Gamma \vdash e : \tau\\ \tau \subtype \tau'}{\Gamma \vdash e : \tau'}
\end{mathpar}

\begin{itemize}
\item This is the most common thing to do, and it's very easy from the perspective of language design.
\item However, it can often be hard to prove things about and to implement.
\item Instead, we can ``bake in'' the subtyping rules to our typing rules.
\item In our case, that only requires changing the \textsc{Record-I} rule:
\end{itemize}

\begin{mathpar}
  \infer*[left=Record-I]{\exists f : \{1, \dots, m\} \hookrightarrow \{1, \dots, n\}.\; \forall i \in \{1, \dots, m\}.\; \ell_{f(i)} = \ell'_i \land \Gamma \vdash e_{f(i)} : \tau_i}{\Gamma \vdash \{\ell_1 = e_1; \dots; \ell_m = e_m\} : \{\ell'_1 : \tau_1; \ldots; \ell'_n : \tau_n\}}
\end{mathpar}

\begin{itemize}
\item If we take this option, then the \textsc{Sub} rule above becomes admissible.
\end{itemize}

\section{Recursive Typing, Redux}
\label{sec:recurs-typing-redux}

\begin{itemize}
\item Last time, we looked at \emph{isorecursive} types: we had an isomorphism $\textsf{unroll} : \rectype{X}{\tau} \cong\tau[X \mapsto \rectype{X}{\tau}] : \textsf{roll}$.
\item However, we often don't want to write the roll and unroll terms; instead, we want to treat $\rectype{X}{\tau}$ and $\tau[X \mapsto \rectype{X}{\tau}]$ as equivalent.
\item Now we have an appropriate notion of equivalence!
\item We can add type variables and recursive types \emph{without} adding \textsf{roll} and \textsf{unroll}, but then we have to add the following to our subtyping rules:
\end{itemize}

\begin{mathpar}
  \infer*[left=$\mu$-Left]{ }{\rectype{X}{\tau} \subtype \tau[X \mapsto \rectype{X}{\tau}]} \and
  \infer*[left=$\mu$-Right]{ }{\tau[X \mapsto \rectype{X}{\tau}] \subtype \rectype{X}{\tau}}
\end{mathpar}

\begin{itemize}
\item This works best if you've added \textsc{Sub} as a rule to your type system, rather than making it admissible (though there are ways around that).
\item Now, you can simply use a recursive type wherever you have its unfolding and vice-versa!
\end{itemize}

\end{document}

%%% Local Variables:
%%% mode: latex
%%% TeX-master: t
%%% TeX-engine: luatex
%%% End:
