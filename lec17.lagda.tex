\documentclass{lecturenotes}

\usepackage[colorlinks,urlcolor=blue]{hyperref}
\usepackage{doi}
\usepackage{xspace}
\usepackage{fontspec}
\usepackage{enumerate}
\usepackage{mathpartir}
\usepackage{pl-syntax/pl-syntax}
\usepackage{forest}
\usepackage{stmaryrd}
\usepackage{epigraph}
\usepackage{xspace}

\setsansfont{Fira Code}

\newcommand{\abs}[2]{\ensuremath{\lambda #1.\,#2}}
\newcommand{\tabs}[3]{\ensuremath{\lambda #1 \colon #2.\,#3}}
\newcommand{\dbabs}[1]{\ensuremath{\lambda.\,#1}}
\newcommand{\dbind}[1]{\ensuremath{\text{\textasciigrave}#1}}
\newcommand{\app}[2]{\ensuremath{#1\;#2}}
\newcommand{\utype}{\textsf{unit}\xspace}
\newcommand{\unit}{\ensuremath{\textsf{(}\mkern0.5mu\textsf{)}}}
\newcommand{\prodtype}[2]{\ensuremath{#1 \times #2}}
\newcommand{\pair}[2]{\ensuremath{(#1, #2)}}
\newcommand{\projl}[1]{\ensuremath{\pi_1\mkern2mu#1}}
\newcommand{\projr}[1]{\ensuremath{\pi_2\mkern3mu#1}}
\newcommand{\sumtype}[2]{\ensuremath{#1 + #2}}
\newcommand{\injl}[1]{\ensuremath{\textsf{inj}_1\mkern2mu#1}}
\newcommand{\injr}[1]{\ensuremath{\textsf{inj}_2\mkern3mu#1}}
\newcommand{\case}[5]{\ensuremath{\textsf{case}\mkern5mu#1\mkern5mu\textsf{of}\mkern5mu\injl{#2} \Rightarrow #3;\mkern5mu\injr{#4} \Rightarrow #5\mkern5mu\textsf{end}}}
\newcommand{\vtype}{\textsf{void}\xspace}
\newcommand{\vcase}[1]{\ensuremath{\textsf{case}\mkern5mu#1\mkern5mu\textsf{of}\mkern5mu\textsf{end}}}

\newcommand{\FV}{\text{FV}}
\newcommand{\BV}{\text{BV}}

\title{Type System Organization}
\coursenumber{CSE 410/510}
\coursename{Programming Language Theory}
\lecturenumber{17}
\semester{Spring 2025}
\professor{Professor Andrew K. Hirsch}

\begin{document}
\maketitle

\section{Introduction and Elimination Rules}
$$\begin{array}{c|c}
  \multicolumn{2}{c}{\infer*[left=Axiom]{x \colon \tau \in \Gamma}{\Gamma \vdash x \colon \tau}}\\[2em]
  \infer*[left=arrI]{\Gamma, x \colon \tau_1 \vdash e \colon \tau_2}{\Gamma \vdash \tabs{x}{\tau_1}{e} \colon \tau_1 \to \tau_2} & \infer*[right=arrE]{\Gamma \vdash f \colon \tau_1 \to \tau_2\\ \Gamma \vdash e \colon \tau_1}{\Gamma \vdash \app{f}{e} \colon \tau_2}\\[2em]
  \infer*[left=unitI]{ }{\Gamma \vdash \unit \colon \utype} & \\[2em]
                                                                             & \infer*[right=voidE]{\Gamma \vdash e \colon \vtype}{\Gamma \vdash \vcase{e} \colon \tau}\\[2em]
  \infer*[left=prodI]{\Gamma \vdash e_1 \colon \tau_1\\ \Gamma \vdash e_2 \colon \tau_2}{\Gamma \vdash (e_1, e_2) \colon \prodtype{\tau_1}{\tau_2}} & \infer*[right=prodE${}_1$]{\Gamma \vdash e \colon \prodtype{\tau_1}{\tau_2}}{\Gamma \vdash \projl{e} \colon \tau_1} \hspace{1em} \infer*[right=prodE${}_2$]{\Gamma \vdash e \colon \prodtype{\tau_1}{\tau_2}}{\Gamma \vdash \projr{e} \colon \tau_2}\\[2em]
  \infer*[left=sumI${}_1$]{\Gamma \vdash e \colon \tau_1}{\Gamma \vdash \injl{e} \colon \sumtype{\tau_1}{\tau_2}} \hspace{1em} \infer*[left=sumI${}_2$]{\Gamma \vdash e \colon \tau_2}{\Gamma \vdash \injr{e} \colon \sumtype{\tau_1}{\tau_2}} &\infer*[right=sumE]{\Gamma \vdash e_1 \colon \sumtype{\tau_1}{\tau_2}\\ \Gamma, x \colon \tau_1 \vdash e_2 \colon \tau_3\\ \Gamma, y \colon \tau_2 \vdash e_3 \colon \tau_3}{\Gamma \vdash \case{e_1}{x}{e_2}{y}{e_3} \colon \tau_3}\\
\end{array}$$

\section{$\beta$ reduction}
Just to remind ourselves, this is the $\beta$ law:
  $$\app{(\tabs{x}{\tau_1}{e_1})}{e_2} \equiv_\beta e_1[x \mapsto e_2]$$

\noindent The typing derivation of the left-hand side is:
  $$\begin{array}{c|c}
    $$\app{(\tabs{x}{\tau_1}{e_1})}{e_2} \equiv_\beta e_1[x \mapsto e_2]$$
    &
    \infer*[left=arrE]{\infer*[left=arrI]{\Gamma, x \colon \tau_1 \vdash e_1 \colon \tau_2}{\Gamma \vdash \tabs{x}{\tau_1}{e_1} \colon \tau_1 \to \tau_2}\\ \Gamma \vdash e_2 \colon \tau_1}{\Gamma \vdash \app{(\tabs{x}{\tau_1}{e_1})}{e_2} \vdash \tau_2}
\end{array}$$\\

\noindent Similarly, the $\beta$ law can be used for the product rules:
  $$\begin{array}{c|c}
    $$\projl{\pair{e_1}{e_2}} \equiv_\beta e_1 \colon$$
    &
    \infer*[left=projE${}_1$]{\infer*[left=prodI]{\Gamma \vdash e_1 : \tau_1\\ \Gamma \vdash e_2 \colon \tau_2}{\Gamma \vdash \pair{e_1}{e_2} \colon \prodtype{\tau_1}{\tau_2}}}{\Gamma \vdash \projl{\pair{e_1}{e_2}} \colon \tau_1}\\
\end{array}$$ 

\noindent As well as for sum: 
    $$\case{\injl{e_1}}{x}{e_2}{y}{e_3} \equiv_\beta e_2[x \mapsto e_1]$$ \\
    $$\infer*[left=sumE]{\infer*[left=sumI${}_1$]{\Gamma \vdash e_1 \colon \tau_1}{\Gamma \vdash \injl{e_1} \colon \sumtype{\tau_1}{\tau_2}}\\ \Gamma, x \colon \tau_1 \vdash e_2 \colon \tau_3\\ \Gamma, y \colon \tau_2 \vdash e_3 \colon \tau_3}{\case{\injl{e_1}}{x}{e_2}{y}{e_3}}$$

\noindent Note: $$C[\vcase{e}] \equiv_\beta \vcase{e}$$
\noindent Note: there's no $\beta$ law for \utype.

\section{$\eta$ law}
$\eta$ Law looks the following way:
$$e \equiv_\eta \abs{x}{(\app{e}{x})}$$

\noindent Every pair of I + E rules constitutes an $\eta$ law.\\
\\
\noindent Arr:
    $$\begin{array}{c|c}
        $$e \equiv_\eta \abs{x}{(\app{e}{x})}$$
        &
        $$\infer*[left=arrI]{ \infer*[left=arrE]{ \Gamma, x \colon \tau_1 \vdash e \colon \tau_1 \to \tau_2\\\infer*[left=Axiom]{ }{\Gamma, x \colon \tau_1 \vdash x \colon \tau_1}}{ \Gamma, x \colon \tau_1 \vdash \app{e}{x} \colon \tau_2}}{\Gamma \vdash \tabs{x}{\tau_1}{(\app{e}{x})} \colon \tau_1 \to \tau_2}$$
    \end{array}$$
\\
\noindent Prod:
    $$\begin{array}{c|c}
        $$e \equiv_\eta \pair{\projl{e}}{\projr{e}}$$
        &
        $$\infer*[left=prodI]{\infer*[left=prodE${}_1$]{\Gamma \vdash e \colon \tau_1 \times \tau_2}{\Gamma \vdash \projl{e} \colon \tau_1}\\\infer*[left=prodE${}_2$]{\Gamma \vdash e \colon \tau_1 \times \tau_2}{\Gamma \vdash \projr{e} \colon \tau_2}}{\Gamma \vdash \pair{\projl{e}}{\projr{e}} \colon \prodtype{\tau_1}{\tau_2}}$$
    \end{array}$$
\\
\noindent Sum:
        $$e \equiv_\eta \case{e}{x}{\injl{x}}{y}{\injr{y}}$$ \\
        $$\infer*[left=sumE]{\infer*[left=sumI${}_1$]{\Gamma \vdash e_1 \colon \tau_1}{\Gamma \vdash \injl{e_1} \colon \sumtype{\tau_1}{\tau_2}}\\ \Gamma, x \colon \tau_1 \vdash e_2 \colon \tau_3\\ \Gamma, y \colon \tau_2 \vdash e_3 \colon \tau_3}{\case{\injl{e_1}}{x}{e_2}{y}{e_3}}$$

\noindent Note: There's no $\eta$ law for \vtype.

\end{document}

%%% Local Variables:
%%% mode: latex
%%% TeX-master: t
%%% TeX-engine: luatex
%%% End: