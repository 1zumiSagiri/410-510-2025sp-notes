\documentclass{lecturenotes}

\usepackage[colorlinks,urlcolor=blue]{hyperref}
\usepackage{doi}
\usepackage{xspace}
% % \usepackage{agda}
% \usepackage{fontspec}
\usepackage{enumerate}
\usepackage{mathtools}
% \setsansfont{Fira Code}
\usepackage{newunicodechar}
\usepackage{mathpartir}
\usepackage[dvipsnames]{xcolor}
\usepackage[many]{tcolorbox}

\usepackage{parskip}

\usepackage{amsfonts}
\usepackage{pl-syntax/pl-syntax}

% % % % %
\newtcolorbox{alert}[3]{
  colback=white,
  colframe={#3},
  fonttitle=\bfseries, 
  title={#1}, 
  width={#2}
}

\newunicodechar{∣}{\ensuremath{\mid}}

\newcommand{\agdanats}{\textsf{ℕ}\xspace}
\newcommand{\funcdef}[4]{#1 \colon #2 #3 #4}
\newcommand{\totfunc}[3]{\funcdef{#1}{#2}{\rightarrow}{#3}}
\newcommand{\parfunc}[3]{\funcdef{#1}{#2}{\rightharpoonup}{#3}}
\newcommand{\sem}[1]{\llbracket #1 \rrbracket}
\newcommand{\lift}[1]{#1_\bot}
\newcommand{\semapp}[2]{\sem{#1}(#2)}
\newcommand{\semappsig}[1]{\semapp{#1}{\sigma}}
\newcommand{\imp}[1]{\textsf{#1}}

\newcommand{\bl}[1]{\textcolor{blue}{#1}}
\newcommand{\g}[1]{\textcolor{Green}{#1}}
\newcommand{\re}[1]{\textcolor{BrickRed}{#1}}
\newcommand{\aq}[1]{\textcolor{Aquamarine}{#1}}
\newcommand{\CC}{\mathbb{C}}
\newcommand{\Tt}[1]{\texttt{#1}}

% % % % % 
\newenvironment{theorem}[2][Theorem]{\begin{trivlist}
\item[\hskip \labelsep {\bfseries #1}\hskip \labelsep {\bfseries #2.}]}{\end{trivlist}}
\newenvironment{lemma}[2][Lemma]{\begin{trivlist}
\item[\hskip \labelsep {\bfseries #1}\hskip \labelsep {\bfseries #2.}]}{\end{trivlist}}
\newenvironment{exercise}[2][Exercise]{\begin{trivlist}
\item[\hskip \labelsep {\bfseries #1}\hskip \labelsep {\bfseries #2.}]}{\end{trivlist}}
\newenvironment{problem}[2][Problem]{\begin{trivlist}
\item[\hskip \labelsep {\bfseries #1}\hskip \labelsep {\bfseries #2.}]}{\end{trivlist}}
\newenvironment{question}[2][Question]{\begin{trivlist}
\item[\hskip \labelsep {\bfseries #1}\hskip \labelsep {\bfseries #2.}]}{\end{trivlist}}
\newenvironment{corollary}[2][Corollary]{\begin{trivlist}
\item[\hskip \labelsep {\bfseries #1}\hskip \labelsep {\bfseries #2.}]}{\end{trivlist}}

\renewcommand{\qedsymbol}{$\blacksquare$}
% % % % %

\title{Simply Typed Lambda Calculus}
\coursenumber{CSE 410/510}
\coursename{Programming Language Theory}
\lecturenumber{14}
\semester{Spring 2025}
\professor{Professor Andrew K. Hirsch}

\begin{document}
\maketitle

\raggedright
 
\section{Motivation of Type Systems}

Previously we have worked with untyped lambda calculus, as well as Imp. 
  Neither had types so why do we need a type system? 

\textbf{Goal of a type system}: Types constrain the ``shapes'' of data. 
  Thus types ensure the programmer has not erred by assuming differently ``shaped'' data than they provide. 

In other words types
\begin{itemize}
  \item Ensure functions are given the expected ``shape'' of data. 
  \item Can warn of mistakes before running or even compiling. 
\end{itemize}

\noindent Note: In this definition we referred to the ``shape'' of data. 
  This is purposefully ambiguous we will get a better sense of how we are using ``shape'' as we continue. 

\section{Types in Lambda Calculus}

So what are the ``shapes'' in lambda calc? 
  Remember $\lambda$ calc is defined just as 

\begin{center}
\begin{alert}{Untyped Lambda Calculus}{0.5\textwidth}{black}
  \begin{syntax}   
    \category[Expressions]{e}
      \alternative{x}
      \alternative{\lambda x:\tau.e}
      \alternative{e_1 e_2}
  \end{syntax}
\end{alert}
\end{center}

All we have to work with are functions, so our ``shape'' must be functions. 
  Therefore we need a type that captures functions. 
  
Let us propose our extension to be:

\begin{center}
\begin{alert}{Simply Typed Lambda Calculus v.1}{0.5\textwidth}{black}
  \begin{syntax}   
    \category[Expressions]{e}
      \alternative{x}
      \alternative{\lambda x:\tau.e}
      \alternative{e_1 e_2}
    \category[Types]{\tau}
      \alternative{A}
      \alternative{\tau_1 \rightarrow \tau_2}
  \end{syntax}
\end{alert}
\end{center}

\textbf{Importance of A}: Imagine a system where the only type is $\tau_1 \rightarrow \tau_2$. 
  This works great to describe a function that takes in a $\tau_1$ and returns a $\tau_2$. 
  But what type does $\tau_1$ have? 
  We can expand $\tau_1$, such that $\tau_1 = \tau_3 \rightarrow \tau_4$. 
  Now we are met with the same problem regarding $\tau_3$. 

$A$ is the key to combat this and allows us to write any types down. 
  $A$ is the nothing type and in fact there are no closed terms that have type $A$. 

Now that we have types let us consider how we utilize them. 

\subsection{Scope}

Recall in a language like OCaml that we attempt to use a variable that has not been declared. 
  The type system raises a flag since the variable is \textbf{not in scope}.

In STLC we will use \textbf{typing contexts} (different from evaluation contexts) as our scopes.

\begin{itemize}
  \item We will refer to them using Gamma ($\Gamma$) or Delta ($\Delta$).
  \item Represents variables that are in scope, with their types.
  \item List of (variable, type) pairs. 
    (Not really lists, since order does not matter. 
    For now think of them as lists or some magic bucket where we can store our pairs.
    Proof that order does not matter is prooved as part of the HW ).
\end{itemize}

Adding the typing contexts to our definition gives us 

\begin{center}
\begin{alert}{Simply Typed Lambda Calculus v.1}{0.5\textwidth}{black}
  \begin{syntax}   
    \category[Expressions]{e}
      \alternative{x}
      \alternative{\lambda x:\tau.e}
      \alternative{e_1 e_2}
    \category[Types]{\tau}
      \alternative{A}
      \alternative{\tau_1 \rightarrow \tau_2}
    \category[Typing Contexts]{\CC}
      \alternative{\cdot}
      \alternative{\Gamma, x : \tau}
  \end{syntax}
\end{alert}
\end{center}

The $\cdot$ is the empty context, where no variable is in scope. 
  Alternatively, $\Gamma, (x : \tau)$ has all of the variables in Gamma in scope, as well as $x$ in scope with type $\tau$. 

\textbf{Note}: From here on out we might use ``context'' without specifying whether it is a type or evaluation context. 
  It should be clear from the context of the use of ``context'' (I am sorry this is hard to word). 

\section{Type Systems}

Finally, with types and contexts we can start implementing our type systems. 

We talk about types using \textbf{judgments}, which we will write $\Gamma \vdash x : \tau$. 
  These judgments should be read ``In the context of $\Gamma$, $e$ has type $\tau$''. 
  Most importantly, the judgment needs all three components, the context ($\Gamma$), turnstile ($\vdash$), and pair of the term and its type ($e: \tau$). 
  Without any part of these the judgment is meaningless.

We define type systems using inference rules. 
  For STLC we have three rules. 

Typing Rules 
  \begin{mathpar}
    \inferrule* [left = T-Var] 
      { x : \tau \in \Gamma}
      { \Gamma \vdash x : \tau}

    \inferrule* [left = T-Abs]
      { \Gamma, x : \tau_1 \vdash e : \tau_2}
      { \Gamma \vdash \lambda x:\tau_1 .e : \tau_1 \rightarrow \tau_2}

    \inferrule* [left = T-App]
      { 
      \Gamma \vdash e_1 : \tau_1 \rightarrow \tau_2 \\ 
      \Gamma \vdash e_2 : \tau_1
      }
      { \Gamma \vdash e_1 e_2 : \tau_2}
  \end{mathpar}
  
\begin{itemize}
    \item[\textsc{T-Var}] This rule types variables. If a variable $x$ has type $\tau$ in the context $\Gamma$, then $x$ has that type $\tau$.  
    
    \item[\textsc{T-Abs}] The abstraction rule allows us to type functions. 
    
    \item[\textsc{T-App}] The application rule allows us to type function application. 
      This restricts what $e_1$ can be. 
      Now if $\Gamma \vdash e_1 : A$ then we cannot use the application rule.  
\end{itemize}

\textbf{Type System and $A$}. $A$ is not used in any of the typing rules. 
  If it did that would mean we could have a closed term of type $A$ or that we could get $\Gamma \vdash x : A$ without using the \textsc{T-Var} rules which would be strange. 

\subsection{Closed Terms}

\textbf{Defn}: As discussed in previous lectures, a term $e$ is closed if and only if $FV(e) = \{\}$. 
  If a term $\cdot \vdash t : \tau$, or term $t$ has type $\tau$ in the empty context, then $t$ is a closed term. (An empty context does not allow for the use of the \textsc{T-Var} rule.)

\begin{theorem}{FV Domain}
  If $\Gamma \vdash e : \tau$ then $FV(e) \subseteq D(\Gamma)$, where $D(\Gamma)$ is the domain of gamma.  $D(\Gamma) = \{x \mid x : \tau \in\Gamma\}$
\end{theorem}

\begin{proof}
  We will prove by induction over the evidence of $\Gamma \vdash e : \tau$   
  
  Proof is left to the reader since each case is simple (Maybe the note writer will be ambitious, but for now nope).
\end{proof}


\begin{corollary}{FV of emtpy context}
  The empty context implies that there are no free variables or $FV(e) = \emptyset$ and thus $e$ is a closed term.     
\end{corollary}


\section{Type Safety}

We have restrained our data ``shapes'', but how can we ensure that we have done so correctly? 
  Type safety says that when we have expect an input of type $x$ and we are given an input of type $y$ (where $ x \neq y$), then we get stuck. 
  So, we want to always be able to take a step or complete our computation. 
  If we can prove this our system has type safety.

\begin{center}
\begin{alert}{Type Safety Proof Plan}{0.6\textwidth}{black}
  Preservation: A well type term steps to a well type term.

  Progress: A term is either a value or can take a step. 

  Type Safety: Progress + Preservation.   
\end{alert}
\end{center}

\begin{lemma}{Preservation of Substitution}
  If $\Gamma \vdash e_1 : \tau_1$ and $\Gamma, x : \tau_1 \vdash e_2 : \tau_2$ then $\Gamma \vdash e_2 [x \mapsto e_1]$. 
\end{lemma}

In other words substitution preserves types. 
  Proof is part of HW 3. 

\subsection{Preservation}

\begin{theorem}{Preservation}
   If $\Gamma \vdash e : \tau$ and $e \rightarrow e'$ then $\Gamma \vdash e' : \tau$. 
\end{theorem}

\begin{proof}
  
 We will prove by induction over the evidence of $\Gamma \vdash e : \tau$.

  There are three rules for how we can type $e$. 
  
  \underline{Case \textsc{T-Var} or \textsc{T-Abs}}: In both cases since the rule cannot take a step preservation is vacuously true. 

  \underline{Case \textsc{{T-App}}}: $e = e_1 e_2$ 
  
    Inductive Hypothesis: If $\Gamma \vdash e : \tau$ and $e \rightarrow e'$ then $\Gamma \vdash e' : \tau$. For $e=e_1$ and $e_2$.

    We must examine three cases for the evidence of $e \rightarrow e'$.

    \begin{enumerate}
      \item $e_1 \rightarrow e_1'$ and $e' = e_1' e_2$

        We know from the IH that $\Gamma \vdash e_1 : \tau_1 \rightarrow \tau_2$, thus we can use the \textsc{T-App} rule to determine that $e_1' e_2: \tau_2$
        \begin{mathpar}
          \inferrule* [left = T-App]
            { 
            \Gamma \vdash e_1' : \tau_1 \rightarrow \tau_2 \\ 
            \Gamma \vdash e_2 : \tau_1
            }
            { \Gamma \vdash e_1' e_2 : \tau_2}
        \end{mathpar} 

    \item $e_2 \rightarrow e_2'$ and $e' = e_1 e_2'$. 

      The same methods for the first case are uses again here. 

    \item $e_1 = (\lambda x.e_1')$ and $e_2$ is a value. 

      Thus we are left with $e' = e_1'[x \mapsto e_2]$, which by the substitutin lemma we know retains its type $\tau_2$. 
    
    \end{enumerate}
   
  Therefore all cases hold and preservation is proven.
\end{proof}

\begin{corollary}{Multi-step Preservation}
  Preservation works across multiple steps. 
\end{corollary}

\subsection{Progress}

\begin{theorem}{Progress}
  If $\cdot \vdash e : \tau$ then either $e$ is a value or there exists $e'$ such that $e \rightarrow e'$. 
\end{theorem}

\begin{proof}
  We will prove by induction over the evidence for typing $\cdot \vdash e : \tau$

  \underline{Case \textsc{T-Var}}: Impossible since the context is empty.

  \underline{Case \textsc{T-Abs}}: $e = \lambda x.e'$ which is a value. 

  \underline{Case \textsc{T-App}}: $e = e_1e_2$ thus we know $\cdot \vdash e_1 : \tau_1 \rightarrow \tau_2$ and $\cdot \vdash e_2 : \tau_1$.

    Inductive Hypothesis: Either $e_1$ is a value or there exists $e_1'$ such that $e_1 \rightarrow e_1'$, similarly for $e_2$. 

    We know by the IH that $e_1$ is a value or can take a step. 
  
    If $e_1$ takes a step then we know $e \rightarrow e_1'e_2$. 

    Now we will assume that $e_1$ is a value. 

    By the IH we know the same is true for $e_2$. 

    When $e_2$ can step we know that $e \rightarrow e_1e_2'$.

    So no we will assume that both $e_1$ and $e_2$ are values. 

    If $e_1$ is a value then it must be that $e_1 = \lambda x. e_1'$ and becuase $e_2$ is a value we can take the step  $(\lambda x. e_1')e_2 \rightarrow e_1' [x \mapsto e_2]$. 

    All cases have been considered and thus progress holds.
\end{proof}

\subsection{Type Safety}

Now we have proved preservation and progress can prove type safety. 


\begin{theorem}{Type System}
  \textbf{Theorem} (Type Safety): If $\cdot \vdash e : \tau $ and $e \rightarrow^* e'$ then $e' \rightarrow e''$ or $e'$ is a value.
\end{theorem} 

\begin{proof}
  Apply multi-step preservation and then progress.   
\end{proof}

\end{document}