\documentclass{lecturenotes}

\usepackage[colorlinks,urlcolor=blue]{hyperref}
\usepackage{doi}
\usepackage{xspace}
\usepackage{agda}
\usepackage{fontspec}
\usepackage{enumerate}
\usepackage{mathpartir}
\setsansfont{Fira Code}
\usepackage{newunicodechar}
\newunicodechar{∣}{\ensuremath{\mid}}

\newcommand{\agdanats}{\textsf{ℕ}\xspace}

\newunicodechar{∣}{\ensuremath{\mid}}
\newunicodechar{′}{\ensuremath{{}^\prime}}
\newunicodechar{ˡ}{\ensuremath{{}^{\textsf{l}}}}
\newunicodechar{ʳ}{\ensuremath{{}^{\textsf{r}}}}
\newunicodechar{ᵣ}{\ensuremath{{}_{\textsf{r}}}}
\newunicodechar{⊗}{\ensuremath{\otimes}}
\newunicodechar{∷}{\ensuremath{\mathrel{::}}}
\newunicodechar{∀}{\ensuremath{\textsf{\forall}}}
\newunicodechar{∎}{\ensuremath{\textsf{\blacksquare}}}
\newunicodechar{ℕ}{\ensuremath{\mathbb{N}}}
\newunicodechar{≤}{\ensuremath{\textsf{\le}}}

\title{Relations}
\coursenumber{CSE 410/510}
\coursename{Programming Language Theory}
\lecturenumber{4}
\semester{Spring 2025}
\professor{Professor Andrew K. Hirsch}

\begin{document}
\maketitle

\begin{center}
\begin{code}
module lec04 where

import Relation.Binary.PropositionalEquality as Eq
open Eq using (_≡_; cong; sym; refl)
open Eq.≡-Reasoning using (begin_; step-≡-∣; step-≡-⟩; _∎)
open import Data.Nat using (ℕ; zero; suc; _+_; _*_)
open import Data.Nat.Properties using
  (+-assoc; +-comm; +-identityˡ; +-identityʳ;
  *-assoc; *-identityˡ; *-identityʳ; *-distribʳ-+)
\end{code}
\end{center}

\section{What is a Relation?}

A relation is describes a property of or relationship between 0 or more members, however, for this class, we will only consider relations that talk about 1 or more members.
We typically first learn about relations from a set theory perspective.
From the set theory perspective, a relation is a set of ordered tuples.
An arbitrary relation $R \subseteq A \times B$ is a set of ordered pairs $(a, b)$.
For example, the less-than-or-equal-to relation on natural numbers is the set of ordered pairs of natural numbers $(x, y)$ such that $x \le y$.
More formally, we say $\le \subseteq \mathbb{N} \times \mathbb{N}$.

For a pair of numbers $x$ and $y$, we say that:

\begin{mathpar}
  x \le y \stackrel{\triangle}{\iff} (x, y) \in \le
\end{mathpar}

We can also define $\le$ over integers and real numbers:

\begin{mathpar}
  \le \subseteq \begin{array}[t]{@{}l@{}}
    \mathbb{Z} \times \mathbb{Z} \\
    \mathbb{R} \times \mathbb{R}
  \end{array}
\end{mathpar}

We can talk about \emph{unary} relations, that takes in only one argument.
Unary relations can be thought of as \emph{predicates}.
For example, consider the unary relation $\text{prime}$.
$\text{prime}$ is a set of natural numbers:

\begin{mathpar}
  \text{prime} \subseteq \mathbb{N}
\end{mathpar}

What does it mean for there to be a natural number $n$ such that $n \in \text{prime}$?
It means that $\text{prime}$ holds for the number $n$, i.e., that $n$ is a prime number.

How do we represent relations in Agda?
We represent them as functions into the \textsf{Set} type.
For example, here is a binary relation in Agda.

\begin{code}
  -- Example of a binary relation in Agda:
  -- R : {A B : Set} -> A -> B -> Set
\end{code}

The relation \textsf{R} takes in two arguments where the first argument has type \textsf{A} and the second argument has type \textsf{B}.
The \textsf{\{A B : Set\}} part tells Agda to infer what the types \textsf{A} and \textsf{B} should be.

In Agda, \textsf{R a b : Set} is evidence that the relation \textsf{R} holds on \textsf{a} and \textsf{b}.

\section{Inductive Relations}

A special kind of relation is called an \emph{inductive} relation.
The evidence of inhabitance in an inductive relation is a tree.
We can define inductive relations using \emph{inference rules}.
Usually, an inference rule is depicted as a number of \emph{hypotheses} or \emph{premises}, followed by a bar below the hypotheses, and then a conclusion below the bar.

Inference rules have the following form:

\begin{mathpar}
  \infer*[left=Rule-name]{H_1 \dots H_n}{P}
\end{mathpar}

Here, $H_1$ through $H_n$ are hypotheses and $P$ is the conclusion.
The way you read an inference rule is if the hypotheses above the line hold, then the conclusion below the line holds.
An example of an inductive relation is the $\le$ relation on natural numbers.
It is defined by two inference rules, where $suc(n)$ means the successor of $n$ (i.e., the next number after $n$):

\begin{mathpar}
  \infer*[left=z\le n]{ }{0 \le n}
  \and
 \infer*[left=s\le s]{n \le m}{suc(n) \le suc(m)}
\end{mathpar}

In the inference rules above, the $\textsc{z\le n}$ rule states that $0$ is less-than-or-equal-to any natural number $n$.
It has no hypotheses, since it is just true that $0$ is less-than-or-equal-to any natural number $n$.
The $\textsc{s\le s}$ rule states that, for any natural numbers $n$ and $m$, if $n \le m$, then it must be true that $n + 1 \le m + 1$.

We can use these inference rules to prove things.
For example, we can prove that $2 \le 3$:

\begin{mathpar}
  \inferrule*[left=s\le s]{
    \inferrule*[left=s\le s]{
    \inferrule*[left=z\le n]{ }{0 \le 1}
  }{1 \le 2}
  }{2 \le 3}
\end{mathpar}

If we start at $2 \le 3$, note that there is only one rule that can be applied, being the $\textsc{s\le s}$ rule.
We apply this rule with a premise of $1 \le 2$, which we must now prove.
Again, only the rule that can be applied is the $\textsc{s\le s}$ rule with a premise of $0 \le 1$.
To prove $0 \le 1$, our only option is to apply the $\textsc{z\le n}$ rule, which completes the proof.

In Agda, we can define $\le$ like this:

\begin{center}
\begin{code}
-- define the less-than-or-equal-to relation as an inductive datatype
data _≤_ : ℕ -> ℕ -> Set where
  z≤n : ∀ {n : ℕ} ->
        zero ≤ n
  s≤s : ∀ {n m : ℕ} ->
        n ≤ m ->
        suc n ≤ suc m

infix 4 _≤_
\end{code}
\end{center}

$\textsf{z\le n}$ and $\textsf{s\le s}$ are the two constructors that can be used to construct evidence for a type $\textsf{n\le m}$.

Let's prove $2 \le 3$ in Agda.
The proof is very similar to the pen-and-paper proof we did above:

\begin{center}
\begin{code}
-- use underscore since don't care about what we call the value here
_ : 2 ≤ 3 -- declaration
_ = s≤s (s≤s z≤n) -- definition
\end{code}
\end{center}

We can also prove that the $\textsf{s\le s}$ inference rule can be inverted.
I.e., if we are given that the conclusion of the rule holds, then so does the premise.

\begin{center}
\begin{code}
-- inversion on the s≤s inference rule
inv-s≤s : ∀ {n m : ℕ} -> suc n ≤ suc m -> n ≤ m
inv-s≤s (s≤s n≤m) = n≤m
\end{code}
\end{center}

Finally, let's define and prove the inductive principle for $\le$.

\begin{center}
\begin{code}
-- inductive principle for ≤
≤-ind : ∀ (P : ℕ -> ℕ -> Set) ->
  (Pz : ∀ (n : ℕ) -> P zero n) ->
  (Ps : ∀ (n m : ℕ) -> n ≤ m -> P n m -> P (suc n) (suc m)) ->
  ∀ (n m : ℕ) -> n ≤ m -> P n m
≤-ind P Pz Ps n m z≤n = Pz m
≤-ind P Pz Ps n m (s≤s Pnm) = Ps _ _ Pnm (≤-ind P Pz Ps _ _ Pnm)
\end{code}
\end{center}

\section{Properties of Relations}

There are several properties that relations can have which are important enough to have names.
There are a number of them that will show up consistently in this course, and which you must have memorized.
Here are most of them:

\begin{center}
\begin{tabular}{|c | c|}
  \hline
  Reflexivity & For every $x$, $x R x$ \\
  \hline
  Symmetry & For every $x$ and $y$, if $x R y$ then $y R x$ \\
  \hline
  Antisymmetry & For every $x$ and $y$, if $x R y$ and $y R x$, then $x = y$ \\
  \hline
  Transitivity & For every $x$, $y$, and $z$, if $x R y$ and $y R z$, then $x R z$ \\
  \hline
  Totality & For every $x$ and $y$, either $x R y$ or $y R x$ \\
  \hline
\end{tabular}
\end{center}

We say that $R$ is reflexive if it has the property of reflexivity, symmetric if it has the property of symmetry, etc.

Different collections of these properties also have names.
In the following table, we say that $R$ is a ...if it has all of the checked properties:

\begin{center}
  \begin{tabular}{|c | c | c | c | c | c|}
    \hline
    Name & Reflexivity & Symmetry & Antisymmetry & Transitivity & Totality \\
    \hline
    Preorder & \checkmark & \times & \times & \checkmark & \times \\
    \hline
    Partial Order & \checkmark & \times & \checkmark & \checkmark & \times \\
    \hline
    Total Order & \checkmark & \times & \checkmark & \checkmark & \checkmark \\
    \hline
    Equivalence & \checkmark & \checkmark & \times & \checkmark & \times \\
    \hline
  \end{tabular}
\end{center}

Note: Every total order is a partial order, and every partial order is a preorder.
Any relation that is both symmetric and antisymmetric can only relate pairs of the form $(x, x)$.
I.e., if $R$ is both symmetric and antisymmetric then for any $x$ and $y$, $x R y$ implies $x = y$.

\end{document}

%%% Local Variables:
%%% mode: latex
%%% TeX-master: t
%%% TeX-engine: luatex
%%% TeX-command-default: "Make"
%%% End: