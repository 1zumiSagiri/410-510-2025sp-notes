\documentclass{lecturenotes}

\usepackage[colorlinks,urlcolor=blue]{hyperref}
\usepackage{doi}
\usepackage{xspace}
\usepackage{agda}
\usepackage{fontspec}
\usepackage{enumerate}
\usepackage{mathpartir}
\setsansfont{Fira Code}
\usepackage{newunicodechar}
\newunicodechar{∣}{\ensuremath{\mid}}
\usepackage{stmaryrd}

\newcommand{\agdanats}{\textsf{ℕ}\xspace}

\newunicodechar{∣}{\ensuremath{\mid}}
\newunicodechar{′}{\ensuremath{{}^\prime}}
\newunicodechar{ˡ}{\ensuremath{{}^{\textsf{l}}}}
\newunicodechar{ʳ}{\ensuremath{{}^{\textsf{r}}}}
\newunicodechar{ᵣ}{\ensuremath{{}_{\textsf{r}}}}
\newunicodechar{⊗}{\ensuremath{\otimes}}
\newunicodechar{∷}{\ensuremath{\mathrel{::}}}
\newunicodechar{∀}{\ensuremath{\textsf{\forall}}}
\newunicodechar{∎}{\ensuremath{\textsf{\blacksquare}}}
\newunicodechar{ℕ}{\ensuremath{\mathbb{N}}}
\newunicodechar{≤}{\ensuremath{\textsf{\le}}}

\title{Axiomatic Semantics}
\coursenumber{CSE 410/510}
\coursename{Programming Language Theory}
\lecturenumber{12}
\semester{Spring 2025}
\professor{Professor Andrew K. Hirsch}

\begin{document}
\maketitle

\section{Axiomatic Semantics}

Axiomatic semantics refers to a way of giving semantics to a programming language by
describing the assertions that hold true about the state of a program before and after executing a command.
Another name for Axiomatic Semantics called Hoare Logic, uses a system called a Hoare Triple in order to portray this idea.
A Hoare triple takes the form: \{ $P$ \} $c$ \{ $Q$ \} , where P and Q are of type: State \rightarrow \; Set.
$P$ and $Q$ represent pre and post conditions, and $c$ represents a command.
If it is a loop instruction, $P$ and $Q$ represent loop invariants and $c$ represents a loop body.
A loop invariant is a condition that holds before and after execution of a loop. \newline

\section{Examples}
1.) (Initializing variables) \newline
\{ true \} (true is a preimposed condition since it must hold prior to running a command) \newline
$x$ := $\tilde{n}$ \newline
\{ $x$ = $n$ \} (this is a post-condition of the previous command, and a pre-condition for the next command) \newline
$y$ := $\tilde{m}$ \newline
\{ $x$ = $n$ \wedge \;$y$ = $m$ \} (this is a post-condition of the previous command, and a pre-condition for the next command) \newline
$z$ := $\tilde{o}$ \newline
\{ $x$ = $n$ \wedge \;$y$ = $m$ \wedge \;$z$ = $o$ \} (this is a post-condition of the previous command) \newline
\newline
2.) (computes the value of $z$ = $x$ * $y$ using a loop, in other words, adding x to z, y times) \newline
\{ true \} (This is a pre-condition as well as a loop invariant) \newline
$x$ := $\tilde{n}$ ; $y$ := $\tilde{m}$ ; $z$ := $\tilde{o}$; \newline
while $o$ < $y'$ do \newline
$z$ := $z$ + $x$ ; \newline
$y$ := $y$ - $\tilde{1}$ \newline
\{ $z$ = $nm$ \} (This is a post-condition as well as a loop invariant) \newline
It is important to note that $z$ = $nm$ may not always be true at any given time, \newline
however it must be true upon entering the loop and exiting it. \newline

\section{Booleans and Hoare Triples}
As mentioned earlier, a Hoare triple takes the form: \{ $P$ \} $c$ \{ $Q$ \} , where P and Q are of type: state \rightarrow \; set.
$P$ and $Q$ represent pre and post conditions, and $c$ is a command. $P$ and $Q$ can be described as predicates of type: state \rightarrow \; set.
We can write boolean expressions about the state, \sigma \; , and its relations to the predicates, $P$ and $Q$. \newline
\newline
$\uparrow$ : Bexpr $\rightarrow$ State $\rightarrow$ Set \newline
($\uparrow$ b) $(\sigma)$ = $\llbracket$ $b$ $\rrbracket$ $(\sigma)$ \newline
\newline
$\uparrow$ takes a boolean expression and gives us back a predicate about states. \newline
\newline
$(P \wedge Q)$ $(\sigma)$ = $P$ $(\sigma)$ $\wedge$ $Q$ $(\sigma)$ \newline
If $P$ and $Q$ hold with respect to the state, $\sigma$, then $P$ is true with respect to $\sigma$ and $Q$ is true with respect to $\sigma$ \newline
\newline
\newpage
\noindent $P \Rightarrow Q$ = $\forall$ $\sigma$. $P$ $(\sigma)$ $\Rightarrow$ $Q$ $(\sigma)$ \newline
$P$ implies $Q$ is equal to saying that for all states $\sigma$, if $P$ is true with respect to a state, $\sigma$, then $Q$ must also be true with respect to $\sigma$ \newline
\newline
$P$ [$x$ $\mapsto$ $f$] $(\sigma)$ $\stackrel{\triangle}{\iff}$ $P$ ($\sigma$ [$x$ $\mapsto$ $f(\sigma)$]) \newline
This shows that if substituting on a predicate and then applying $\sigma$ to it yields the correct result, \newline
then substituting on a $\sigma$ and then applying it to the predicate will also yield the correct result. \newline

\section{Inductive Defenition of Hoare Logic}
Here are the inductive rules for Hoare Logic:
\begin{mathpar}
  \infer*[left=Reasoning]{$R$ \Rightarrow $P$ \\ $Q$ \Rightarrow $S$ \\ \vdash \{ $P$ \} \;c \; \{ $Q$ \}}{\vdash \{ $R$ \} \; $c$ \; \{$S$ \}}
\end{mathpar}

\noindent This rule aims to show that if we have two predicates, $R$ and $Q$ and both hold true, then the predicates they imply, $P$ and $S$, must also be true. \newline
If this is all true, and the Hoare Triple \{$P$\} $c$ \{$Q$\} is provable, then the Hoare Triple \{$R$\} $c$ \{$S$\} must also be provable or else there would be a contradiction. \newline

\begin{mathpar}
  \infer*[left=Skip]{{}}{\vdash \{ $P$ \} \; Skip \; \{ $P$ \}}
\end{mathpar}

\noindent This rule simply shows that if $c$ is a skip command, then the pre-condition and post-condition will be equivalent to each other. \newline

\begin{mathpar}
  \infer*[left=Assgn]{{}}{\{ $P$ \; [$x$ \; \mapsto \llbracket \; $e$ \; \rrbracket] \}$x$ \; := \; $e$ \; \{ $P$ \}}
\end{mathpar}

\noindent This rule states that if the pre-condition is only true after performing the substitution in the command, then the post-condition will be true after the command's execution. \newline

\begin{mathpar}
\infer*[left=Seq]{\vdash \{ $P$ \} \; c_1 \; \{ $Q$ \} \\ \vdash \{ $Q$ \} \; c_2 \; \{ $R$ \}}{\vdash \{ $P$ \} \; c_1 \; ; c_2 \; \{ $R$ \}}
\end{mathpar}

\noindent This rule essentially shows that the transitive property holds for predicates when executing a series of commands. \newline
If the Hoare Triple \{$P$\} $c_1$ \{$Q$\} is provable and the Hoare Triple \{$Q$\} $c_2$ \{$R$\} is provable, \newline
Then the Hoare Triple \{$P$\} $c_1$ ; $c_2$ \{$Q$\} must also be provable. \newline
If the first two Hoare Triples are provable then the third one must be provable because the second Hoare Triple takes the post-condition of the first and uses it as its pre-condition,
so we must be able to use the chain of commands $c_1$ and $c_2$ to skip from $P$ to $R$ instead of having two separate programs.
\newpage

\begin{mathpar}
  \infer*[left=if]{\vdash \{ $P$ \; \wedge \;\uparrow b \} \; c_1 \; \{ $Q$ \} \\ \vdash \{ $P$ \; \wedge \;\uparrow \neg \; b \} \; c_2 \; \{ $Q$ \}}
  {\vdash \{ $P$ \} \; if \; $b$ \; then \; c_1 \; else \; c_2 \; \{ $Q$ \}}
\end{mathpar}

\noindent This rule shows that Hoare Triples are true for if/else statements no matter the value of the boolean. \newline
The fist Hoare Triple covers the case of a true boolean value and the second Hoare Triple covers the case of a false boolean value. \newline
Together they form the third Hoare Triple which uses the assertions of the other two to ensure the pre-conditions and post-conditions will be true before and after execution. \newline

\begin{mathpar}
\infer*[left=while]{\vdash \{ $P$ \; \wedge \;\uparrow b \} \; $c$ \; \{ $P$ \}}{\vdash \{ $P$ \} \; while \; $b$ \; do \; $c$ \; \{ $P$ \; \wedge \;\uparrow \neg \; b \}}
\end{mathpar}

\noindent This rule shows that Hoare Triples are functional for while loops. \newline
The Hoare Triple on top is saying that if $P$ is true and the boolean value $b$ is true then after executing a command, $P$ will still be true. \newline
The bottom Hoare Triple is saying that for a while loop, $P$ will be true before execution and will remain true after execution, but the boolean value will definetly be false after execution, \newline
this makes sense as its the only way to ensure the while loop terminates. \newline

\section{Axiomatic Soundness}

If $\vdash$ \{$P$\} $c$ \{$Q$\} and $P$ $(\sigma)$ \newline
and $\llbracket$ $c$ $\rrbracket$ $(\sigma)$ = $\sigma'$ \newline
Then $Q$ $(\sigma')$ \newline
\newline
The proof is stating that if the Hoare Triple, \{$P$\} $c$ \{$Q$\} is provable and $P$ is true with respect to state $\sigma$, \newline
and after command $c$ is executed on the current state, the state is upated and named $\sigma'$. \newline
Then it must be the case that $Q$ is true with respect to the new state $\sigma'$ or else the Hoare Triple would't be provable. \newline
This fundamentally must be true or else none of the other rules or proofs can be true.
\end{document}

%%% Local Variables:
%%% mode: latex
%%% TeX-master: t
%%% TeX-engine: luatex
%%% TeX-command-default: "Make"
%%% End:
