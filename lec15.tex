\documentclass{lecturenotes}

\usepackage[colorlinks,urlcolor=blue]{hyperref}
\usepackage{doi}
\usepackage{xspace}
% % \usepackage{agda}
% \usepackage{fontspec}
\usepackage{enumerate}
\usepackage{mathtools}
% \setsansfont{Fira Code}
\usepackage{newunicodechar}
\usepackage{mathpartir}
\usepackage[dvipsnames]{xcolor}
\usepackage[many]{tcolorbox}

\usepackage{amsfonts}
\usepackage{pl-syntax/pl-syntax}

% % % % %
\newtcolorbox{alert}[2]{
  colback=white,
  colframe=black,
  fonttitle=\bfseries, 
  title={#1}, 
  width={#2}
}

% % % % %

\newunicodechar{∣}{\ensuremath{\mid}}

\newcommand{\agdanats}{\textsf{ℕ}\xspace}
\newcommand{\funcdef}[4]{#1 \colon #2 #3 #4}
\newcommand{\totfunc}[3]{\funcdef{#1}{#2}{\rightarrow}{#3}}
\newcommand{\parfunc}[3]{\funcdef{#1}{#2}{\rightharpoonup}{#3}}
\newcommand{\sem}[1]{\llbracket #1 \rrbracket}
\newcommand{\lift}[1]{#1_\bot}
\newcommand{\semapp}[2]{\sem{#1}(#2)}
\newcommand{\semappsig}[1]{\semapp{#1}{\sigma}}
\newcommand{\imp}[1]{\textsf{#1}}

\newcommand{\bl}[1]{\textcolor{blue}{#1}}
\newcommand{\g}[1]{\textcolor{Green}{#1}}
\newcommand{\re}[1]{\textcolor{BrickRed}{#1}}
\newcommand{\aq}[1]{\textcolor{Aquamarine}{#1}}
\newcommand{\CC}{\mathbb{C}}
\newcommand{\Tt}[1]{\texttt{#1}}

\title{Basic Type Constructors}
\coursenumber{CSE 410/510}
\coursename{Programming Language Theory}
\lecturenumber{15}
\semester{Spring 2025}
\professor{Professor Andrew K. Hirsch}

\begin{document}
\maketitle
 
\section{Simply Typed Lambda Calculus (STLC)}

In previous lessons we talked about a tiny \textbf{lambda calculus}, a small functional language. 
  From there we expanded lambda calculus to include simple types, giving us \textbf{simply typed lambda calculus}.
  In STLC we discussed progress and preservation, as well as touched more on substitution. 

\subsection{STLC Defined}
Today we are extending STLC to include 4 other types, but before that, let's take stock of the STLC we are working with. 

\begin{center}
\begin{alert}{STLC}{0.7\textwidth}
  \begin{syntax}
    \category[Types]{\tau}
      \alternative{A}
      \alternative{\tau_1 \rightarrow \tau_2}      
    \category[Expressions]{e}
      \alternative{x}
      \alternative{\lambda x:\tau.e}
      \alternative{e_1 e_2}
    \category[Values]{v}
      \alternative{\lambda x:\tau.e}
    \category[Typing Context]{\Gamma}
      \alternative{\cdot}
      \alternative{\Gamma, x : \tau}
    \category[Evaluation Context]{\CC}
      \alternative{[ ~\cdot~ ]}
      \alternative{\CC ~e}
      \alternative{v~\CC}
  \end{syntax}
\end{alert}
\end{center}

Evaluation Rules 
  \begin{mathpar}
    \inferrule* [left = BetaStep]
      { }
      { (\lambda x.e) v \rightarrow e [x \mapsto v] }

    \inferrule* [left = ContextStep]
      { e_1 \rightarrow e_2}
      { \CC~[ e_1 ] \rightarrow \CC~[ e_2 ] }
  \end{mathpar}

Typing Rules 
  \begin{mathpar}
    \inferrule* [left = T-Var] 
      { x : \tau \in \Gamma}
      { \Gamma \vdash x : \tau}

    \inferrule* [left = T-Abs]
      { \Gamma, x : \tau_1 \vdash e : \tau_2}
      { \Gamma \vdash \lambda x:\tau_1 .e : \tau_1 \rightarrow \tau_2}

    \inferrule* [left = T-App]
      { 
      \Gamma \vdash e_1 : \tau_1 \rightarrow \tau_2 \\ 
      \Gamma \vdash e_2 : \tau_1
      }
      { \Gamma \vdash e_1 e_2 : \tau_2}
  \end{mathpar}


\noindent A few notes on this definition that are different from what we have used before. 
\begin{itemize}
  \item Here we are using Chruch style typing so functions (lambdas) now have a type within them. 
    Before we used $\lambda x.e$, now we have $\lambda x:\tau.e$. 
    The $\tau$ is the type of the binding variable $x$, the function itself has its own type.
  \item We are now using evaluation contexts as well. 
    This simplifies the evaluation and typing rules needed for the language. 
    Right now our contexts allow us to evaluate an expression $[~\cdot~]$, it also ensures that when we have application $e_1 e_2$, we evaluate the first term to a value ($\mathbb{C}~e$), before we evaluate the second term ($v~\mathbb{C}$).
  \item Typing rules will all take the form T-Name. 
    Here we have the rule for variables (\textsc{T-Var}), function abstraction (\textsc{T-Abs}), and function application (\textsc{T-App}). 
\end{itemize}

\subsection{Extension Motivation}

Originally we started with untyped or unityped which was great, other than the fact that we were allowed to write programs that could never really run. 
  Then, we added a type system so we could restrict the ``shape'' of the programs that could be written. 
  STLC does this, but we accidentally restricted the programs too much, making our language too weak.
  So weak that STLC is not Turing complete (we will talk more about this in later lectures). 

The solution, add more types. Be expanding the type system of STLC we can \textbf{increase expressivity}, while \textbf{maintaining safety}. 
  Today we will be adding 4 types to STLC, all types that we have seen before, and will see again. 

\section{Unit}

Adding the type $A$ was great to make our simple type system possible, but since we can't have a closed term of type $A$, it is not very useful.
  Instead we will replace it with the type \textit{unit}. 

\begin{center}
\begin{alert}{Extending with Unit}{0.7\textwidth}
  \begin{syntax}
    \category[Types]{\tau}    
      \alternative{\tau_1 \rightarrow \tau_2}
      \alternative{\textit{unit}}
    \category[Expressions]{e}
      \alternative{\dots}
      \alternative{\Tt{()}}
    \category[Values]{v}
      \alternative{\dots}
      \alternative{\Tt{()}}
  \end{syntax}
\end{alert}
\end{center}

Typing Rules 

\begin{mathpar}
  \inferrule* [left = T-Unit] 
    { }
    { \Gamma \vdash \Tt{()}:\textit{unit}}
\end{mathpar}

\noindent\textbf{Note on formatting }

Since we are extending STLC in the interest of keeping these definitions readable and letting us focus on one extension at a time, these new definitions just show the changes. 
  $A$ is the only entry we will be removing so the above Types are all types, for expressions the $\dots ~|~ \Tt{()}$ should be read all all the previous expressions with the addition of $\Tt{()}$. Similarly for type rules, this type rule is now added to our total collection of type rules. 
At the end of this document (\ref{sysdefn}) is the full definition with all addition that is color coded, if you prefer to view it in this format. 
\vspace{0.5cm}

\noindent What is the unit type? 
\begin{itemize}
  \item Unit is the type with \textbf{one value} (In agda it was the type with one constructor).
  \item Unlike $A$ unit has meaning, we can have a closed term with the type \textit{unit}.
\end{itemize}

\noindent \textbf{Theorem:} (Unit Inversion Theorem) If $\Gamma \vdash v : \textit{unit}$ and $v$ is a value, then $v = \Tt{()}$

\noindent \textbf{Proof:} Only one typing rule allows for this to happen, \textsc{T-Unit}. 
  The \textsc{T-Var} and \textsc{T-App} rules do not result in values, and while the \textsc{T-Abs} rule results in a value, it does not have type \textit{unit}. 

  Since only one typing rule works, the value $v = \Tt{()}$.

\section{Product Types}

\noindent Now that we have a more meaningful base type, lets talk about expanding the types available. 
  For simplicity we are only adding pair product types. 
  Triples and beyond can be built out of pairs, so we lose no expressivity and keep our reasoning simpler. 

\begin{center}
\begin{alert}{Extending with Product}{0.7\textwidth}
  \begin{syntax}
    \category[Types]{\tau}   
      \alternative{\dots}
      \alternative{\tau_1 \times \tau_2}
    \category[Expressions]{e}
      \alternative{\dots}
      \alternative{ (e_1,e_2) }
      \alternative{\pi_1~e}
      \alternative{\pi_2~e}
    \category[Values]{v}
      \alternative{\dots}
      \alternative{(v_1, v_2)}
    \category[Evaluation Context]{\CC}
      \alternative{\dots}
      \alternative{(\CC,e)}
      \alternative{(v,\CC)}
      \alternative{\pi_1~\CC}
      \alternative{\pi_2~\CC}
  \end{syntax}
\end{alert}
\end{center}

\noindent Evaluation Rules 
  \begin{mathpar}
    \inferrule* [left = ProjL] 
      { }
      { \pi_1 (v_1, v_2) \rightarrow v_1 }

    \inferrule* [left = ProjR] 
      { }
      { \pi_2 (v_1, v_2) \rightarrow v_2 }
  \end{mathpar}

\noindent Typing Rules 
  \begin{mathpar}
    \inferrule* [left = T-Pair] 
      { 
      \Gamma \vdash e_1 : \tau_1 \\ 
      \Gamma \vdash e_2 : \tau_2 
      }
      { \Gamma \vdash (e_1,e_2): \tau_1 \times \tau_2 }

    \inferrule* [left = T-ProjL] 
      { \Gamma \vdash e : \tau_1 \times \tau_2 }
      { \Gamma \vdash \pi_1 ~ e : \tau_1 }

    \inferrule* [left = T-ProjR] 
      { \Gamma \vdash e : \tau_1 \times \tau_2 }
      { \Gamma \vdash \pi_2 ~ e : \tau_2 }
  \end{mathpar}

\noindent Notes on the definition
\begin{itemize}
  \item In expressions we have the pair $(e_1,e_2)$ and then projection, or proj, left and right. 
  \item The value of product types is the first value we have seen where every part is also a value.
  \item Similarly to how we evaluate function application, we use our contexts to dictate that we evaluate the left part of a pair $(\CC,e)$ before the right $(v,\CC)$. 
  \item Our contexts also allow us to evaluate what we are projecting $(\pi_1~\CC)$. 
\end{itemize}

\noindent \textbf{Theorem:} (Product Inversion Theorem) If $\Gamma \vdash v : \tau_1 \times \tau_2$ and $v$ is a value, then $v = (v_1,v_2)$.

\noindent \textbf{Proof:} Similarly with the unit inversion theorem there is only one typing rule allows for this to happen.

\vspace{0.25cm}

Why do we need inversion theorems? We organized our language such that the type system correctly restrains our new types, and this theorem is our proof.
  Later we will use these theorems to prove preservation. 
  
\section{Sum Types}

Now we have products, so we should add their dual, Sum types. 
  In Agda we used the $\uplus$ symbol for sum types. 
  Sum types are also sometimes referred to as Co-Products, but we will try to stay consistent with Sum types.

  
\begin{center}
\begin{alert}{Extending with Sum}{0.7\textwidth}
  \begin{syntax}
    \category[Types]{\tau}  
      \alternative{\dots}
      \alternative{\tau_1 + \tau_2}
    \category[Expressions]{e}
      \alternative{\dots}
      \alternative{\Tt{inj}_1~e}
      \alternative{\Tt{inj}_2~e} \\
      \alternative{\Tt{case}~e_1~\Tt{of}~\Tt{inj}_1~x \Rightarrow e_2; ~\Tt{inj}_2~y \Rightarrow e_3 ~\Tt{end}}
    \category[Values]{v}
      \alternative{\Tt{inj}_1~v}
      \alternative{\Tt{inj}_2~v}
    \category[Evaluation Context]{\CC}
      \alternative{\dots}
      \alternative{\Tt{inj}_1~\CC}
      \alternative{\Tt{inj}_2~\CC} \\ 
      \alternative{\Tt{case}~\CC~\Tt{of}~\Tt{inj}_1~x \Rightarrow e_2; ~\Tt{inj}_2~y \Rightarrow e_3 ~\Tt{end}}
  \end{syntax}
\end{alert}
\end{center}

\noindent Evaluation Rules 
  \begin{mathpar}
    \inferrule* [left = injL] 
      { }
      { ( \Tt{case}~\Tt{inj}_1v~\Tt{of}~\Tt{inj}_1~x \Rightarrow e_2; ~\Tt{inj}_2~y \Rightarrow e_3 ~\Tt{end})\rightarrow e_1 [x \mapsto v] }

    \inferrule* [left = injR] 
      { }
      { ( \Tt{case}~\Tt{inj}_2v~\Tt{of}~\Tt{inj}_1~x \Rightarrow e_2; ~\Tt{inj}_2~y \Rightarrow e_3 ~\Tt{end})\rightarrow e_2 [y \mapsto v] }
  \end{mathpar}

\noindent Typing Rules 
  \begin{mathpar}
    \inferrule* [left = T-InjL] 
      { \Gamma \vdash e : \tau_1}
      { \Gamma \vdash \Tt{inj}_1~e : \tau_1 + \tau_2}

    \inferrule* [left = T-InjR] 
      { \Gamma \vdash e : \tau_2}
      { \Gamma \vdash \Tt{inj}_2~e : \tau_1 + \tau_2}

    \inferrule* [left = T-Case] 
      { 
      \Gamma \vdash e_1 : \tau_1 + \tau_2  \\ 
      \Gamma, x : \tau_1 \vdash e_2 : \tau \\
      \Gamma, y : \tau_2 \vdash e_3 : \tau 
      }
      { \Tt{case}~e_1~\Tt{of}~\Tt{inj}_1~x \Rightarrow e_2; ~\Tt{inj}_2~y \Rightarrow e_3 ~\Tt{end} : \tau }
  \end{mathpar}

\noindent Notes on the definition
\begin{itemize}
  \item Products are pairs, but sum types are like an ADT with two branches. 
    Importantly a sum type has two options, a value can only be one of those options, and we can tell which option it is. 
  \item We read $\Tt{inj}$ as ``inject''.
  \item Case can be confusing, it is best explained with the \textsc{InjL} and \textsc{InjR} rules. 
    When we do case analysis we start with an expression, $e$, that has a product type. 
    We determine if $e$ is an $\Tt{inj}_1v_1$ or a $\Tt{inj}_2v_2$ and we then we select the appropriate branch. I have rewritten the case statement below, just adding new lines and indents, to make it similar to OCaml's match statements, which I think helps conceptually. 

    $
    \begin{array}{l}
         \Tt{case}~e_1~\Tt{of} \\ 
         ~~~~~\Tt{inj}_1~x \Rightarrow e_2; \\
         ~~~~~\Tt{inj}_2~y \Rightarrow e_3 ~\Tt{end} \\
    \end{array}
    $

  \item Look at the \textsc{T-InjL} rule. Where did the type $\tau_2$ come from and what it is?
    The answer is $\tau_2$ is any type. 
    We are not concerned with what type it is since we know the $\Tt{inj}_1v$ we have is of type $\tau_1$. (If you tell me I am holding an integer or some other type and I have an integer, do I really care what this other type is?)
  \item The evaluation contexts we defined allow us to only evaluate the branch we want to take in a case statement. 
    We can only take on branch, so it make sense to only evaluate the one we will need. 
\end{itemize}

\noindent \textbf{Theorem:} (Sum Inversion Theorem) If $\Gamma \vdash v : \tau_1 + \tau_2$ and $v$ is a value, then $v = \Tt{inj}_1v_1 : \tau_1$ or $v = \Tt{inj}_2v_2 : \tau_2$.

\noindent \textbf{Proof:} Only two typing rules allow for this scenario. 
  Either way the result is trivial, using the appropriate left or right injection rule. 

\section{Void}

Our final type is similar to the $A$ we removed, but not quite the same.
  Next we will add the \textit{void} type.

\begin{center}
\begin{alert}{Extending with Void}{0.7\textwidth}
  \begin{syntax}
    \category[Types]{\tau}    
      \alternative{\dots}
      \alternative{\textit{void}}
    \category[Expressions]{e}
      \alternative{\dots}
      \alternative{\Tt{case}~ e ~\Tt{of end}}
    \category[Evaluation Context]{\CC}
      \alternative{\dots}
      \alternative{\Tt{case}~ \CC ~\Tt{of end}}
  \end{syntax}
\end{alert}
\end{center}

Typing Rules 

\begin{mathpar}
  \inferrule* [left = T-Void]
    { \Gamma \vdash e : \textit{void}}
    { \Gamma \vdash \Tt{case}~ e ~\Tt{of end} : \tau}
\end{mathpar}

\noindent Notes on the definition
\begin{itemize}
  \item Again we are using the case of syntax, except here there are no branches. 
    In the sense that Sum type was an ADT with two constructors , void is an ADT with no constructors. 
  \item The typing rule \textsc{T-Void} is the ex-falso idea we talked about previously. 
    If you give me the type \textit{void} I can return to you any type I like.
\end{itemize}

\noindent \textbf{Theorem:} There is no closed value $v$ such that $\cdot \vdash v : \textit{void}$. 

We did not discuss a formal proof, but did mention that there is no constructor to create a void. 
  You cannot create a void in our language so how can you have a value of type \textit{void}. 

\vspace{0.25cm} \noindent \textbf{Void vs $A$}

  At surface value the type \textit{void} and the type $A$ that we removed when adding \textit{unit} seem similar. 
    Neither can have closed terms and sort of serve as the ``nothing'' typing. 
  
  However \textit{void} is intentional and we have rules for how we handle receiving a void (although we never expect to). 
    $A$ however just served to let us write down function types, otherwise we couldn't write any terms in STLC.

\section{Conclusion }

We extended the type system of STLC to include unit, product, sum, and void types. 
  We now have increased expressivity, but have maintained our safety. (Well the safety is inferred by our inversion theorems, we will probably be proving that safety in later lessons). 

Overall, we are much closer to a practical language such as OCaml with these new types. 

\newpage

\subsection{Full System Definition} \label{sysdefn}

The additions are color coded as \re{Unit}, \g{Product Types}, \bl{Sum Types}, and \aq{Void}


\begin{center}
\begin{alert}{STLC w/Basic Types Constructors}{0.85\textwidth}
  \begin{syntax}
    \category[Types]{\tau}
      \alternative{\tau_1 \rightarrow \tau_2}
      \alternative{\re{\textit{unit}}}
      \alternative{\g{\tau_1 \times \tau_2}}
      \alternative{\bl{\tau_1 + \tau_2 }}
      \alternative{\aq{\textit{void}}}
      
    \category[Expressions]{e}
      \alternative{x}
      \alternative{\lambda x:\tau.e}
      \alternative{e_1 e_2}

      \alternative{\re{\Tt{()}}}

      \alternative{\g{(e_1,e_2)}}
      \alternative{\g{\pi_1~e}} 
      \alternative{\g{\pi_2~e}} 

      \alternative{\bl{\Tt{inj}_1~e}} \\ 
      \alternative{\bl{\Tt{inj}_2~e}}
      \alternative{\bl{\Tt{case}~e_1~\Tt{of}~\Tt{inj}_1~x \Rightarrow e_2; ~\Tt{inj}_2~y \Rightarrow e_3 ~\Tt{end}}} \\ 

      \alternative{ \aq{\Tt{case}~ e ~\Tt{of end}}}
      
    \category[Values]{v}
      \alternative{\lambda x:\tau.e}

      \alternative{\re{\Tt{()}}}
      
      \alternative{\g{(v_1, v_2)} }
      
      \alternative{\bl{\Tt{inj}_1~v}}
      \alternative{\bl{\Tt{inj}_2~v}}
      
    \category[Typing Context]{\Gamma}
      \alternative{\cdot}
      \alternative{\Gamma, x : \tau}
      
    \category[Evaluation Context]{\CC}
      \alternative{[ ~\cdot~ ]}
      \alternative{\CC ~e}
      \alternative{v~\CC}

      \alternative{\g{(\CC,e)}}
      \alternative{\g{(v,\CC)}}
      \alternative{\g{\pi_1~\CC}}
      \alternative{\g{\pi_2~\CC}} 

      \alternative{\bl{\Tt{inj}_1~\CC}} \\ 
      \alternative{\bl{\Tt{inj}_2~\CC}}
      \alternative{\bl{\Tt{case}~\CC~\Tt{of}~\Tt{inj}_1~x \Rightarrow e_2; ~\Tt{inj}_2~y \Rightarrow e_3 ~\Tt{end}}} \\ 

      \alternative{\aq{\Tt{case}~ \CC ~\Tt{of end}}}
  \end{syntax}
\end{alert}
\end{center}

Evaluation Rules 
  \begin{mathpar}
    \inferrule* [left = BetaStep]
      { }
      { (\lambda x.e) v \rightarrow e [x \mapsto v] }

    \inferrule* [left = ContextStep]
      { e_1 \rightarrow e_2}
      { \CC~[ e_1 ] \rightarrow \CC~[ e_2 ] }

    \g{
    \inferrule* [left = ProjL] 
      { }
      { \pi_1 (v_1, v_2) \rightarrow v_1 }
    }

    \g{
    \inferrule* [left = ProjR] 
      { }
      { \pi_2 (v_1, v_2) \rightarrow v_2 }
    } 

    \bl{
    \inferrule* [left = injL] 
      { }
      { ( \Tt{case}~\Tt{inj}_1v~\Tt{of}~\Tt{inj}_1~x \Rightarrow e_2; ~\Tt{inj}_2~y \Rightarrow e_3 ~\Tt{end})\rightarrow e_1 [x \mapsto v] }
    }

    \bl{
    \inferrule* [left = injR] 
      { }
      { ( \Tt{case}~\Tt{inj}_2v~\Tt{of}~\Tt{inj}_1~x \Rightarrow e_2; ~\Tt{inj}_2~y \Rightarrow e_3 ~\Tt{end})\rightarrow e_2 [y \mapsto v] }
    }
  \end{mathpar}

Typing Rules 
  \begin{mathpar}
    \inferrule* [left = T-Var] 
      { x : \tau \in \Gamma}
      { \Gamma \vdash x : \tau}

    \inferrule* [left = T-Abs]
      { \Gamma, x : \tau_1 \vdash e : \tau_2}
      { \Gamma \vdash \lambda x:\tau_1 .e : \tau_1 \rightarrow \tau_2}

    \inferrule* [left = T-App]
      {
      \Gamma \vdash e_1 : \tau_1 \rightarrow \tau_2 \\ 
      \Gamma \vdash e_2 : \tau_1 
      }     
      { \Gamma \vdash e_1 e_2 : \tau_2}   
      
    \re{ 
    \inferrule* [left = T-Unit] 
      { }
      { \Gamma \vdash \Tt{()}:\textit{unit}}
    }

    \g{
    \inferrule* [left = T-Pair] 
      { 
      \Gamma \vdash e_1 : \tau_1 \\ 
      \Gamma \vdash e_2 : \tau_2 
      }
      { \Gamma \vdash (e_1,e_2): \tau_1 \times \tau_2 }
    } 

    \g{
    \inferrule* [left = T-ProjL] 
      { \Gamma \vdash e : \tau_1 \times \tau_2 }
      { \Gamma \vdash \pi_1 ~ e : \tau_1 }
    }

    \g{
    \inferrule* [left = T-ProjR] 
      { \Gamma \vdash e : \tau_1 \times \tau_2 }
      { \Gamma \vdash \pi_2 ~ e : \tau_2 }
     }

    \bl{
    \inferrule* [left = T-InjL] 
      { \Gamma \vdash e : \tau_1}
      { \Gamma \vdash \Tt{inj}_1~e : \tau_1 + \tau_2}
    }

    \bl{
    \inferrule* [left = T-InjR] 
      { \Gamma \vdash e : \tau_2}
      { \Gamma \vdash \Tt{inj}_2~e : \tau_1 + \tau_2}
    }

    \bl{
    \inferrule* [left = T-Case] 
      { 
      \Gamma \vdash e_1 : \tau_1 + \tau_2  \\ 
      \Gamma, x : \tau_1 \vdash e_2 : \tau \\
      \Gamma, y : \tau_2 \vdash e_3 : \tau 
      }
      { \Tt{case}~e_1~\Tt{of}~\Tt{inj}_1~x \Rightarrow e_2; ~\Tt{inj}_2~y \Rightarrow e_3 ~\Tt{end} : \tau }
    }

    \aq{
    \inferrule* [left = T-Void]
      { \Gamma \vdash e : \textit{void}}
      { \Gamma \vdash \Tt{case}~ e ~\Tt{of end} : \tau}
    }
  \end{mathpar}
  
\end{document}
