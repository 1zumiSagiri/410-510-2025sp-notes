\documentclass{lecturenotes}

\usepackage[colorlinks,urlcolor=blue]{hyperref}
\usepackage{doi}
\usepackage{xspace}
\usepackage{agda}
\usepackage{fontspec}
\usepackage{enumerate}
\setsansfont{Fira Code}
\usepackage{newunicodechar}
\newunicodechar{∣}{\ensuremath{\mid}}

\newcommand{\agdanats}{\textsf{ℕ}\xspace}


\title{Naturals and Induction}
\coursenumber{CSE 410/510}
\coursename{Programming Language Theory}
\lecturenumber{2}
\semester{Spring 2025}
\professor{Professor Andrew K. Hirsch}

\begin{document}
\maketitle

We start by importing some things from the standard library.
These will be explained later.
\begin{center}
\begin{code}
  module lec02 where
  
  import Relation.Binary.PropositionalEquality as Eq
  open Eq using (_≡_; cong; sym; refl)
  open Eq.≡-Reasoning using (begin_; step-≡-∣; step-≡-⟩; _∎)
\end{code}
\end{center}

\section{The Natural Numbers}
\label{sec:natural-numbers}

Agda is at the same time much more powerful and much more bare bones than languages that you've seen before.
A lot of the things you're used to being included in a language has to be defined in Agda.
This includes things as simple as numbers!

We start with the simplest form of number: the natural numbers.
Recall: $$\mathbb{N} = \{0, 1, 2, \dots\}$$
In this class, and in programming-languages as a field, the natural numbers always include $0$.

We can build up the natural numbers as follows: $0$ is a natural number, and for every natural number~$n$, $n + 1$ is a natural number.
Hence $1 = 0 + 1$ is a natural number, $2 = 1 + 1$ is a natural number, and so on.

This sounds a lot like an OCaml variant (or, in general, an Algebraic Data Type from functional programming).
In fact, we can encode this exactly!
In Agda, we would write the following:
\begin{center}
\begin{code}
  data ℕ : Set where
    zero : ℕ
    suc : ℕ -> ℕ
\end{code}
\end{center}
Here, we're defining an algebraic data type called \agdanats.
This is our type of natural numbers.
Note the use of unicode: this is very common in Agda!

We start a definition of a new ADT with the keyword \textsf{data}, followed by the name of this new type.
This is similar to \textsf{type ℕ = \dots} in OCaml.
However, in Agda, we have to do more: we must tell Agda what the type of our new type is!
In OCaml, we never have to do this, but Agda's much more powerful type system requires us to.
We tell it that \agdanats has type \textsf{Set}, the type of Agda types.
We'll see what other options exist later in the course.

After the \textsf{where}, we find a list of constructors, all indented the same amount.
Note that Agda, unlike OCaml, is whitespace-sensitive.
Each constructor then has a type, which must be a function into \agdanats.
The constructor \textsf{zero}, representing $0$, carries no data, and so has no arguments.
The constructor \textsf{suc}, on the other hand, represents ${} + 1$.
Thus, $1 = \textsf{suc zero}$, $2 = \textsf{suc (suc zero)}$, and so on.
We thus need a natural-number argument for \textsf{suc}.

Nobody wants to write \textsf{suc (suc (suc zero))} for three every time they use it, so Agda gives us the ability to use normal numbers to represent natural numbers.
We do this using the following \emph{pragma} (command to the Agda compiler):
\begin{center}
\begin{code}
  {-# BUILTIN NATURAL ℕ #-}
\end{code}
\end{center}
This also has the pleasant side effect of speeding up several operations that we will define on \agdanats.

\pagebreak
As in OCaml, we can write code involving natural numbers using recursion.
For instance, we could write a program that \textsf{double}s its input as follows:
\begin{center}
\begin{code}
  double : ℕ -> ℕ
  double zero = zero
  double (suc n) = suc (suc (double n))
\end{code}
\end{center}
We start by describing the type of \textsf{double}: it is a function that takes a \agdanats to an \agdanats.
In OCaml, we usually allow the inference algorithm to figure this out for us; in Agda, we should always give the type of a function.
We then give the definition of the function using pattern-matching clauses.
This is similar to \textsf{function} in OCaml: rather than naming an argument and then immediately using something like \textsf{match}, we only give the patterns for the argument.
In Agda, we do this by giving the name of the function along with patterns for every argument, followed by the body of the function \emph{for that pattern match}, repeatedly.
In this case, we have two patterns: \textsf{zero} and \textsf{suc n}.
If the input is \textsf{zero}, we simply return \textsf{zero}, since twice zero is zero.
If the input is \textsf{suc n}, we double \textsf{n}, and then we add two successors to it, effectively doubling \textsf{suc n}.

In order to write \textsf{double}, we take advantage of Agda's emacs mode.
We start by writing a single clause of the definition, with a single variable, and a question mark as the definition.
$$
\begin{array}{l}
  \textsf{double : ℕ -> ℕ}\\
  \textsf{double n = ?}
\end{array}
$$
We then load the file, and the question mark turns into a \emph{hole}.
Agda tells us that this hole has type \textsf{ℕ}.
$$
\begin{array}{l}
  \textsf{double : ℕ -> ℕ}\\
  \textsf{double n = 0\{~\}}
\end{array}
$$
We can then \emph{split} this definition by putting the cursor in the hole and typing \textsf{C-c~C-c} (i.e., \textsf{Ctrl + c} twice).
It then asks us for a variable to split on; if we type \textsf{n} and enter, it turns into the following:
$$
\begin{array}{l}
  \textsf{double : ℕ -> ℕ}\\
  \textsf{double zero = 0\{~\}}\\
  \textsf{double (suc n) = 1\{~\}}\\
\end{array}
$$
where we now have two holes, one for each case.
We can then put our cursor in the first hole, type \textsf{zero} and \textsf{C-spc} to fill in the hole.
After filling in both holes, we get the answer we were looking for.
The Agda mode can give you a lot of information about this hole, including all of the variables that are in context and their types.
You should read through \href{https://agda.readthedocs.io/en/latest/tools/emacs-mode.html}{the emacs mode docs} (\url{https://agda.readthedocs.io/en/latest/tools/emacs-mode.html}).

Mathematically, we would describe this function differently than in either OCaml or Agda:
$$
\textsf{double}(\textsf{n}) = \left\{\begin{array}{ll}
  \textsf{zero} & \text{if}~\textsf{n}=\textsf{zero}\\
  \textsf{suc (suc n)} & \text{if}~\textsf{n}=\textsf{suc n}
\end{array}\right.
$$
This is \emph{definition by parts}, which you might be familiar with from e.g., calculus class.

In OCaml, we can define operators using parentheses, so \textsf{(+)} would be an infix addition operator.
Agda gives a much more powerful operator definition mechanism, which allows symbols, words, and arguments to be intermixed almost arbitrarily.
For instance, we can define the infix addition operator as follows:
\begin{center}
\begin{code}
  _+_ : ℕ -> ℕ -> ℕ
  zero + m = m
  suc n + m = suc (n + m)

  {-# BUILTIN NATPLUS _+_ #-}
\end{code}
\end{center}

This defines addition recursively by parts.
Note that in the pattern-matching clauses, we use + infix, as we want to use it later in the code.
Here, we match on the first argument.
Thus, this is equivalent to the following mathematical definition by parts:

$$
n + m = \left\{\begin{array}{ll}
  m & \text{if}~ n = \textsf{zero}\\
  \textsf{suc}~(n' + m) & \text{if}~n=\textsf{suc}~n'
\end{array}
\right.
$$

The \textsf{BUILTIN} pragma tells Agda to use a more-efficient binary representation when running \textsf{\_+\_}.
That way, we can ``pretend'' that it has this simpler unary definition, while running code that uses a more-efficient binary representation.
Don't worry, Agda checks to make sure that your code is correct before it lets you use the \textsf{BUILTIN} pragmas.

We can similarly define multiplication:\\
\begin{minipage}{0.5\linewidth}
\begin{code}
  _*_ : ℕ -> ℕ -> ℕ
  zero * m = zero
  suc n * m = m + (n * m)

  {-# BUILTIN NATTIMES _*_ #-}
\end{code}
\end{minipage}
\begin{minipage}{0.5\linewidth}
  $$
  n * m = \left\{\begin{array}{ll}
    \textsf{zero} & \text{if}~n=\textsf{zero}\\
    m + (n' * m) & \text{if}~n=\textsf{suc}~n'
  \end{array}\right.
$$
\end{minipage}

\section{Equality in Agda}
\label{sec:equality-agda}

We have mentioned several times that Agda's type system is significantly more powerful than OCaml's.
In particular, Agda's type system is \emph{dependent}.
Dependent types allow programs to appear inside of types.
The first dependent type we're going to see is equality.

In Agda, \textsf{t ≡ u} is a type for any \textsf{t} and \textsf{u} that have the same type.
We program with this type a little differently than with other types that we've seen before.
(N.B.: Right now, I'm lying a bit to you.
You can program with this type just like it's any other type.
We're going to go with the lie for now.)
In particular, you build programs of type \textsf{t ≡ u} by building chains of equalities.

Let's start by proving something that we already know: doubling $1$ gives us $2$.
\begin{center}
\begin{code}
  _ : double 1 ≡ 2
  _ =
    begin
      double 1
    ≡⟨⟩
      double (suc zero)
    ≡⟨⟩
      suc (suc (double zero))
    ≡⟨⟩
      suc (suc zero)
    ≡⟨⟩
      2
    ∎
\end{code}
\end{center}
We start with \textsf{double 1}, which is one side of our equality.
We then \emph{unfold} the definition of \textsf{1}, which is \textsf{suc zero}.
Then we get to apply the pattern-matching rules for double, which in this case allow us to ``move the call to \textsf{double} past the \textsf{suc}, doubling it along the way.''
After applying the double pattern-matching rule for zero (which just gets rid of the call to \textsf{double}), we end up with \textsf{suc (suc zero)}.
We can then \emph{fold} that definition again, ending up with just \textsf{2}.

\pagebreak
In plain math, we reason exactly the same way:
\begin{thm}
  $\textsf{double 1} = \textsf{2}$
\end{thm}
\begin{proof}
  $$
  \begin{array}{ll}
    \textsf{double}~1 &= \langle\text{definition}\rangle\\
    \textsf{double (suc zero)} &= \langle\text{computing}\rangle\\
    \textsf{suc (suc (double zero))} &= \langle\text{computing}\rangle\\
    \textsf{suc (suc zero)} &= \langle\text{definition}\rangle\\
    2
  \end{array}
  $$
\end{proof}
Notice that every line is a term that is equal to the one before it, and we always give a reason why these terms are equal.
In Agda, if that reason is ``obvious,'' then you don't need to give such a reason.
``Obvious'' here means that after we compute the terms as much as possible, they end up being the same.
In fact, that's already true of \textsf{double 1} and \textsf{2}: they are the same after you run the computation \textsf{double 1}!
We can thus give a simpler proof:
\begin{center}
\begin{code}
  _ : double 1 ≡ 2
  _ = refl
\end{code}
\end{center}
Here, \textsf{refl} is an Agda expression that says that any two terms that are the same (after computing) are equal.

Things get more complex once you have something other than a constant in the equality.
Agda lets you prove things for \emph{all} numbers, not just for specific numbers.
This acts like a function (in fact, it \emph{is} a function): we get an input number~$n$, and need to provide a proof for that $n$.
\begin{center}
\begin{code}
  oneplus_suc : ∀ (n : ℕ) -> 1 + n ≡ suc n
  oneplus_suc n =
    begin
      1 + n
    ≡⟨⟩
      (suc zero) + n
    ≡⟨⟩
      suc (zero + n)
    ≡⟨⟩
      suc n
      ∎

  double_suc : ∀ (n : ℕ) -> double (suc n) ≡ 2 + double n
  double_suc n =
    begin
      double (suc n)
    ≡⟨⟩
      suc (suc (double n))
    ≡⟨⟩
      suc (suc (zero + double n))
    ≡⟨⟩
      suc (1 + double n)
    ≡⟨⟩
      2 + double n
    ∎
\end{code}
\end{center}
In both of these cases, we get a number as input and then run a chain of reasoning in order to provide a proof of the equality for that number.

\pagebreak
\section{Induction}
\label{sec:induction}

Recall the principle of induction: to prove that some property $P$ holds of all natural numbers, it suffices to prove
\begin{enumerate}[(a)]
\item $P(0)$ holds, and
\item if $P(n)$ holds, then $P(n + 1)$ holds.
\end{enumerate}

We can turn this into a \emph{proof template}.
If you are lost when writing a proof, it can be useful to apply this proof template.
\begin{thm}
  For any $n \in \mathbb{N}$, $P(n)$ holds.
\end{thm}
\begin{proof}
  We proceed by induction on $n$.\\
  \textbf{P =} $P$

  \noindent\textbf{Base case: $n = 0$:}\\
  \dots
  
  \noindent\textbf{Inductive Case: $n = \textsf{suc}~m$:}\\
  \textbf{IH: $P(m)$}\\
  \textbf{To Show: $P(\textsf{suc}~m)$}\\
  \dots
\end{proof}

In order to understand how to do induction in Agda, it helps to understand what is going on here.
Remember that a forall statement is like a function.
If you claim that $P$ holds for any $n$, then if I give you a number~$n$, you better be able to explain why $P(n)$ holds.
One way you can do this is by recursion: ``well, $P$ holds of three because $P(2)$ holds, which holds because $P(1)$ holds, which holds because $P(0)$ holds, which holds because ....''
This is exactly what induction is doing!
Let's see this in practice, starting with a paper-and-pencil proof.
\begin{thm}
  For any $n \in \mathbb{N}$, $n + 1 = \textsf{suc}~n$.
\end{thm}

Note that this is different from asking $1 + n = \textsf{suc}~n$ which holds by computation.
In that case, the first argument to \textsf{\_+\_} is $1$, so we know which branch of the pattern match to take.
In this case, the first argument is $n$, and so we don't know which branch of the pattern match to take.

\begin{proof}
  We proceed by induction on $n$.\\
  \textbf{P(n) = $n + 1 = \textsf{suc}~n$}

  \noindent \textsf{Base Case: $n = 0$}\\
  If $n = 0$, then we take the first branch of the pattern match.
  Thus, we just get $1$, but $1 = \textsf{suc}~0 = \textsf{suc}~n$.

  \noindent \textbf{Inductive Case: $n = m + 1$}\\
  \textbf{IH: $m + 1 = \textsf{suc}~m$}\\
  \textbf{To Show: $(\textsf{suc}~m) + 1 = \textsf{suc}~(\textsf{suc}~m)$}\\
  $$\begin{array}{l}
    (\textsf{suc}~m) + 1 =\{\text{computation}\}\\
    \textsf{suc}~(m + 1) =\{\text{IH}\}\\
    \textsf{suc}~(\textsf{suc}~m)
  \end{array}$$
\end{proof}

\pagebreak
In Agda, this would look like the following:
\begin{center}
\begin{code}
  plusone-suc : ∀ (n : ℕ) -> n + 1 ≡ suc n
  plusone-suc zero = refl
  plusone-suc (suc n) =
    begin
      (suc n) + 1
    ≡⟨⟩
      suc (n + 1)
    ≡⟨ cong suc (plusone-suc n) ⟩
      suc (suc n)
    ∎
\end{code}
\end{center}
Where we had simply written ``IH'' before, we've now written a recursive call.
However, note the use of \textsf{cong}.
If we remove this, Agda will complain that $\textsf{plusone-suc}~n$ does not prove that $\textsf{suc}~(n+1) = \textsf{suc}~(\textsf{suc}~n)$; instead, it only proves $n + 1 = \textsf{suc}~n$.
However, we are applying the same function to two equal inputs; we should get equal outputs.
That's exactly what \textsf{cong~suc} tells Agda: we're feeding the \textsf{suc} function two equal inputs, so the outputs should be equal as well.



\end{document}

%%% Local Variables:
%%% mode: latex
%%% TeX-master: t
%%% TeX-engine: luatex
%%% TeX-command-default: "Make"
%%% End:
